\chapter*{Appendix}
\addcontentsline{toc}{chapter}{Appendix}
\section*{List of appendices}
\vspace{-8em}
\listofanhang
\clearpage
\spezialkopfzeile{Appendix} % damit in der Kopfzeile das Wort "Anhang" angezeigt wird

\anhang{Interviews}\label{Interviews}

\lstset{language=TeX, % hervorzuhebende Keywords definieren
  morekeywords={anhang, anhangteil}
}


% Um den Anforderungen der Zitierrichtlinien nachzukommen, wird das Paket \verb|tocloft| verwendet. Jeder Anhang wird mit dem (neu definierten) Befehl \lstinline|\anhang{Bezeichnung}| begonnen, der insbesondere dafür sorgt, dass ein Eintrag im Anhangsverzeichnis erzeugt wird. Manchmal ist es wünschenswert, auch einen Anhang noch weiter zu unterteilen. Hierfür wurde der Befehl \lstinline|\anhangteil{Bezeichnung}| definiert.

% In~\ref{anhang:abbildung} finden Sie eine bekannte Abbildung und etwas Source Code in~\ref{anhang:sourcecode}. 

\anhangteil{Interview with Daniel Kaltenbach}\label{anhang:InterviewDK}
This is the transcription of the complete interview between Daniel Kaltenbach (Interviewee, DK) and the author (Interviewer, LM). The interview was conducted on 22th of March, 2018 from 9:00 to 9:30 in person in the HPE office in Böblingen. 

\begin{xlist}
     \item[LM] Ok, jetzt haben wir das Interview gestartet. Noch eine Sache ganz vorweg, ist es okay, wenn ich deinen Namen in der Bachelorarbeit nenne oder möchtest du es lieber anonym machen?
     \item[DK] Lass uns das am Ende klären.
     \item[LM] Das war immer so am Anfang, also bei den Ausführungen in den Büchern ist immer noch: Ja, das muss man einmal festhalten. Aber erstmal etwas über dich! Was ist denn deine Rolle in HPE?
     \item[DK] Ja, also mein Name for the record. Ich heiße Daniel Kaltenbach und bin bei HPE  verantwortlich für das Thema IOT. Ich bin für Deutschland, Österreich und Schweiz sowie in Russland der designierte IOT Verantwortliche der eben die Kooperationsstrategie dieses Themas vorantreibt und eine cross-funktionale Inkubationsorganisation führt um das Thema IOT für HPE greifbar zu machen, Portfolio, Partnerschaften und strategische Initiativen zu treiben, zu verantworten und nachher zu Marktkapitalisierung zu führen.
     \item[LM] Ok, und wer zählt so zu deinen Kunden? Oder hast du direkt Kontakt zu Kunden?
     \item[DK] Ich habe ganz viel direkten Kontakt mit Kunden. Ich habe aber keine direkten eigenen Kunden, also ich bin ja kein Account Manager oder so, weil ich vertrete das Thema im Namen der HPE und damit ist mein Kunde die Geographie Deutschland, Österreich, Schweiz und Russland.
     \item[LM] Okay, worauf ich eher damit hinaus wollte, so was was macht ihr denn für eure Kunden? Also welche Angebote und Dienstleistungen von IOT her bietet ihr an?
     \item[DK] Enorm unterschiedlich. Ich nutze ganz gern die Analogie, die die PDC geprägt hat. Wenn sie IOT definieren, dann geht es im Prinzip um intelligente Produkte: Die Kaffeemaschine, die nach hause telefoniert und weiß, wann sie kaputt gehen wird. Wohlgleich das zweite Thema des  intelligenten Prozesses, wo es eben um horizontale Integration von Prozessketten geht, die schlussendlich in der Summe intelligenter werden, oder nicht. Also wenn ich weiß, was eben nach mir passiert, dann kann ich möglicherweise nacharbeiten was davor nicht so gut gelaufen ist. Ich kann die Qualität im Gesamtprozess betrachten, den Durchlauf optimieren. Wenn ich  weiß, dass der vor mir fertig ist kann die nachfolgende Maschine schon einmal anlaufen und sich aufwärmen, weil sie ja gleich einen Prozess bekommen wird et cetera. Das kann aber auch mein Customer Journey sein, der durch ein Retail läuft und eben, wenn jemand rein läuft dann ist es wichtig, dass ein Mitarbeiter hinten weiß, da ist jemand reingelaufen, damit ich ihn möglicherweise richtig ansprechen kann. Solche Dimensionen, also intelligentes Produkt oder intelligenter Prozess und darin beraten wir und liefern jetzt die Technologien, die helfen diese Daten von diesen Informationen, die da sind, Herr zu werden und darauf basierend intelligente Entscheidungen zu treffen.
     \item[LM] Ok dann würde ich jetzt auch schon direkt zum nächsten Teil gehen. Zum Thema: das Problem beim Verständnis von Blockchain. Ganz allererst: Was verstehst du genau unter Blockchain und was stellst du dir unter der Funktionsweise davon vor?
     \item[DK] Ich stelle mir erstmal vor, dass es etwas unheimlich Kompliziertes ist und ich glaube ich nehme damit teilweise schon die Frage vorweg was das Problem dadran ist. Ich glaube das ist Teil des Problems. Also es ist eine Technologie ja erstmal, die eben ähnlich einer Datenbank funktioniert aber eben dezentral die Daten hält und gegenseitig autorisiert. Das würde ich jetzt mal so wie als eine Blockchain beschreiben, wie eine Art verteiltes Dateisystem, verteiltes Datenbanksystem, das eben aber ein in sich, durch die Verteilung notwendiges Problem löst, eben die Autorisierung der Daten und der Richtigkeit. Wahrscheinlich medium richtig, aber das war jetzt zumindest mein Verständnis.
     \item[LM] Das reicht ja auch und wie hast du dir… also woher hast du jetzt dieses Verständnis?Also hast du das von vom Hörensagen, von irgendwelchen Artikeln oder wie hat sich das so zusammen geflochten?
     \item[DK] Mischung von allem. Ich glaube, dass warum es noch so medium richtig ist, liegt daran dass ich mich einfach nie selbst mit dem Thema intensiv auseinandergesetzt habe und einfach mal wirklich hinter das Thema gestiegen bin und deshalb hier die Arbeit auch ganz gut, das ist was, was die Industrie an der Stelle momentan nicht schafft: Eine sinnvolle Artikulation der Technologie und damit verbunden der Anwendungsszenarien. Also die Übersetzungen von einfachen menschen, wie mir hin zu „ Wir verstehen die Technologie und wir benutzen sie für dich“. Dazu ist mein Wissen nicht ausreichend und das schafft die Industrie auch nicht. Also mein Wissen basiert auf unterschiedlichsten Gesprächen, Konferenzen und ich war durchaus mal bei dem ein oder anderen Blockchain-artigen Workshop, wo durchaus nochmal so Grundlagen waren, aber immer wenn ich es gehört habe war es immer irgendwie nochmal ein bisschen neu und ein bisschen anders. Und einer hatte das so und der andere wieder so und ich bin halt nicht Techniker, ich mache Vertrieb am Ende des Tages.
     \item[LM] Ja das ist auch je nachdem wo man im Internet guckt, gibt es ganz viele verschiedene Ansätze das zu erklären. Ich versuche eben auch einmal die zu konsolidieren und zu schauen, dass man einen Ansatz hat der das ganzheitlich aber auch relativ einfach erklärt. Und wieso noch einmal meintest du denn, dass du dich eben nicht weiter damit auseinandergesetzt hast? Also du meintest ja so: Ja es ist ein kompliziertes Thema; hat dich das einfach davon abgeschreckt oder war es die fehlende Zeit? Was waren so die Gründe, dass du jetzt noch nicht da groß hinter stehst. 
     \item[DK] Zeit und Relevanz. Also wenn ich mich mit jeder Technologie irgendwie auseinandersetzen würde, die irgendwie IOT relevant ist, dann würde ich wahrscheinlich die nächsten zehn Jahre studieren und hätte es dann erst recht nicht gelernt, weil in den zehn Jahren wieder neue Sachen draußen sind. Also ich muss natürlich extrem selektiv wahrnehmen und das tue ich auch. Ich verstehe und das ist glaube ich mehr dieses HörenSagen dass es eine signifikante Implikation haben kann für bestimmte Anwendungsszenarien. Ich habe bisher für mich noch nicht den Bedarf verspürt mich intensiver damit auseinanderzusetzen, weil immer dann wenn so eine Diskussion aufgekommen wäre oder hätte kommen können, habe ich mir einfach einen Experten dazu geholt. Das bringt meine Rolle glücklicherweise mit, dass ich darin nicht Experte sein muss.
     \item[LM] Ja du meintest gerade, dass es schon Implikationen für IOT/deinen Bereich hat. Hast du da irgendwelche Beispiele im kopf, wo du sagen könntest: Ja Blockchain könnte da und da bedeutend sein? Also eher für deinen Bereich, jetzt nicht auf Finanzderivate bezogen.
     \item[DK] Ja, du wirst jetzt unfair. Nenee, ich glaube es ist ein gute Frage! Ich habe da jetzt natürlich irgendwie 20 Beispiele im Kopf, die irgendwie so allgemein diskutiert werden. Aber wenn ich jetzt so darüber nachdenke, kann ich jetzt nicht mit Bestimmtheit sagen, dass ich weiß, dass dieser UseCase oder dieser Anwendungsfall wirklich idealtypisch für eine Blockchain-Implementierung ist, weil ich auch weiß Blockchain ist eine Technologie, ne. Es gibt auch noch mögliche andere Technologien, die möglicherweise diesen Use Case genauso gut oder besser gar bedienen können. Von daher kann ich dir jetzt gerade keinen idealtypischen Fall nennen, weil mir wahrscheinlich auch die Technologie zu unbekannt ist.
     \item[LM] Darauf möchte ich auch eigentlich gar nicht hinaus, also es soll jetzt nicht DER UseCase sein, wo du denkst, auf jeden Fall muss da Blockchain hin. Sondern es geht auch gerade um dieses HörenSagen, was die meisten anderen Leute so als UseCases nennen. Ich würde gerne von dir wissen, was du denn so dazu zählst.
     \item[DK] Ja, ähm, alles was natürlich in den Bereich Monetarisierung geht, wo man eben den Mediator ausschalten möchte, also wie eine Bank oder so, wo man ein aufwändiges Transaktionssystem  benötigen würde. Wo eben bei Maschinen mit Maschinen Payments irgendwie ablaufen, würde ich jetzt mal so zu diesen Daten-Monetarisierungsbereich und Maschinen zu Maschinen Payments und natürlich das Thema Smart Contracting in der zweiten Dimension. Wo eben Maschinen miteinander verhandeln und so sich schließlich einigen auf Arbeitsausführungen und ja dann das Payment möglicherweise als zusätzliche notwendige Komponente mit einfließen lassen aber nicht notwendigerweise. Das wären jetzt so die zwei schwergewichtigsten. Aber ich bin wahrscheinlich zu weit weg um sagen zu können, das wäre perfekt oder so.
     \item[LM] Das ist ja alles gut und musst du auch gar nicht. Gut, zu der Frage wieso das so schwer zu verstehen ist haben wir ja schon festgehalten, dass es eben eine komplizierte Technologie ist. Meinst du da gibt es noch andere Punkte, dass sich eben die Leute/die Personen/potenzielle Kunden in deinem Gebiet sich noch nicht so sehr mit dem Thema auseinandergesetzt haben oder dieses Halbwissen haben, woran das liegen könnte?
     \item[DK] Mehrere Antworten. Weiterführend zu dem Thema "komplizierte Technologie" ist glaube ich der typischerweise  (und ich glaube das ist in allen Technologien so) beginnen über die Technologie zu sprechen, dann ist das immer ein Technologen-Kreis. Und ein Mensch/ein Typ Mensch, der natürlich hochtechnisch ist, wenn der dir versucht nachher, also ich meine dir und mir, ich will dir jetzt nichts unterstellen, aber zumindest mir, der einfache Mensch, dann versucht etwas kompliziertes zu erklären dann muss der immer fachlich total richtig sein. Wenn Physiker, der dir gegenüber sitzt und dir die Welt erklärt. Die Welt ist manchmal ganz einfach. Einen Fuss vor den anderen setzen, bewegst dich. Ja das ist für den nicht so einfach, das sind Energie, Bewegungsabläufe, da sind Erdanziehung, schlag mich tot. So ist es hier auch ein bisschen. Ich glaube, es ist noch nicht die Community oder die Technologiegruppe nenne ich sie mal, jetzt mal irgendwie verdreht, dass sie noch nicht in der Lage  sind eine einfache Sprache zu finden, als sie sich selbst einfach noch so Technologie-verliebt sind und den Transfer in die einfache Menschenwelt nicht schaffen. Nummer 1, also die Gruppe der Menschen, die damit beschäftigt sind. Die zweite Dimension ist wir leben momentan, glaube ich jetzt in so einer Technologiespirale in der einfach enorm viele Techniken gerade unterwegs sind, die irgendwie alle auf uns einpressen. Sagen wir mal 5G als Kommunikationsstandard, das ist künstliche Intelligenz, 3D Druck, es gibt irgendwie 50 solcher Dinger. Und die menschen sind in Summe noch in so einer Abwartephase. Welche von den Dingern werden jetzt revolutionsartig meine Welt verändern, mit welchen muss ich mich auseinandersetzen? Das sind so viele, dass wir eine Überforderung haben an Dingen, die die Menschen typischerweise nicht verstehen, weil sie einfach kompliziert sind. Weil das eben die Evolution von Technik ist. Ich glaube, das sind so die zwei Dimensionen. Also, in der Gruppe der Technologen ist man noch nicht bereit eine einfache Sprache zu wählen, weil das technische Thema einfach noch zu technisch ist. Oder man sich aber so sehr die Technik differenziert. Und gleichwohl bewegen wir uns in einem Umfeld, wo Technologien so rapide da sind und auch wieder weggehen, dass die Leute sich nicht genug damit auseinandersetzen zu können.
     \item[LM] Vielen Dank. Jetzt zu der nächsten Frage, die war: So wir haben jetzt schon gesagt, ja welche möglichen Konzepte bei IOT für Blockchain gibt es. Denkst du auch, dass das für HPE und die Kunden interessant sein könnte, das mal einzuführen für ein paar UseCases?
     \item[DK] Technologie ist ein Vehikel. Und Blockchain ist an der Stelle ein Vehikel, weil es eine Datenbank ist, und Technik ist ein Vehikel, die am Ende des Tages Mehrwerte schaffen müssen.  Und für Kunden sind Mehrwerte relevant, wir müssen Business Outcomes mit denen diskutieren, Gründe, Verbesserungen des Produktes, für Qualität, wie waren wir beim Ausbringen und mehr ausbringen, Kosten reduzieren oder ganz klassisch und wissensfragen und sobald wir es schaffen zu übersetzen, warum Blockchain da einen Beitrag leisten kann oder ein auf Blockchain basierendes Anwendungsszenario, das deswegen komplizierter ist aber Blockchain als ein technologisches Element dadrin hat. Dann sind die hoch zufrieden damit und momentan sind wir noch in so einer Phase, wo die Leute geil finden, dass Blockchain da drin ist. Das keinen Mehrwert hat, weil es halt geil ist Blockchain gemacht zu haben. Der Sache wegen geil. Aber ich glaube, das kühlt sich in der nächsten, das dauert gar nicht mehr lange, in den nächsten sechs Monaten wird sich das noch ein bisschen regen und dann wird der Markt da sein, verstanden haben: Okay, ist es ein Werkzeug in einem großen Werkzeugkasten und setzt das ein bitte. Also ihr macht das, ihr löst das IT Problem da. Das ist offensichtlich eins der Werkzeuge, sowie beim Handwerker, mir ist dann vollkommen egal ob der einen großen Schraubenzieher oder kleinen hat, solange die Schraube eingedreht wird. Schraube in der Wand. Im Grunde: Reduktion von Kosten und wenn Blockchain das Thema dafür ist, dann geil machst du das. Aber ich glaube, das ist nicht momentan, sechs Monate noch.
     \item[LM] Glaubst du denn, dass wenn man dem Kunden dann erklärt wie die Blockchain funktioniert, ob das einen Einfluss darauf hat? Also so wie du es anläuten ließest war es ja eher nicht der Fall. Also dass er sagt, Blockchain für mich egal wie das funktioniert, Hauptsache es funktioniert. Oder denkst du auch, dass er daran interessiert sein könnte zu wissen, wie es funktioniert? 
     \item[DK] Mehrere Stufen. Zuerst interessiert ihn, was ist der Business Outcome. Und natürlich will er irgendwann verstehen, ist dieses Angebot, dass ich von dir bekomme, liebes Technologieunternehmen,  hat das Substanz? Wie ist das umgesetzt? Hat das, möglicherweise (weil wir erklären natürlich ihm die Straße nach nirgendwo oder zu seinem Business Problem). Aber er hat ein Ökosystem, in das er es einbetten muss. in ein Business Continuity System, das heißt er muss verstehen was passiert, wenn diese Lösung oder diese Technik von dir ausfällt. Gibt es ein Backup und Disasterplan, was wollen wir nicht. Und wir werden in diesem Angebot, in dieser Erstdiskussion, werden wir natürlich alle Aspekte erläutern. Das heißt in dem Moment, wo er eben tiefer einsteigen möchte oder wir bewiesen haben, ja wir können ihr technisches Problem lösen mit Technologie. Würde er fragen: Welche Technologie verwendest du und was sind dafür die Disasterprozesse? Was denn, wie sicher ist das? Ist die Technik morgen weg? Also gibt es ein Substitut?Was auch immer,  da wird es sicherlich eine Diskussion geben und dann ist die Technik wichtig, wenn dann die Technologie Blockchain eine entsprechende Kredibilität erreicht, die ihm nützt, dann absolut. Es gibt Technologien von der wir heute sagen würden, wir machen den Prozess und der Kunde guckt dahinter und du würdest sagen: Und ich liefere Ihnen dann eine DVD. Dann würde er tot umfallen. Wahrscheinlich wenn ich eine Kassette noch liefern würde, noch zweimal tot umfallen und sagen: Ihr seid kein kredibles Unternehmen, weill sie keine Technologien nehmen, die nicht mehr State of the art ist, und eben kein zukünftiges ordentliches Substitut mehr hat oder keine Wiedersupply hat. Von daher hat es da sicherlich eine Relevanz, wenn sich die Technik durch setzt, aber die Technologie alleine macht keinen Mehrwert.
     \item[LM] Ja, da stimme ich auch zu. Mhmm und woran meinst du liegt es, dass Blockchain momentan so ein Hype ist? Beziehungsweise ich empfinde das so,  weil ich mich mehr damit auseinandersetze, aber AI ist ja auch ein großes Thema. Woran glaubst du liegt das?
     \item[DK] Schwierig, ich glaube da gibt es jetzt wieder 1000 Theorien oder wir können über die mediale Diskussion darüber reden,  so Fake News. Es kann sein, dass es jemand möchte, dass es ein total wichtiges Thema ist und gesteuert ist deswegen, Boah Verschwörungstheorien. Aber ich weiß es nicht, um ehrlich zu sein. Ich glaube durchaus, dass da eine Dimension drin ist, dass die Technik ziemlich gut ist.  Also ich glaube, das ist durchaus ein technologischer Durchbruch, der Eigenschaften mitbringt, die tatsächlich einige Anwendungsszenarien revolutionär sein können. Sonst hätte sich wahrscheinlich dieser Hype auch nicht so lange gehalten und zumindest bei uns in der Technologie-Community würde sich das nicht so lange halten, aber ich glaube da sind wir zu kritisch als Industrie. Von daher gehe ich mal schwer davon aus, dass da ein revolutionsartiges Element drin ist, das eben ganz, ganz besonders ist und uns deswegen weit nach vorne bringen kann. Ich kann jetzt auch nicht sagen, dass es genau diese eine Sache ist, die uns nach vorne gebracht hat. 
     \item[LM] Okay, aber Dankeschön. Und dann kommen wir jetzt auch schon zum dritten Teil: Was sollte denn in dieser Anwendung drin sein? Wie stellst du dir eine optimale Erklärung von Blockchain vor? Auf welchem Sprachniveau, weil wir das eben schon angeschnitten haben, und welche Konzepte?
     \item[DK] Also ich halte, im Grunde jetzt geht es um die Lerntheorie. Ich halte interaktive lernetheoretische Grundlagen als Basis. Ich glaube das sind die, die möglichst viele Sinne stimulieren, weil die für die Neulinge gut sind. Deshalb sind diese Erklärvideos immer sehr gut, wenn die dann auch eine Überleitung haben in einen praktischen Teil. So okay, jetzt mach das mal selbst. Oder für Erfahrene irgendwie selbst, hat das den höchsten Mehrwert. Ich glaube, weil es so ein hochtechnisches Thema ist, müssen wir unterschiedliche Ebenen bedienen und wir müssen das wahrscheinlich, halt mich daran jetzt nicht fest, in drei Ebenen diskutieren. Das sollte auf einer sehr abstrakten Art und Weise sein, dann muss es was eher technologisch vergleichend sein. Also mann muss sagen wie grenze ich mich gegen Technik links von mir und rechts von mir ab.  Keine Ahnung, welche die jetzt sind. Vielleicht Datenbank, etwas anderes,  verteiltes Dateisystem oder keine Ahnung, weiß ich jetzt nicht. Abgrenzen. das was nicht ist und dann muss es natürlich eine Erklärung der Technik zu der technischen Art und Weise haben, die einfach ausreichend einfach ist. Also nicht ein 10 Seiten Papier mit 0en und 1en, wo ich das erstmal mehr in Sprachtext übersetzen muss, sondern, dass - sage ich mal - in einer Art und Weise konsumierbar ist, dass ich wenn ich eine Viertelstunde aufgeregt lese und interessiert lese und dann sage: ja habe ich verstanden. Ich glaube, das sind mehrere Elemente, die wir schlussendlich brauchen, aber ich glaube dass, die größte Hürde bis heute ist der abstrakte Teil, das ganz oben, der möglicherweise auch diesen vergleichenden Teil drin hat. Weil das die Sprache ist, die mir noch fehlt. Ich habe mir eine Handvoll von diesen Erklärvideos angeschaut, immer mal wieder. Aber sie waren irgendwie nicht schlüssig, die waren nicht differenzierend genug,  und dass ich das Gefühl hatte, das erklärt mir jetzt gerade alles und nichts. 
     \item[LM] Gut, bei den nächsten Fragen wollte ich eigentlich darauf eingehen, welche Komponenten ich bei Blockchain. erklären sollte, aber von deinem Verständnis her... Blocks, weißt du, kennst du denn die Begriffe: Private Key, Public Key, Blocks, Hashfunktionen, kannst du dir darunter etwas vorstellen? Oder weißt du, dass das etwas mit Blockchain zu tun hat und hättest die Dinge auch gerne erklärt? Oder reicht da nur anschneidend?
     \item[DK] Ja genau das meine ich mit kaskadierend. Wir müssen, du musst das auf einer Ebene erklären, wo du diese Begrifflichkeiten nicht benötigst. Und natürlich, weil ich meine es gibt tonnenweise Informationen zu Public Key/Private Key Geschichten, aber ich glaube das musst du einweben. Und du musst eine Geschichte bauen, dass am Ende des Tages klar wird, was diese Technologie hat und das ist genau der Punkt: es nutzt diese Technologie (public und private key) für Security Geschichten, das ist jetzt nicht das einzige. Weil jetzt wieder ich als geneigter Nichtversteher sage mir: Gut, die haben jetzt das eine verwendet, aber wahrscheinlich gibt es da draußen 1000, und warum ist das jetzt genau das Richtige und warum ist das das Beste? Ist das substituierbar? Wenn jetzt so übermorgen jemand sagt, die Public Private Key Geschichte ist total hackbar, es ist falsch! Ist dann Blockchain per se auch falsch oder also es ist eine notwendige Verbindung oder eine Kann-Verbindung? Also das ist einfach zu viel für mich. Für mich, unpräzise oder unspezifisch, aber ich sage, weil das muss nur eben kaskadierend aufgebaut werden. Du musst erst ganz ganz oben anfangen. Erklären, was ist die Vision dahinter, was ist das Thema das du baust. Dann muss man eine Ebene tiefer gehen, also was sind die Bausteine technologischer Art und dann musst du erklären warum sind das genau diese und auch wieder abgrenzend zu links und rechts. Und dann sagen: Okay, jetzt erkläre ich es euch eine Ebene tiefer. Ich glaube dieses kaskadierende Modell baut Verständnis auf, in der jeder entscheiden kann, wann er austritt.  Für meine Rolle wäre es wahrscheinlich notwendig, alle Ebenen einmal durchgemacht zu haben. Sobald ich feststelle, dass es von relevanter Größe für die Industrie ist. Aber im Moment ist es, wen ich das Oberste haben möchte, dann würde ich die sprache von A, wenn ich das Unterste haben möchte die Sprache von Z und die zwei passen aber nicht zusammen. Also es gibt kein durchgängiges Erklärsystem, weil es einfach noch keiner gemacht hat, deshalb finde ich die Arbeit gut. 
     \item[LM] Dankeschön, ich hoffe das auch. Perfekt, die nächste Frage war nämlich: Wie sollte die Erklärung am besten aufgebaut sein? Das haben wir schon erledigt. Meinst du denn jetzt, wenn ich diese Erklärung habe von abstrakt runter und zurück... Noch einmal im Bereich Kundenprojekte. Ob man das dann dort auch anwenden sollte? Um zu sehen, also dass die Anwendung, die ich jetzt gestalte, auch beim Kunden gezeigt werden kann um mit ihm gemeinsam zu erarbeiten was Blockchain denn jetzt eigentlich bedeutet.
     \item[DK] Okay, ich habe mich jetzt gerade gefragt... Wenn ich jetzt wieder zwei Schritte zurücknehme, da müsste ich eigentlich sagen: Nein. Den Kunden interessiert Blockchain nicht, den Kunden interessieren Business Outcomes. So habe ich das gesagt. Aber du hast die Frage richtig gestellt. Um ein besseres Business Outcome mit ihm zu erarbeiten. Erarbeiten ist das wichtige dadrin.  Wir sind heute, glaube ich nicht in der Dimension oder nicht in der Marktsituation, wo wir ein "Friss"-Ergebnis hinlegen und der Kunde frisst am Ende des Tages. Wir sind nicht in dem Commodity-Markt und wir wissen, wenn ich ein Stückchen Hühnchen hole, dann frage ich nicht wo das herkam,  und wie es zerlegt wurde, weil ich gehe davon aus, dass es wenn es hinter dem Tresen liegt und den Bio-Stempel hat, dann war die Kühlkette sauber, was auch immer. So sind wir noch nicht. Das heißt, wir sind im komplexen Bereich. Also wir werden immer diese Ebene benötigen, wo ich erkläre und wir sind auch in einem sehr kooperativem Umfeld. Wir erarbeiten Lösungen mit Kunden und um eine Lösung zu erarbeiten müssen wir ihm eben Bausteine geben,  die er versteht. Und das ist genau meine Aufgabe. Und wenn wir da genau die richtige Sprache treffen. Wenn wir dem Kunden vermitteln, dass wir seine Sprache sprechen und das ist genau das, was wir hier bauen. Wir müssen seine Sprache sprechen um ihnen zu erklären, was der technologische Mehrwert dieser Technik ist um dabei ein Business Problem zu lösen. Wenn wir das sauber erklären können, dann bauen wir damit eine hohe Kredibilität auf, ein hohes Vertrauen und damit die Möglichkeit mit ihm was weitergehendes zu machen. Blockchain erklären kann theoretisch jeder, nur wir können es halt besser und wir treffen seine Sprache. Und damit sind wir der vertrauensvollere Anbieter wie irgendjemand anderes.
     \item[LM] Ok, perfekt, vielen Dank. Das war es auch schon. Wir sind tatsächlich sehr viel schneller durch gekommen, als ich gedacht habe, aber vielen Dank, du hast wirklich alles gut beantwortet und damit kann ich auch jetzt gut arbeiten.
     \item[DK] Abschließend: Ja, du darfst meinen Namen verwenden, sonst haben wir es wieder vergessen. 
\end{xlist}

\anhangteil{Interview with Ralph Beckmann}\label{anhang:InterviewRB}


