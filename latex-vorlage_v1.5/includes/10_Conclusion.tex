\chapter{Conclusion} \label{chap:conclusion}

This chapter summarizes the purpose of the thesis and its key findings, reflects critically upon the chosen research methods and their execution, before providing an overview of the paper's contribution to practice and research. To conclude, this chapter presents a final outlook which describes possible future fields of research. 

\section{Purpose of the thesis} \label{sec:findings}
\textit{What and how should the artifact visualize information, so it explains the blockchain technology in a comprehensible way to a novice audience?} This is the leading question this piece of research has tried to resolve. By first introducing blockchain technology and placing it at the intersection of financial, economic and \ac{IT} aspects, called cryptoeconomics, the paper argues that the majority of novices have difficulty in understanding blockchain because existing learning resources do not cover the broader concepts of cryptoeconomics in an easy to understand manner. The paper is based on Mayer's cognitive theory of multimedia learning, which states that presenting the information in the form of interactive multimedia instructions enhances meaningful learning. Because of this, the paper argues that an interactive explanation of the blockchain technology, with various graphics and animated visualizations, is better suited to introduce a novice audience to the technology than existing multimedia resources such as YouTube videos, articles, or other visualizations. To resolve the research question, the paper follows the design science research process, defining a solution space based on the identified problem space and then building an instantiation of this solution which is finally demonstrated and evaluated to prove the solution's feasibility. This paper should be understood as the culmination of the research process (the communication of research),\footnote{For more information, see \ref{subsec:GuidelinesDesignScience}} to provide an answer (which is both rigorously researched and relevant to practice) to the research question in the form of a developed and instantiated artifact, an interactive multimedia learning resource for a novice audience. 

\section{Key findings}
To define the solution space, interviews are conducted which focus on two groups of questions (1) \enquote{Where lies the problem of understanding the blockchain technology?}, and 2) \enquote{What information should an optimal blockchain explanation include and how should it be structured?)}.
By conducting these interviews with experts of various fields of work, the research reveals that a clear and simple-to-understand introduction to this topic is missing in the industry.\footcite[Cf.][]{DanielKaltenbach_Interview} The experts also state that an interactive learning experience with simple and pithy real-world examples would be more beneficial to novices than a detailed technical introduction.\footcite[Cf.][]{BjoernPaulewicz_Interview} While the internet provides access to millions of different YouTube videos and articles regarding blockchain technology, there are only a few resources that provide an engaging and interactive learning experience. However, even these visualizations, presented in chapter \ref{sec:ExistingSolutions}, are not appropriate for a novice audience interested in blockchain.  They are either focused on a specific blockchain implementation such as bitcoin or they are focused on the technical details and fail to present the learner to the overall value of the technology. Based on the existing solutions and insights from the expert interviews, the paper specifies which requirements an appropriate solution must meet. These include requirements regarding the content of the explanation, its graphical design, its structure, as well as requirements regarding the technology.

The subsequently designed and developed artifact, an interactive learning resource in the form of a website (using \ac{HTML}, \ac{CSS}, and \ac{JS}), takes all these requirements into consideration. It is divided into two parts, so that the user may choose whether to follow a tutorial or to see an overview of the blockchain technology. The tutorial takes place as a guided tour which introduces the concepts of cryptoeconomics, transactions, blocks, consensus mechanisms, and use cases in the form of a simple story. Whereas the overview skips these introductory explanations and leads to further more detailed explanations about hash functions, consensus mechanisms, and transactions. The guided tour is developed in such a manner that the user determines the speed of the presentation and that various animations illustrate the presented concepts. The guided tour, once finished, seamlessly leads to the overview which lets the user decide freely about which topics they want to learn more. While the overview gives insights into different consensus mechanisms, transactions and public and private key cryptography, it also lets the user interact with the presented information about hash functions.

The ensuing evaluation compares the level of understanding of the technology between two groups\footnote{The groups of each nine students are divided into one group which reviews the artifact and another group which serves as comparison and watches existing YouTube videos.} and shows that the artifact leads to better results than existing videos. The evaluation also assesses the artifact's efficacy, effectiveness, efficiency, and utility, as well as its purposefulness and implementability, arguing in favor of all these properties. Nonetheless, the evaluation also shows that there are areas of improvement: The artifact should implement more visualizations, more interactive elements, and more practical examples in addition to a better user experience in order to further enhance the learning experience.

Overall, this paper therefore argues that the developed artifact is a useful instantiation of an answer to the above-stated research question. The most important concepts of blockchain technology are abstracted and presented in layman's terms in the form of interactive multimedia instructions, the user has the opportunity to interact with the information, and the artifact leads to provably better results than the evaluated existing learning resources.

\section{Critical Reflection} \label{sec:Reflection}
Although the research is conducted as rigorously as possible, the results are subject to certain limitations which are discussed in the following. 
While all guidelines defined in \ref{subsec:GuidelinesDesignScience} have been included into the research process,\footnote{The designed artifact is an instantiation of a solution to the problem that no appropriate and effective learning resources for a novice audience exist. The research contribution to the knowledge base is the instantiation of the artifact which has been rigorously designed grounded by scientific methods, and whose results are communicated in this paper} due to constraints in time and resources not all activities could be conducted with the same diligence. The demonstration of the artifact, for example, could have been refined and instead has been transformed to serve as the basis for the evaluation. In addition, the artifact is an instantiation and therefore very specific to its problem context. As there are various interactive learning resources available for different topics, some critics may not consider the creation of this interactive learning resource, specific to blockchain, an interesting and innovative solution. However, the artifact nevertheless solves a relevant and interesting problem for the organization and should therefore still be considered relevant. 

The research is also limited by its focus on Mayer's cognitive theory of multimedia learning and the cognitive load theory. Even though both are widely accepted theories, they may not represent the mechanics of the human brain correctly. The field is being actively researched and refined at the time of writing, and therefore any conclusions drawn from these theories might be regarded as faulty in the future. The process of requirements gathering may also be subject to limitations. The purposive sampling method is subject to bias and the small sample size might not be representative of the total population of \enquote{novices}. Additionally, some of the posed questions may be suggestive and therefore skew the responses. However, as the interviews are held in a conversational style and the experts come from different fields of work, the gathered information should nonetheless be broad enough to allow the specification of appropriate requirements. Finally, it should be noted that the evaluation focuses on one specific instantiation of the problem context and has been conducted in a short time frame. It is not clear if and how the results would change if conducted at an exhibit with other participants.\footnote{All participants in the groups are corporate students, who studying either business information systems or applied informatics.} Or if the questions to assess the participant's understanding would be asked an hour, a day, or a week after the participants have reviewed the learning resources instead of immediately afterwards.

Notwithstanding these limitations, the results of this research nonetheless prove the artifact's utility, quality, and efficacy. The contributions of this research should not be disregarded and are briefly discussed in the following section. 

\section{Implications for practice and research} \label{Implications}
The paper has shown that an interactive learning resource which introduces a person to a new topic is beneficial to the overall learning experience as it engages the person more easily in meaningful learning and helps them to gain a first high-level understanding of the topic. The evaluation has proven that (in the context of onboarding new students to a team) interactive learning resources are more effective than other multimedia instructions. Therefore, this paper suggests that practitioners should include interactive elements to their learning environments if possible. Furthermore, the paper has revealed that explanations regarding a new technology are commonly too technical.\footcite[Cf.][]{DanielKaltenbach_Interview} Thus it suggests to design explanations for novice audiences in a more general way that first explains the value of the technology before discussing technicalities that may be overwhelming for a beginner.

With regard to the research community, the created artifact, an instantiation, should be classified as an improvement contributing to the prescriptive knowledge base of design science research. While it is specific to its problem context, interesting new methods, models or theories may be inferred from the created artifact. Additionally, even though the number of publications related to the blockchain technology has risen significantly in the recent years, the question of how to best explain this complicated technology has not been covered. This gap is what this paper has attempted to fill.

\section{Future research} \label{sec:FutureResearch}
As suggested above, future research might include assessing whether any models, methods, or design theories might be inferred from this artifact. An example might be the question whether the number of interactive elements (which allow the manipulation of information) is positively linked to the learning progress. During this research, it has become apparent that accurately classifying a person's understanding with only a few simple questions is difficult and often inconclusive as the responses are subjective. Hence it would also be interesting to research more precise methods to assess a person's level of knowledge.

In addition, as design science research is an inherently iterative process, a next iteration of the research cycle might include the identified improvements to the artifact and assess the artifact's utility within a more thorough and extensive evaluation. It will be interesting to see if and how the approaches to explaining blockchain technology change once the technology gains more attention in not only the technical and business-related communities but in the overall population.