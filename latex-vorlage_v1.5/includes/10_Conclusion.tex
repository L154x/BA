\chapter{Conclusion}

\textbf{Length: 3-4 pages!}

\section{Purpose of the thesis} \label{sec:findings}

mention rigor and relevance

\section{Critical Reflection} \label{sec:Reflection}
Evaluation hat sich nur auf eine Instanz des Problemes bezogen. Wie sähe es aus mit der Wirksamkeit, wenn das Tutorial bei einer Messe angewendet wird?
Evaluation hätte anders aufgebaut sein können: mehr Teilnehmer, aus anderen Alters-/Personengruppen, andere Methode der Evaluation (zwar schon eine naturalistic ex post, aber eventuell ein Experiment, oder eine Fokusgruppe), längerer Zeitraum um auch zu sehen ob das im Langzeitgedächtnis eingegangen ist, Fragen vielleicht nicht eindeutig genug um daraus schlüsse zu ziehen

Definition der Anforderungen -> auf Basis der Interviews und bestehender Visualisierungen (aber keine Videos oder Artikel, obwohl die ja auch oft gelesen werden), zu wenig Interview-Partner, selektives Sampel ist nicht repräsentativ, 

Methodik: großer Fokus auf Mayer's cognitive load theory. Ist zwar sehr weit anerkannt, aber es gibt auch andere Theorien die hier nicht beachtet wurden. In dem Feld wird noch sehr viel geforscht und auch nicht alle Ergebnisse bezüglihc Multimedia und Interaktivität bezeugen eindeutig dass sie beim Lernen helfen.

Design Science: Artifakt ist "bloß" eine Instantiation, eventuell zu spezifisch auf den einen Kontext bezogen. Bzw. es gibt schon eine große Vielfalt an interaktiven Webseiten, welche Informationen visualisieren und beibringen, also nicht so eine krasse innovative Lösung -> aber für den spezifischen Fall schon.
Fokus lag klar auf der Relevanz, eventuell hat darunter Rigor geleidet (Evaluation und der Design Prozess hätten rigoroser durchgeführt werden können)

\section{Key findings}

\section{Implications for practice and research} \label{Implications}

\section{Future research} \label{sec:FutureResearch}