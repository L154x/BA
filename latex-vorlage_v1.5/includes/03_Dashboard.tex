% \chapter{Dashboard design}

% \section{General motivation behind dashboard design} \label{sec:DashboardGeneral}
% \begin{itemize}
%     \item What are dashboards?
%     \item Where do dashboards come from?
%     \item For what purpose where they developed?
%     \item How has their purpose changed throughout the years?
%     \item How has their use/amount/necessity of dashboards changed?
% \end{itemize}
% \section{Current practice} \label{sec:Dashboard_CurrentPractice}
% \begin{itemize}
%     \item What differentiates good dashboard design from bad dashboard design?
%     \item What methods are used to design dashboards?
%     \item What are current practices?
% \end{itemize}


% \textbf{ALTERNATIVELY: }

\chapter{Interactive multimedia instructions} \label{chapter:Multimedia}

% This chapter presents a short synthesis of the development of multimedia learning resources, it identifies the role of interactivity during the learning process and presents the current best practices regarding the design of learning material for a novice audience based on Mayer's cognitive theory of multimedia learning. 

Multimedia is defined as the joint presentation of both words (in verbal or written form) and pictures.\footcites[Cf.][p.2]{MayerMultimediaLearning2009}[cf.][p.1205]{MarraffinoApplyingMultimediaLearning2016}[cf.][p.13]{MayerAnimationAidMultimedia2001} The use of multimedia in instructions or multimedia learning environments has been shown to allow people to learn more deeply and to promote better learning outcomes and transfer.\footcites[Cf.][p.3]{MayerMultimediaLearning2009}[cf. in addition][]{MunzerLearningmultimediapresentations2009} For this reason, the paper argues that the artifact design should take the following theory and derived principles into account in order to present a relevant, useful and meaningful learning resource. This chapter presents Mayer's cognitive theory of multimedia learning and the guidelines that have been derived from it in order to establish a basis of good design principles for the ensuing artifact design. Additionally, one section investigates which role interactivity plays during the learning process.  

\section{Mayer's cognitive theory of multimedia learning}
Researching in the field of educational psychology for over 20 years and specializing in the area of multimedia design, Richard E. Mayer developed his cognitive theory of multimedia learning regarding the mechanics in the human brain in different learning environments.\footcites[Cf.][]{MayerMultimediaLearning2009}
% He argues for instructional design to take into consideration how the humain brain works and  
The theory is one of the most recognized principles of instruction and as virtually all instruction has become multimedia in one way or the other, this theory, promoting a learner-centered approach (based on how the human brain works) to instructional design, presents an important contribution to educational psychology.\footcites[Cf.][chapter 1, paragraph 3]{ClarkElearningscienceinstruction2016}[cf.][pp.4 et seq]{MayerMultimediaLearning2009}[cf. in addition][]{SordenCognitiveTheoryMultimedia}
His theory is based on Sweller's cognitive load theory and the findings derived from it, which are briefly outlined in the following paragraph.

\paragraph{Cognitive load theory} This theory was developed by Sweller in order to create a framework for instructional design. It starts from the idea that human processing capacity is limited and that only few information items may be processed by working memory at a time.\footcites[Cf.][p.250]{SwellerCognitiveArchitectureInstructional1998}[cf.][p.490]{Gervenefficiencymultimedialearning2003} The author argues that the information imposed on the working memory (the cognitive load) must not exceed simple cognitive activities or either be segmented in smaller chunks, otherwise overwhelming the brain and impeding schema construction (knowledge acquisition in the long-term memory).\footcites[Cf.][pp.255 et seq]{SwellerCognitiveArchitectureInstructional1998}[cf.][p.2]{SwellerVisualisationInstructionalDesign2002}[cf.][p.562]{Gervenefficiencymultimedialearning2003} Sweller therefore states that instructional design must take these working memory limitations into consideration and provides examples and guidelines to be accommodated in good design. For instance, he argues that instruction should be focused on domain specific information rather than teaching general reasoning strategies.\footcites[Cf.][p.255]{SwellerCognitiveArchitectureInstructional1998}[cf.][p.301]{SwellerCognitiveloadtheory1994} Additionally, alternative problem statements have been identified which allow the learners to adapt to new tasks: goal-free problems ("calculate all possible values"), worked examples (all steps for problem solution are provided), and completion problems (selected intermediate steps are provided).
As part of this theory, Sweller has also identified a continuum of element interactivity which shows that if several information items need to be processed simultaneously, the working memory is under a high load, which in turn complicates understanding/learning of this material.\footcites[Cf.][p.261]{SwellerCognitiveArchitectureInstructional1998} Due to the split-attention (separate presentation of pictures and words results in extraneous load), redundancy (unnecessary extraneous load if both representations contain the same information), and modality (use of different sensory processing channels allows deeper learning) effects, which have also been identified by Mayer in his cognitive theory of multimedia learning, Sweller argues that well-designed formats of multimedia instructions lead to a better acquisition of complex information than regular instructions (e.g. plain text).\footcites[Cf.][p.4]{PaasCognitiveLoadTheory2004}

\paragraph{Meaningful learning} Learning is broadly defined as the acquisition of knowledge.\footcites[Cf.][p.226]{MayerRotemeaningfullearning2002} From a more detailed point of view, this includes the strengthening of correct and weakening of incorrect responses, the addition of new information to the learner's memory (rote learning), and the sense making of the presented material by attending to relevant information, mentally reorganizing it, and connecting it with what the learner already knows (meaningful learning).\footcites[Cf.][chapter 2, paragraph 5]{ClarkElearningscienceinstruction2016} Meaningful learning differentiates from rote learning in so far as meaningful learning allows the construction of mental schemata and transfer of these schemata to new situations.% For such construction to occur, instructions must go beyond the simple presentation of factual knowledge.
\footcites[Cf.][p.227]{MayerRotemeaningfullearning2002}[cf.][p.299]{SwellerCognitiveloadtheory1994} Meaningful learning on the basis of multimedia instructions occurs when the learner is engaged in the five cognitive processes of selecting relevant words (1a), selecting relevant images (1b), organizing selected words into a verbal model (2a), organizing images into a pictorial model (2b), and integrating both models with each other and with prior knowledge (3).\footcites[Cf.][p.111]{MayerCognitiveTheoryMultimedia1999}[cf.][p.35]{SordenCognitiveTheoryMultimedia}[cf.][p.43]{MayerNineWaysReduce2003} This process is presented in figure \ref{fig:CTML_process}. Furthermore, the process relies on three assumptions, which form the basis of the theory: 1) there are different channels for auditory and visual sensory input (dual processing), 2) there is only limited processing capacity available (limited capacity), and 3) learning requires significant cognitive processing in both channels (active processing).\footcites[Cf.][p.44]{MayerNineWaysReduce2003}

\begin{figure}
    \centering
    \includesvg[width=\linewidth]{graphics/CTMLProcess}
    \caption[The cognitive processes for meaningful learning with multimedia instructions.]{The cognitive processes for meaningful learning with multimedia instructions.\footnotemark}
    \label{fig:CTML_process}
\end{figure}
\footnotetext{Taken with changes from \cite{MayerCognitiveTheoryMultimedia1999}, p.111.}

On the basis of this theory, \cite{MayerNineWaysReduce2003} have identified several methods to reduce the cognitive load during instructional design. Cognitive load may be reduced by offloading (move processing from visual to auditory channel), segmenting (divide information in small segments), weeding (eliminate unnecessary information), signaling (provide cues for important information), aligning (place words near corresponding graphics), eliminating redundancy (avoid identical information), or by synchronizing (present narration and animation simultaneously).\footcites[Cf.][p.46]{MayerNineWaysReduce2003}

If these aspects are considered during instructional design, multimedia serves as a cognitive aid to knowledge construction enabling a deeper and more meaningful understanding of the presented information. It is needless to state that for such understanding to occur, instructions must go beyond the simple presentation of factual knowledge.\footcites[Cf.][p.229]{MayerRotemeaningfullearning2002}

\section{The role of interactivity in the learning process}
In his cognitive theory of multimedia learning, Mayer focuses on the role of multimedia instructions. However, he has also presented additional research (among other authors) arguing for interactivity to be considered during instructional design.\footcites[Cf.][chapter 2, paragraph 12]{ClarkElearningscienceinstruction2016} Interactivity refers to the possibility of user interaction with the educational content.\footcites[Cf.][p.292]{PatwardhanWhendoeshigher2015} 



\section{Guidelines for designing multimedia instructions}
On the basis of this theory, Mayer (and others) have derived a number of principles that need to be considered when designing multimedia instructions. 

\paragraph{Principles}


Additionally to these principles, there are some other aspects of instructional design to be considered: Preliminary work, Cueing, Engaging students 

\paragraph{Distribution of instructions} -> Online learning 

% \section{Purpose of multimedia learning materials}
% \begin{itemize}
%     \item What is multimedia?
%     \item Why is it used?
%     \item What is the learning process?
%     \item Development of e-learning / online learning material
% \end{itemize}

% \section{Current practice}
% \begin{itemize}
%     \item What differentiates good from bad multimedia learning materials?
%     \item What are current practices for good design? (interactivity, good user experience, white space, pictures and explanatory text, ...)
% \end{itemize}