\chapter{Dashboard design}

\section{General motivation behind dashboard design} \label{sec:DashboardGeneral}
\begin{itemize}
    \item What are dashboards?
    \item Where do dashboards come from?
    \item For what purpose where they developed?
    \item How has their purpose changed throughout the years?
    \item How has their use/amount/necessity of dashboards changed?
\end{itemize}
\section{Current practice} \label{sec:Dashboard_CurrentPractice}
\begin{itemize}
    \item What differentiates good dashboard design from bad dashboard design?
    \item What methods are used to design dashboards?
    \item What are current practices?
\end{itemize}




\textbf{ALTERNATIVELY: }

\chapter{Multimedia learning} \label{chapter:Multimedia}

This chapter presents a short synthesis of the development of multimedia learning resources and current best practices regarding the design of learning material for a novice audience based on Mayer's cognitive theory of multimedia learning. 

Multimedia is defined as the presentation of both words (in verbal or written form) and pictures together\footcites[Cf.][p.2]{}

\section{Mayer's cognitive theory of multimedia learning}
Researching in the field of educational psychology for over 20 years and specializing in the area of multimedia design, Richard E. Mayer developed his cognitive theory of multimedia learning regarding the mechanics in the human brain in different learning environments. 
Why did I choose this theory as basis for the design? (why?)

His theory is based on Sweller's cognitive load theory and the findings derived from it, which are briefly outlined in the following paragraph.

\paragraph{The cognitive load theory} Define the theory, say its a framework for instructional design and -> working memory capacity is limited, resulting in xxx effects. Sweller identified alternative instructions (goal free problem, ...) to adapt to these findings and to optimize learning outcomes and transfer


\paragraph{Meaningful learning} What does learning mean? (learning definition) When is it meaningful? (Meaningful learning)



\section{Guidelines for designing multimedia instructions}
On the basis of this theory, Mayer (and others) have derived a number of principles that need to be considered when designing multimedia instructions. 

\paragraph{Principles}


Additionally to these principles, there are some other aspects of instructional design to be considered: Preliminary work, Cueing, Engaging students 

\paragraph{Distribution of instructions} -> Online learning 

% \section{Purpose of multimedia learning materials}
% \begin{itemize}
%     \item What is multimedia?
%     \item Why is it used?
%     \item What is the learning process?
%     \item Development of e-learning / online learning material
% \end{itemize}

% \section{Current practice}
% \begin{itemize}
%     \item What differentiates good from bad multimedia learning materials?
%     \item What are current practices for good design? (interactivity, good user experience, white space, pictures and explanatory text, ...)
% \end{itemize}