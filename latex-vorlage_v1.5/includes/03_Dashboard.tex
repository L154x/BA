\chapter{Interactive multimedia instructions} \label{chapter:Multimedia}

Multimedia is defined as the joint presentation of both words (in verbal or written form) and pictures.\footcites[Cf.][p.2]{MayerMultimediaLearning2009}[cf.][p.1205]{MarraffinoApplyingMultimediaLearning2016}[cf.][p.13]{MayerAnimationAidMultimedia2001} The use of multimedia in instructions or multimedia learning environments is proven to allow people to learn more deeply and to promote better learning outcomes and transfer.\footcites[Cf.][p.3]{MayerMultimediaLearning2009}[cf. in addition][]{MunzerLearningmultimediapresentations2009} For this reason, the paper argues that the artifact design should take the following theory and derived principles into account to present a relevant, useful and meaningful learning resource. Since it appears that existing learning resources have not been proven successful, it is suggested that a theory-led approach results in a better learning experience. This chapter, therefore, presents Mayer's cognitive theory of multimedia learning and the guidelines which are derived from it to establish a basis of good design principles for the ensuing artifact design. Additionally, one section investigates which role interactivity plays during the learning process. 

\section{Mayer's cognitive theory of multimedia learning} \label{sec:MayersCTML}
While researching in the field of educational psychology for over 20 years, Richard E. Mayer has specialized in the area of multimedia design in educational psychology. As part of this research, he has developed a cognitive theory of multimedia learning regarding the mechanics in the human brain in different learning environments.\footcites[Cf.][]{MayerMultimediaLearning2009}
The theory, promoting a learner-centered approach (based on how the human brain works) to instructional design, presents an important contribution to educational psychology, and is one of the most recognized principles of instruction.\footcites[Cf.][chapter 1, paragraph 3]{ClarkElearningscienceinstruction2016}[cf.][p.4 et seqq]{MayerMultimediaLearning2009}[cf. in addition][]{SordenCognitiveTheoryMultimedia2012} Because of its proven reputation, the research relies on this theory to create an effective and useful explanation of the blockchain technology.
Mayer's theory is based on Sweller's cognitive load theory and the findings derived from it, which are briefly outlined in the following paragraph.

\paragraph{Cognitive load theory} Sweller developed this theory to create a framework for instructional design. It starts from the idea that human processing capacity is limited and that only few information items may be processed by working memory at a time.\footcites[Cf.][p.250]{SwellerCognitiveArchitectureInstructional1998}[cf.][p.490]{Gervenefficiencymultimedialearning2003} The author argues that the information imposed on the working memory (the cognitive load) must not exceed simple cognitive activities or be segmented in smaller chunks, otherwise overwhelming the brain and impeding schema construction (knowledge acquisition in the long-term memory).\footcites[Cf.][p.255 et seqq]{SwellerCognitiveArchitectureInstructional1998}[cf.][p.2]{SwellerVisualisationInstructionalDesign2002}[cf.][p.562]{Gervenefficiencymultimedialearning2003} Sweller therefore states that instructional design must take these working memory limitations into consideration. He also provides examples and guidelines to be accommodated in good design. For instance, he argues that instruction should be focused on domain-specific information rather than teaching general reasoning strategies.\footcites[Cf.][p.255]{SwellerCognitiveArchitectureInstructional1998}[cf.][p.301]{SwellerCognitiveloadtheory1994} Additionally, alternative problem statements are identified which allow the learners to adapt to new tasks: goal-free problems (\enquote{calculate all possible values}), worked examples (providing all steps for the problem solution), and completion problems (providing selected intermediate steps).
As part of this theory, Sweller has also identified a continuum of element interactivity which shows that if several information items need to be processed simultaneously, the working memory is under a high load, which in turn complicates understanding/learning of this material.\footcites[Cf.][p.261]{SwellerCognitiveArchitectureInstructional1998} Sweller and Mayer also have identified the following effects: the split-attention effect (separate presentation of pictures and words results in extraneous load), the redundancy effect (unnecessary extraneous load if both representations contain the same information), and the modality effect (use of different sensory processing channels allows deeper learning). Sweller argues that well-designed formats of multimedia instructions, which consider these effects, lead to a better acquisition of complex information than regular instructions (e.g., plain text), and thus enhance learning.\footcites[Cf.][p.4]{PaasCognitiveLoadTheory2004} What exactly constitutes learning and how it may be enhanced is discussed in the following.

\paragraph{Meaningful learning} Learning is broadly defined as the acquisition of knowledge.\footcites[Cf.][p.226]{MayerRotemeaningfullearning2002} This includes the strengthening of correct and weakening of incorrect responses, the addition of new information to the learner's memory (rote learning), and the making-sense of the presented material by attending to relevant information, mentally reorganizing it, and connecting it with what the learner already knows (meaningful learning).\footcites[Cf.][chapter 2, paragraph 5]{ClarkElearningscienceinstruction2016} Meaningful learning differentiates from rote learning in so far as meaningful learning allows the construction of mental schemata and transfer of these schemata to new situations.% For such construction to occur, instructions must go beyond the simple presentation of factual knowledge.
\footcites[Cf.][p.227]{MayerRotemeaningfullearning2002}[cf.][p.299]{SwellerCognitiveloadtheory1994} Meaningful learning on the basis of multimedia instructions occurs when the learner is engaged in the five cognitive processes\footnote{The five processes divide into three parallel processes: selecting, organizing and integrating} presented in figure \ref{fig:CTML_process}.\footcites[Cf.][p.111]{MayerCognitiveTheoryMultimedia1999}[cf.][p.35]{SordenCognitiveTheoryMultimedia2012}[cf.][p.43]{MayerNineWaysReduce2003} This process relies on three assumptions, which form the basis of Sweller's theory: 1) There are different channels for auditory and visual sensory input (dual processing); 2) There is only limited processing capacity available (limited capacity); and 3) Learning requires significant cognitive processing in both channels (active processing).\footcites[Cf.][p.44]{MayerNineWaysReduce2003}

Based on this theory, \cite{MayerNineWaysReduce2003} have identified several methods to reduce the cognitive load during instructional design. Cognitive load may be reduced by offloading (moving processing from visual to auditory channel), segmenting (dividing information into small segments), weeding (eliminating unnecessary information), signaling (providing cues for important information), aligning (placing words near corresponding graphics), eliminating redundancy (avoiding identical information), or by synchronizing (presenting narration and animation simultaneously).\footcites[Cf.][p.46]{MayerNineWaysReduce2003}

If these aspects are considered during instructional design, multimedia serves as a cognitive aid to knowledge construction enabling a deeper and more meaningful understanding of the presented information. It is needless to say that for such understanding to occur, instructions must go beyond the simple presentation of factual knowledge.\footcites[Cf.][p.229]{MayerRotemeaningfullearning2002}

\begin{figure}[!htb]
    \centering
    \includesvg[width=0.8\linewidth]{graphics/CTMLProcess}
    \caption[The cognitive processes for meaningful learning with multimedia instructions.]{The cognitive processes for meaningful learning with multimedia instructions.\protect\footnotemark}
    \label{fig:CTML_process}
\end{figure}
\footnotetext{With changes taken from \cite{MayerCognitiveTheoryMultimedia1999}, p.111}


\paragraph{Critique} Even though many studies have shown strong support for Mayer's thesis, others were not able to reproduce these findings. For instance, \cite{RaschInteractivenoninteractivepictures2009} did not see any improvements when comparing comprehension from written and graphical instructions with that from written instructions.\footcites[Cf.][]{RaschInteractivenoninteractivepictures2009} However, the researchers acknowledge these problems and continue to develop the theory and the utilized research techniques.\footcites[Cf.][p.52]{SordenCognitiveTheoryMultimedia2012} 

\section{The role of interactivity in the learning process} \label{sec:Interactivity}
Mayer focuses on the role of multimedia instructions. In addition to his cognitive theory of multimedia learning, Mayer has also presented further research (among other authors) arguing for interactivity to be considered during instructional design.\footcites[Cf.][chapter 2, paragraph 12]{ClarkElearningscienceinstruction2016} Interactivity refers to the possibility of user interaction with educational content.\footcites[Cf.][p.292]{PatwardhanWhendoeshigher2015} In the narrower context of multimedia instructions, this paper defines interactivity as the reciprocal activity between a learner and the learning environment, in which an action of one party triggers a response from the other party.\footcites[Cf.][p.1025]{DomagkInteractivitymultimedialearning2010} This kind of interactivity is suggested to support meaningful learning (understanding) by engaging the user in active learning, for example by manipulating virtual objects or by simulating experiments or industrial processes.\footcites[Cf.][p.161]{CairncrossInteractiveMultimediaLearning2001}[cf.][p.1159]{Evansinteractivityeffectmultimedia2007} Interactive features may be categorized into five different types: 1) Dialoguing (a learner receives questions and answers or feedback on their input); 2) Controlling (a learner controls the pace and/or the order of the presentation); 3) Manipulating (a learner is free to set parameters for a simulation, or to move objects around); 4) Searching (a learner may find new information by specifically querying for it); 5) Navigating (a learner moves to different content areas).\footcites[Cf.][p.311]{MorenoInteractiveMultimodalLearning2007}

Although interactivity has been heralded by many as one of the key features of multimedia learning to support the learning process, results have been inconclusive. Some studies support this hypothesis and report better scores in interactive multimedia learning environments, whereas others report the opposite suggesting that interactive features might actually hinder understanding.\footcites[Cf.][p.1024]{DomagkInteractivitymultimedialearning2010}[cf.][]{MayerWhenLearningJust2001}[cf.][p.156]{CairncrossInteractiveMultimediaLearning2001}[cf.][p. 1148 et seqq]{Evansinteractivityeffectmultimedia2007}[cf.][p.48]{SordenCognitiveTheoryMultimedia2012} These mixed results may originate from the different types of experiments conducted, the evaluation methods used to assess the differences in understanding, or from the different types of interactivity utilized. When designing multimedia instructions, interactive features should, therefore, be skillfully designed to draw learners attention to important pieces of information, to motivate and to help learners achieve an understanding of the material, and not to distract.\footcites[Cf.][p.21]{KirshInteractivitymultimediainterfaces1997}[cf.][p.15]{LeeScreenDesignGuidelines1999} For this purpose, \cite{MorenoInteractiveMultimodalLearning2007} have developed some design principles which will be presented in the following section.

\section{Guidelines for designing multimedia instructions} \label{sec:GuidelinesMultimediaInstr}
As this paper's aim is to design an easy to understand explanation of the blockchain technology, the utilized multimedia instructions must be well-designed in order to support meaningful learning. For this purpose, the paper relies on existing guidelines and best practices for instructional design. While different aspects, ranging from psychological and aesthetic questions to the distribution of the final instructions, need to be considered, this section primarily focuses on design principles that are derived from the cognitive theory of multimedia learning. Thereafter, general best practices regarding screen design and user experience are discussed which constitute the foundation for the ensuing artifact development.

\subsection{Principles derived from the cognitive theory of multimedia learning}
On the basis of the cognitive theory of multimedia learning, a number of principles are identified. Some of these may be mapped directly to the nine ways to reduce cognitive load. All principles presented below are empirically studied and proven to enhance the learning process, which is why their consideration during instructional design is of key importance.\footcites[Cf. for more detail][]{SordenCognitiveTheoryMultimedia2012}

\begin{enumerate}
    \item \textbf{Multimedia principle}: The first principle states that the combined presentation of graphics (images, animations, or visualizations) and words (written or spoken) is better than the presentation of a single material. This is because by presenting both materials, students build a mental connection of the two presentations, and hence engage in meaningful learning.\footcites[Cf.][p.19]{MayerAnimationAidMultimedia2001}[cf.][chapter 4, paragraph 7]{ClarkElearningscienceinstruction2016}[cf.][p.13]{MayerCognitiveTheoryMultimedia1999} It is important to consider that not all graphics are equally useful for this tasks. Only graphics that are relevant to the instructional purpose should be used (these may be organizational, transformational or interpretative graphics). Decorative graphics may even hurt understanding (additional cognitive load). Furthermore, the multimedia principle is especially important for novices (little to no prior knowledge), as experts already have existing mental schemata of the material.\footcites[Cf.][chapter 4, paragraphs 7 et seq]{ClarkElearningscienceinstruction2016}[cf. in addition][]{MayerWhenillustrationworth1990}
    \item \textbf{Contiguity principle}: By presenting words and pictures or by coordinating spoken words and graphics in proximity to each other (both in place and in time), students are said to learn more deeply. The rationale behind this principle is that students build mental schemata more easily if both information pieces are processed at the same time.\footcites[Cf.][p. 19 et seqq]{MayerAnimationAidMultimedia2001}[cf.][chapter 5, paragraphs 1 et seq]{ClarkElearningscienceinstruction2016} If screen space is limited, this may also be achieved by tooltips which appear when hovering over a specific piece of material.
    \item \textbf{Modality principle}: Also sometimes referred to as the split-attention principle, it states that words should be presented in spoken form rather than written form. This corresponds to the offloading method presented in \ref{sec:MayersCTML}, as some of the processing load is taken off the visual channel and transferred to the auditory channel.\footcites[Cf.][p.22]{MayerAnimationAidMultimedia2001}[cf.][p.14]{MayerCognitiveTheoryMultimedia1999}
    \item \textbf{Redundancy principle}: Based on the same rationale as the modality principle, this principle states that students learn more deeply when the same information is not presented in more than one format. This means that graphics should be explained with words in either audio or written format, but not both as redundant information causes additional extraneous load.\footcites[Cf.][chapters 6 and 7]{ClarkElearningscienceinstruction2016}[cf.][p.6]{MayerMultimediaLearning2009}[cf.][p.22]{MayerAnimationAidMultimedia2001}[cf. in addition][]{MayerPrinciplesreducingextraneous2014}
    \item \textbf{Coherence principle}: The coherence principle describes the fact that extra material (be it for interest, technical depth or to expand on key ideas) hurts meaningful learning. It suggests the designers should focus on concise descriptions and to delete any material that does not support the instructional goal. This principle relates to the weeding method from \ref{sec:MayersCTML}, as any extraneous words and pictures are removed to provide one coherent summary instead of a longer version.\footcites[Cf.][chapter 8, paragraphs 1 et seq]{ClarkElearningscienceinstruction2016}[cf.][p.6]{MayerMultimediaLearning2009}[cf.][p.22]{MayerAnimationAidMultimedia2001}
    \item \textbf{Personalization \& embodiment principle}: Mayer and Moreno suggest that by using effective on-screen coaches (embodiment) and polite wording in a conversational style (personalization) a feeling of social presence is created which stimulates students to more deeply engage in the learning process. 
    \item \textbf{Segmenting \& pretraining principle}: In situations involving complex material, it is helpful to manage the complexity by breaking the lesson into smaller parts (segmenting, see also segmenting in \ref{sec:MayersCTML}) and to ensure that learners know the names and characteristics of key concepts (pretraining).\footcites[Cf.][chapters 9 and 10]{ClarkElearningscienceinstruction2016}
    \item \textbf{Guided discovery principle}: The guided discovery principle states that students learn better when given an orientation in a discovery-based learning environment.\footcites[Cf.][p.7]{MayerMultimediaLearning2009}
    \item \textbf{Animation \& interactivity principle}: The animation and interactivity principle states that animations as well as interactive visualizations facilitate understanding because dynamic processes that may be difficult to imagine are visualized.\footcites[Cf.][p.290]{Betrancourtanimationinteractivityprinciples2005}[cf.][p.81]{MunzerLearningmultimediapresentations2009}[cf.][p.19]{LeeScreenDesignGuidelines1999}[cf.][p.814]{MayerNineWaysReduce2003} However, as stated earlier (see \ref{sec:Interactivity}), animations and interactive visualizations are not always helpful, and hence, their presentation should be skillfully designed. For this purpose, \cite{MorenoInteractiveMultimodalLearning2007} have defined more detailed principles regarding guided activity, reflection, feedback, and pacing.\footcites[Cf.][p.292]{PatwardhanWhendoeshigher2015}[cf.][p.7]{MayerMultimediaLearning2009}[cf.][p.316]{MorenoInteractiveMultimodalLearning2007} % \begin{enumerate}
    %     \item \textbf{Guided activity:} A pedagogical agent who helps guide students' cognitive processing is helpful for meaningful learning.
    %     \item \textbf{Reflection}: By asking students to reflect upon answers, the process of sense-making is facilitated.
    %     \item \textbf{Feedback}: Students learn better with explanatory rather than corrective feedback.
    %     \item \textbf{Pacing}: If allowed to control the pace of presentation of the learning material, students are allowed to adapt the presentation to their processing rate, thus facilitating learning.
    % \end{enumerate}
    \item \textbf{Site map principle}: Students learn better if the learning environment contains a map showing the learner's progress.\footcites[Cf.][p.7]{MayerMultimediaLearning2009}
  \item \textbf{Individual differences principle}: This principle relativizes the afore-described principles in so far as it states that multimedia, contiguity and split-attention effects depend on individual differences in the learners.\footcites[Cf.][p.15]{MayerCognitiveTheoryMultimedia1999}
\end{enumerate}
By following these principles, instructional designers may create learning environments that minimize extraneous loads, effectively managing the brain's limited processing capacity, and thus foster meaningful learning.\footcites[Cf.][chapter 2, paragraph 6]{ClarkElearningscienceinstruction2016}

\subsection{Best practices for instructional design} \label{subsec:BestPracticesDesign}
As the listed principles apply to all multimedia instructions, they need to be considered during the artifact design process. However, they do not cover all aspects of instructional design, which is why the next paragraphs add to these principles by providing additional guidelines for instructional design.

\paragraph{Preliminary work and student engagement} Before the instructions are designed, their objectives should be defined. By defining the target group, analyzing the potential users' needs and identifying the system requirements, this preliminary work allows the designers to adapt the learning environment in such a way that it may better engage students, designing it to be more relevant to their needs. Additionally, student engagement may be enhanced by pursuing active learning strategies (e.g., focused-is-more approach: giving guidance to motivate meaningful learning), or by promoting flexibility (with modules or layers). For the latter, the use of clear navigational techniques (tooltips, guided tours, or defined learning outcomes) is suggested.\footcites[Cf.][p.202]{BlummerBestPracticesCreating2009}

\paragraph{Cueing} Cueing is defined as giving visual cues to the learner. This directs the learner's attention to a specific part of the material. Cueing serves three functions: 1) guide the attention to facilitate the selection of essential information; 2) emphasize major parts of instruction and their organization; and 3) foster integration by making relationships between information elements more visible. Cues may be implemented by animating or increasing the luminance of an element. Even though cues are shown to capture attention effectively, this does not necessarily improve understanding. The implementation of such features should always be considered under the light of the principles mentioned above (e.g., coherence).\footcites[Cf.][p.114 et seqq]{deKoningFrameworkAttentionCueing2009}[cf.][chapter 2, paragraph 14]{ClarkElearningscienceinstruction2016}

\paragraph{User experience} For a good user experience, the instructions should satisfy the principle of visibility. This means that users ought to see available actions, receive immediate feedback from executed actions and get timely information about the consequences of these actions. Additionally, the environment should be aesthetically pleasing, which may be achieved by eliminating clutter (offloading, weeding), selecting matching colors, consistent style and use of elements, as well as by following the general rules of visual composition (balance, symmetry, unity, harmony).\footcites[Cf.][p. 16 et seqq]{LeeScreenDesignGuidelines1999}[cf.][p. 16 et seqq]{Nadelhoffer10BestPractices}[cf.][p.20]{KirshInteractivitymultimediainterfaces1997}

% Regarding the distribution of the created multimedia instructions, the format of online learning environments offers the highest accessibility. Computers are a well-relied on medium today as they represent one of the most flexible and widespread media options and by providing the instructions in the form of a web application, anybody with computer access may view and learn from it.\footcites[Cf.][chapter 1, paragraphs 6 et seq]{ClarkElearningscienceinstruction2016}[cf.][p.906]{BaharinEvaluationSatisfactionUsing2015}

This chapter has presented the fundamental principles that are considered during the artifact design. Such a theory-led design approach sets a theoretical foundation for any design decisions. However, the designing of the artifact is still a creative process which relies heavily on human creativity which theoretical guidelines may not (or only arduously) represent.\footcites[Cf.][p.7]{VaishnaviDesignScienceResearch}