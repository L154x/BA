% \chapter{Dashboard design}

% \section{General motivation behind dashboard design} \label{sec:DashboardGeneral}
% \begin{itemize}
%     \item What are dashboards?
%     \item Where do dashboards come from?
%     \item For what purpose where they developed?
%     \item How has their purpose changed throughout the years?
%     \item How has their use/amount/necessity of dashboards changed?
% \end{itemize}
% \section{Current practice} \label{sec:Dashboard_CurrentPractice}
% \begin{itemize}
%     \item What differentiates good dashboard design from bad dashboard design?
%     \item What methods are used to design dashboards?
%     \item What are current practices?
% \end{itemize}


% \textbf{ALTERNATIVELY: }

\chapter{Interactive multimedia instructions} \label{chapter:Multimedia}

% This chapter presents a short synthesis of the development of multimedia learning resources, it identifies the role of interactivity during the learning process and presents the current best practices regarding the design of learning material for a novice audience based on Mayer's cognitive theory of multimedia learning. 

Multimedia is defined as the joint presentation of both words (in verbal or written form) and pictures.\footcites[Cf.][p.2]{MayerMultimediaLearning2009}[cf.][p.1205]{MarraffinoApplyingMultimediaLearning2016}[cf.][p.13]{MayerAnimationAidMultimedia2001} The use of multimedia in instructions or multimedia learning environments has been shown to allow people to learn more deeply and to promote better learning outcomes and transfer.\footcites[Cf.][p.3]{MayerMultimediaLearning2009}[cf. in addition][]{MunzerLearningmultimediapresentations2009} For this reason, the paper argues that the artifact design should take the following theory and derived principles into account in order to present a relevant, useful and meaningful learning resource. This chapter presents Mayer's cognitive theory of multimedia learning and the guidelines that have been derived from it in order to establish a basis of good design principles for the ensuing artifact design. Additionally, one section investigates which role interactivity plays during the learning process.  

\section{Mayer's cognitive theory of multimedia learning} \label{sec:MayersCTML}
Researching in the field of educational psychology for over 20 years and specializing in the area of multimedia design, Richard E. Mayer developed his cognitive theory of multimedia learning regarding the mechanics in the human brain in different learning environments.\footcites[Cf.][]{MayerMultimediaLearning2009}
% He argues for instructional design to take into consideration how the humain brain works and  
The theory is one of the most recognized principles of instruction and as virtually all instruction has become multimedia in one way or the other, this theory, promoting a learner-centered approach (based on how the human brain works) to instructional design, presents an important contribution to educational psychology.\footcites[Cf.][chapter 1, paragraph 3]{ClarkElearningscienceinstruction2016}[cf.][pp.4 et seq]{MayerMultimediaLearning2009}[cf. in addition][]{SordenCognitiveTheoryMultimedia2012}
His theory is based on Sweller's cognitive load theory and the findings derived from it, which are briefly outlined in the following paragraph.

\paragraph{Cognitive load theory} This theory was developed by Sweller in order to create a framework for instructional design. It starts from the idea that human processing capacity is limited and that only few information items may be processed by working memory at a time.\footcites[Cf.][p.250]{SwellerCognitiveArchitectureInstructional1998}[cf.][p.490]{Gervenefficiencymultimedialearning2003} The author argues that the information imposed on the working memory (the cognitive load) must not exceed simple cognitive activities or either be segmented in smaller chunks, otherwise overwhelming the brain and impeding schema construction (knowledge acquisition in the long-term memory).\footcites[Cf.][pp.255 et seq]{SwellerCognitiveArchitectureInstructional1998}[cf.][p.2]{SwellerVisualisationInstructionalDesign2002}[cf.][p.562]{Gervenefficiencymultimedialearning2003} Sweller therefore states that instructional design must take these working memory limitations into consideration and provides examples and guidelines to be accommodated in good design. For instance, he argues that instruction should be focused on domain specific information rather than teaching general reasoning strategies.\footcites[Cf.][p.255]{SwellerCognitiveArchitectureInstructional1998}[cf.][p.301]{SwellerCognitiveloadtheory1994} Additionally, alternative problem statements have been identified which allow the learners to adapt to new tasks: goal-free problems ("calculate all possible values"), worked examples (all steps for problem solution are provided), and completion problems (selected intermediate steps are provided).
As part of this theory, Sweller has also identified a continuum of element interactivity which shows that if several information items need to be processed simultaneously, the working memory is under a high load, which in turn complicates understanding/learning of this material.\footcites[Cf.][p.261]{SwellerCognitiveArchitectureInstructional1998} Due to the split-attention (separate presentation of pictures and words results in extraneous load), redundancy (unnecessary extraneous load if both representations contain the same information), and modality (use of different sensory processing channels allows deeper learning) effects, which have also been identified by Mayer in his cognitive theory of multimedia learning, Sweller argues that well-designed formats of multimedia instructions lead to a better acquisition of complex information than regular instructions (e.g. plain text).\footcites[Cf.][p.4]{PaasCognitiveLoadTheory2004}

\paragraph{Meaningful learning} Learning is broadly defined as the acquisition of knowledge.\footcites[Cf.][p.226]{MayerRotemeaningfullearning2002} From a more detailed point of view, this includes the strengthening of correct and weakening of incorrect responses, the addition of new information to the learner's memory (rote learning), and the sense making of the presented material by attending to relevant information, mentally reorganizing it, and connecting it with what the learner already knows (meaningful learning).\footcites[Cf.][chapter 2, paragraph 5]{ClarkElearningscienceinstruction2016} Meaningful learning differentiates from rote learning in so far as meaningful learning allows the construction of mental schemata and transfer of these schemata to new situations.% For such construction to occur, instructions must go beyond the simple presentation of factual knowledge.
\footcites[Cf.][p.227]{MayerRotemeaningfullearning2002}[cf.][p.299]{SwellerCognitiveloadtheory1994} Meaningful learning on the basis of multimedia instructions occurs when the learner is engaged in the five cognitive processes of selecting relevant words (1a), selecting relevant images (1b), organizing selected words into a verbal model (2a), organizing images into a pictorial model (2b), and integrating both models with each other and with prior knowledge (3).\footcites[Cf.][p.111]{MayerCognitiveTheoryMultimedia1999}[cf.][p.35]{SordenCognitiveTheoryMultimedia2012}[cf.][p.43]{MayerNineWaysReduce2003} This process is presented in figure \ref{fig:CTML_process}. Furthermore, the process relies on three assumptions, which form the basis of the theory: 1) there are different channels for auditory and visual sensory input (dual processing), 2) there is only limited processing capacity available (limited capacity), and 3) learning requires significant cognitive processing in both channels (active processing).\footcites[Cf.][p.44]{MayerNineWaysReduce2003}

\begin{figure}
    \centering
    \includesvg[width=\linewidth]{graphics/CTMLProcess}
    \caption[The cognitive processes for meaningful learning with multimedia instructions.]{The cognitive processes for meaningful learning with multimedia instructions.\footnotemark}
    \label{fig:CTML_process}
\end{figure}
\footnotetext{Taken with changes from \cite{MayerCognitiveTheoryMultimedia1999}, p.111.}

On the basis of this theory, \cite{MayerNineWaysReduce2003} have identified several methods to reduce the cognitive load during instructional design. Cognitive load may be reduced by offloading (move processing from visual to auditory channel), segmenting (divide information in small segments), weeding (eliminate unnecessary information), signaling (provide cues for important information), aligning (place words near corresponding graphics), eliminating redundancy (avoid identical information), or by synchronizing (present narration and animation simultaneously).\footcites[Cf.][p.46]{MayerNineWaysReduce2003}

If these aspects are considered during instructional design, multimedia serves as a cognitive aid to knowledge construction enabling a deeper and more meaningful understanding of the presented information. It is needless to state that for such understanding to occur, instructions must go beyond the simple presentation of factual knowledge.\footcites[Cf.][p.229]{MayerRotemeaningfullearning2002}

\paragraph{Critique} Even though many studies have shown strong support for Mayer's thesis, others were not able to reproduce these findings. For instance, \cite{RaschInteractivenoninteractivepictures2009} did not see any improvements when comparing comprehension from text and graphic instructions with that from text instructions.\footcites[Cf.][]{RaschInteractivenoninteractivepictures2009} However, the researchers acknowledge these problems and continuously develop the theory and the utilized research techniques.\footcites[Cf.][p.52]{SordenCognitiveTheoryMultimedia2012} 

\section{The role of interactivity in the learning process} \label{sec:Interactivity}
Mayer focuses on the role of multimedia instructions. Additionally to his cognitive theory of multimedia learning, Mayer has also presented additional research (among other authors) arguing for interactivity to be considered during instructional design.\footcites[Cf.][chapter 2, paragraph 12]{ClarkElearningscienceinstruction2016} Interactivity refers to the possibility of user interaction with educational content.\footcites[Cf.][p.292]{PatwardhanWhendoeshigher2015} In the narrower context of multimedia instructions, this paper defines interactivity as the reciprocal activity between a learner and the learning environment, in which one action of one party triggers a response from the other party.\footcites[Cf.][p.1025]{DomagkInteractivitymultimedialearning2010} This kind of interactivity is suggested to support meaningful learning (understanding) by engaging the user in active learning, for example by manipulating virtual objects or by simulating experiments or industrial processes.\footcites[Cf.][p.161]{CairncrossInteractiveMultimediaLearning2001}[cf.][p.1159]{Evansinteractivityeffectmultimedia2007} Interactive features may be categorized in five different types: 1) Dialoguing (a learner receives questions and answers or feedback to their input); 2) Controlling (a learner controls the pace and/or the order of the presentation); 3) Manipulating (a learner is free to set parameters for a simulation, or to move objects around); 4) Searching (a learner may find new information by specifically querying for it); 5) Navigating (a learner moves to different content areas).\footcites[Cf.][p.311]{MorenoInteractiveMultimodalLearning2007} Visualizations represent another kind of interactive visual representations, which does not directly correlate to the previous types. Visualizations present data in interesting ways, thus amplyfying cognition and enhancing meaningful learning (better comprehension and shortened learning time due to students' higher engagement in the learning process).\footcites[Cf.][p.294]{PatwardhanWhendoeshigher2015} 

Although interactivity has been heralded by many as one of the key features of multimedia learning to support the learning process, results have been inconclusive. Some studies support this hypothesis and report better scores in interactive multimedia learning environments, whereas others report the opposite suggesting that interactive features might actually hinder understanding.\footcites[Cf.][p.1024]{DomagkInteractivitymultimedialearning2010}[cf.][]{MayerWhenLearningJust2001}[cf.][p.156]{CairncrossInteractiveMultimediaLearning2001}[cf.][pp. 1148 et seq]{Evansinteractivityeffectmultimedia2007}[cf.][p.48]{SordenCognitiveTheoryMultimedia2012}

These mixed results may originate from the different types of experiments conducted, the evaluation methods used to assess the differences in understanding, or from the different types of interactivity utilized. Hence, when designing multimedia instructions, interactive features should be skillfully designed in order to draw learners attention to important pieces of information, to motivate and to help learners achieve understanding of the material.\footcites[Cf.][p.21]{KirshInteractivitymultimediainterfaces1997}[cf.][p.15]{LeeScreenDesignGuidelines1999}

For this purpose, \cite{MorenoInteractiveMultimodalLearning2007} have developed a number of design principles (presented under the Animation and Interactivity principle), which will be presented in the following section.


\section{Guidelines for designing multimedia instructions}
In order to ensure well-designed multimedia instructions, a variety of different guidelines exists. While different aspects, ranging from psychological and aesthetic questions to the distribution of the final instructions, need to be considered, this section primarily focuses on design principles that are derived from the cognitive theory of multimedia learning. Thereafter, general best practices regarding screen design and user experience, as well as distribution methods are discussed.
%Based on the findings of the cognitive load theory and the cognitive theory of multimedia learning, a number of principles that need to be considered during instructional design have been defined. These principles serve as 

\subsection{Principles derived from the cognitive theory of multimedia learning}
On the basis of the cognitive theory of multimedia learning, a number of principles have been identified. Some of these may be mapped directly to the nine ways to reduce cognitive load, while others do not. However, all of the principles presented below have been empirically studied and are proven to enhance the learning process, which is why their consideration during instructional design is of key importance.\footcites[Cf. for more detail][]{SordenCognitiveTheoryMultimedia2012}

\begin{enumerate}
    \item \textbf{Multimedia principle}: The first principle states that the combined presentation of graphics (images, animations, or visualizations) and words (written or spoken) is better than the presentation of one material. This is because by presenting both materials, students build a mental connection of the two presentations, and hence engage in meaningful learning.\footcites[Cf.][p.19]{MayerAnimationAidMultimedia2001}[cf.][chapter 4, paragraph 7]{ClarkElearningscienceinstruction2016}[cf.][p.13]{MayerCognitiveTheoryMultimedia1999} It is important to consider that not all graphics are equally useful for this tasks. Only graphics that are relevant to the instructional purpose should be used (these may be organizational, transformational or interpretative graphics), decorative graphics may even have an adverse effect (additional cognitive load). Furthermore, the multimedia principle is especially important for novices (little to no prior knowledge), as experts already have existing mental schemata of the material.\footcites[Cf.][chapter 4, paragraphs 7 et seq]{ClarkElearningscienceinstruction2016}[cf. in addition][]{MayerWhenillustrationworth1990}
    \item \textbf{Contiguity principle}: By presenting words and pictures or by coordinating spoken words and graphics in close proximity to each other (both in place and in time), students are said to learn more deeply. The rationale behind this principle is that students build mental schemata more easily if both information pieces are processed at the same time.\footcites[Cf.][pp. 19 et seq]{MayerAnimationAidMultimedia2001}[cf.][chapter 5, paragraphs 1 et seq.]{ClarkElearningscienceinstruction2016} If screen space is limited, this may also be achieved by tooltips which appear when hovering over a specific piece of material.
    \item \textbf{Modality principle}: Also sometimes referred to as the split-attention principle, it states that words should be presented in spoken form rather than written form. This corresponds to the offloading method presented in \ref{sec:MayersCTML}, as some of the processing load is taken off the visual channel and transferred to the auditory channel.\footcites[Cf.][p.22]{MayerAnimationAidMultimedia2001}[cf.][p.14]{MayerCognitiveTheoryMultimedia1999}
    \item \textbf{Redundancy principle}: Based on the same rationale as the modality principle, this principle states that students learn more deeply when the same information is not presented in more than one format. This means that graphics should be explained with words in either audio or written format, but not both as redundant information causes additional extraneous load.\footcites[Cf.][chapters 6 and 7]{ClarkElearningscienceinstruction2016}[cf.][p.6]{MayerMultimediaLearning2009}[cf.][p.22]{MayerAnimationAidMultimedia2001}[cf. in addition][]{MayerPrinciplesreducingextraneous2014}
    \item \textbf{Coherence principle}: The coherence principle describes the fact that extra material (be it for interest, technical depth or to expand on key ideas) hurts meaningful learning. It suggests the designers to focus on concise descriptions and to remove any material that does not support the instructional goal. This principles relates to the Weeding method from \ref{sec:MayersCTML}, as any extraneous words and pictures are removed to provide one coherent summary instead of a longer version.\footcites[Cf.][chapter 8, paragraphs 1 et seq]{ClarkElearningscienceinstruction2016}[cf.][p.6]{MayerMultimediaLearning2009}[cf.][p.22]{MayerAnimationAidMultimedia2001}
    \item \textbf{Personalization \& embodiment principle}: Mayer and Moreno suggest that by using effective on-screen coaches (embodiment) and polite wording in a conversational style (personalization), a feeling of social presence is created which stimulates students to more deeply engage in the learning process. 
    \item \textbf{Segmenting \& pretrainig principle}: In situations which involve complex material to be understood, it is helpful to manage the complexity by breaking the lesson into smaller parts (segmenting, see also segmenting in \ref{sec:MayersCTML}) and to ensure that learners know the names and characteristics of key concepts (pretraining). If done correctly, these methods allow to more effectively manage the cognitive load needed to understand the material.\footcites[Cf.][chapters 9 and 10]{ClarkElearningscienceinstruction2016}
    \item \textbf{Guided discovery principle}: The guided discovery principle states that students learn better when given an orientation in a discovery-based learning environment.\footcites[Cf.][p.7]{MayerMultimediaLearning2009}
    \item \textbf{Animation \& interactivity principle}: The animation and interactivity principle states that animations as well as interactive visualizations facilitate understanding as dynamic processes that may be difficult to visually imagine are visualized (external support for visual-spatial processing).\footcites[Cf.][p.290]{Betrancourtanimationinteractivityprinciples2005}[cf.][p.81]{MunzerLearningmultimediapresentations2009}[cf.][p.19]{LeeScreenDesignGuidelines1999}[cf.][p.814]{MayerNineWaysReduce2003} However, as stated earlier (see \ref{sec:Interactivity}), animations and interactive visualizations are not always helpful.\footcites[Cf.][p.292]{PatwardhanWhendoeshigher2015}[cf.][p.7]{MayerMultimediaLearning2009} Their presentation should be skillfully designed. For this purpose, \cite{MorenoInteractiveMultimodalLearning2007} has defined the following, more detailed principles: 
    \begin{enumerate}
        \item \textbf{Guided activity:} A pedagogical agent who helps guide students' cognitive processing is helpful for meaningful learning.
        \item \textbf{Reflection}: By asking students to reflect upon answers, the process of sense making is facilitated.
        \item \textbf{Feedback}: Students learn better with explanatory rather than corrective feedback.
        \item \textbf{Pacing}: If allowed to control the pace of presentation of the learning material, students are allowed to adapt the presentation to their processing rate, thus facilitating learning.\footcites[Cf.][p.316]{MorenoInteractiveMultimodalLearning2007}
    \end{enumerate}
    
\end{enumerate}

Additionally to these principles, there are some other aspects of instructional design to be considered: Preliminary work, Cueing, Engaging students 
As well as general user experience design, ...

\subsection{Best practices for instructional design and user experience} -> Online learning 

% \section{Purpose of multimedia learning materials}
% \begin{itemize}
%     \item What is multimedia?
%     \item Why is it used?
%     \item What is the learning process?
%     \item Development of e-learning / online learning material
% \end{itemize}

% \section{Current practice}
% \begin{itemize}
%     \item What differentiates good from bad multimedia learning materials?
%     \item What are current practices for good design? (interactivity, good user experience, white space, pictures and explanatory text, ...)
% \end{itemize}