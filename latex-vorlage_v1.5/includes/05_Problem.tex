\chapter{Requirements gathering} \label{chap:ReqEng}

This chapter determines which requirements are possible 
\section{Research method and objective}


\begin{itemize}
    \item the requirements will be extracted from expert interviews
    \begin{itemize}
        \item Why an interview?
        \item What kind of interview? explorative expert interview
        \item What experts? Frank Bloch, Bernd Kammholz, Torsten Milsmann, Daniel Kaltenbach, Thomas Maier -> all experts in their respective field
        \item What questions should I ask them? What do you not understand about blockchain? What would you wish to understand the most? Where do you feel lies the difficulty to blockchain and its transactions? What do you imagine under such a term as a dashboard visualizing blockchain transactions? What would be the best way to learn about it? Do you want it to be portable?
    \end{itemize}
    \item the interviews will be analyzed using the Qualitative Content Analysis of Mayring
    \begin{itemize}
        \item Why this method and not a quantitative method or different qualitative? -> Steigleider citation about how it is both the broadest and most exact technique. 
        \item How does the analysis work? Which process steps are there?
        \item Choose a summarizing content analysis and explain the steps for this
    \end{itemize}
\end{itemize}

\subsection{Expert interviews}

„[An] interview is the most direct method, among all the probes, of assessing a person’s understanding“ \cite{WhiteProbingunderstanding1992}

\subsection{Qualitative Content Analysis}

Mayring Vorgehen vereinfacht deutsch

Festlegung des Materials: Welches Material wird analysiert? Z.B. nur relevante Interviewabschnitte.

Analyse der Entstehungssituation: Wie ist die Situation zu kenn­zeichnen? Z.B. Auflistung anwesender Personen oder Betrachtung des soziokulturellen Rahmens.

Formale Charakterisierung des Materials: In welcher Form liegt das Material vor? Z.B. als wörtliche Transkription.

Richtung der Analyse: Worauf soll sich der Interpretationsfokus richten? Z.B. eher kognitive oder eher emotionale Aspekte be­trachten.

Theoriegeleitete Differenzierung der Fragestellung: Nach welcher Forschungsfrage wird das Material untersucht?

Bestimmung der Analysetechnik: Welches Verfahren soll bei der Materialanalyse eingesetzt werden? Z.B. Zusammenfassung, Explikation oder Strukturierung (s.u.).

Definition der Analyseeinheit: Welche Kriterien werden bei der Auswahl und Kategorisierung von Textabschnitten angelegt? Dabei legt die Kodiereinheit den kleinsten und die Kontexteinheit den größten Materialbestandteil fest, welche noch in eine Kategorie fallen. Schließlich bestimmt die Auswertungseinheit die Abfolge bei der Bearbeitung der Textabschnitte (Flick, 2002, S. 280).

Analyse des Materials: eigentlicher Analysevorgang, bei dem eine oder mehrere der drei verfügbaren Techniken angewendet wird.

Interpretation, um abschließend in Richtung der Hauptfragestellung die einzelnen Fälle zu generalisieren.

\section{Data collection via Expert Interviews}

\section{Gathering existing solution approaches}