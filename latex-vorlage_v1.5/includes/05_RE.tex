\chapter{Requirements gathering} \label{chap:ReqEng}

This chapter determines which requirements are needed in order to solve the problem. For this purpose, the problem needs to be atomized in greater detail than its statement in the introduction. Following the relevance cycle as presented in \ref{topic:relevance cycle}, the process of gathering the requirements for the solution gives additional insights into the problem space, which in turn helps define the solution and the requirements the artifact must meet to solve the problem. The chapter discusses the methods (expert interviews and qualitative content analysis after Mayring) used to elicit the requirements and places this process in the context of the requirements development process. In the ensuing section, the data is collected in form of expert interviews and the content is qualitatively analyzed. In the third section, existing approaches to solve this problem are investigated. These approaches may serve as input for the ensuing design process. This chapter lays the groundwork for the solution definition. It serves as input for the following chapter. 

\section{Research method and objective}
The requirement gathering process for this research follows closely the requirements elicitation activity, which is an activity of requirements engineering in software engineering. Requirements engineering is a set of activities to develop and document the specifications a system is expected to meet.\footcite[Cf.][p.16]{SommervilleIntegratedrequirementsengineering2005} The information is gathered from a number of people that develop, interact with or have an interest in the system.\footcite[Cf.][p.38]{PatakiSystemRequirementsAnalysis2003} and ultimately describes what these stakeholders really need.\footcite[Cf.][p.1]{GoguenTechniquesrequirementelicitation1993}

Firstly, it is important to clearly understand what requirements are: Requirements are descriptors of what a system should do. They define the services the ideal system must provide and the constraints under which the system must operate, effectively describing the business needs to which the system responds.\footcites[Cf.][p.100]{SommervilleSoftwareengineering2011}[cf.][p.95]{IEEEIEEEstandardglossary1990} Requirements can range from high-level abstract statements of services or system constraints to very detailed functional specifications\footcite[Cf.][p.215]{DavisSoftwarerequirementsobjects1993} and often are grouped in two categories: 
\begin{itemize}
    \item \textbf{Functional requirements} A functional requirement defines a function the system must perform.\footcite[Cf.][p.34]{IEEEIEEEstandardglossary1990} It specifies how the system should react to particular inputs and how the system should behave in specific situations.\footcite[Cf.][p.102]{SommervilleSoftwareengineering2011} It defines what actions the system must do (functionality) rather than how it performs these actions. 
    \item \textbf{Non-functional requirements} A non-functional requirement refers to additional requirements which put constraints on the functionality.\footcite[Cf.][p.102]{SommervilleSoftwareengineering2011} These constraints may be a consequence of product requirements, organizational policies or external factors (e.g. reliability, standard processes, interoperability) and often apply to the whole system rather than individual features.\footcite[Cf.][p.102]{SommervilleSoftwareengineering2011}
\end{itemize}

The fact that the requirements gathering is separated from the design process helps find an objective solution. This means, that the solution is not influenced by design considerations and the requirements don't limit the implementation of the system.\footcite[Cf.][p.18]{SommervilleIntegratedrequirementsengineering2005}

These requirements are collected during the requirements elicitation phase, which is the first of a set of activities in requirements engineering.\footcites[Cf.][p.116]{SommervilleSoftwareengineering2011}[cf.][p.17]{SommervilleIntegratedrequirementsengineering2005} The requirements engineering process is somewhat ill-defined, with different authors presenting different activities and practitioners using a large number of different methods to develop the requirements.\footcite[Cf.][p.225]{ZhangEffectiverequirementsdevelopmentA2007} The main aspects of the requirements engineering process is visualized in figure \ref{fig:REAll}. The process starts with the elicitation of the requirements from a variety of knowledge sources in order to collect needed information. The requirements are consequently analyzed in order to detect overlaps and conflicts and the generated knowledge is then added to the understood knowledge.\footcite[Cf.][p.17]{SommervilleIntegratedrequirementsengineering2005} That understood knowledge serves as input for the next step, the specification, where the requirements are structured and recorded in a document (the structured requirements specification (SRS)). The documentation may be done via natural language or dedicated conceptual models, so that the requirements are clear, understandable and correct, and constitutes the final output of the overall requirements engineering process.\footcite[Cf.][p.17]{SommervilleIntegratedrequirementsengineering2005}[chapter 1]{PohlRequirementsengineeringfundamentals2011} The fourth activity comprises the validation of these requirements. It checks again if the requirements are correct and that they indeed correspond to the stakeholders' needs.\footcite[Cf.][p.17]{SommervilleIntegratedrequirementsengineering2005}
It is an iterative cycle, as inconsistent, ambiguous or missing information that is uncovered during the specification or validation steps is tried to be resolved in another elicitation session. 
Major players in the literature agree on these four main activities,\footcites[Cf.][p.225]{ZhangEffectiverequirementsdevelopmentA2007}[cf.][p.220]{DavisSoftwarerequirementsobjects1993}[cf.][chapter 1]{PohlRequirementsengineeringfundamentals2011}[cf.][p.116]{SommervilleSoftwareengineering2011} but Sommerville and Pohl and Rupp also add negotiation (reconciling conflicting stakeholders' views) and management (actively controlling the changes to the requirements) to create a full picture of the requirements engineering process.\footcites[Cf.][p.17]{SommervilleSoftwareengineering2011}[cf.][chapter 1]{PohlRequirementsengineeringfundamentals2011}


\begin{figure}
    \centering
    \includesvg[width=\textwidth]{graphics/Zhang_textcurves}
    \caption[Summary of the requirements engineering process.]{Summary of the requirements engineering process.\footnotemark}
    \label{fig:REAll}
\end{figure}
\footnotetext{With changes taken from \cite{ZhangEffectiverequirementsdevelopmentA2007}, p.225}
Since this research bases the requirements gathering on the requirements elicitation activity. The activity and the methods used to execute this activity are examined in further detail in the following paragraph. 

\paragraph{Requirements Elicitation} Requirements elicitation refers to the gathering of information to extract the requirements that the system has to fulfill. During the first activity in the requirements engineering process, possible sources of information that allow the discovery of requirements need to be identified.\footcites[Cf.][p.2]{TiwariMethodologySelectionRequirement2017}[cf.][p.17]{SommervilleIntegratedrequirementsengineering2005} These knowledge sources may be the stakeholders (users, project sponsors, managers) as presented in figure \ref{fig:REAll}, or other resources (e.g. existing systems). It is important to not only discover but to fully understand the needs of these potential stakeholders in order to communicate them to the system developers.\footcite[Cf.][p.21]{ZowghiRequirementselicitationsurvey2005} Therefore, the elicitation of the requirements is an important and critical step in the RE process.\footcites[Cf.][p.232]{ZhangEffectiverequirementsdevelopmentA2007}[cf.][p.19]{ZowghiRequirementselicitationsurvey2005} Consequently, the use of appropriate methods is imperative to a well-executed and comprehensive requirement elicitation.\footcite[Cf.][p.232]{ZhangEffectiverequirementsdevelopmentA2007}

The gathering of information, and subsequently of the requirements may be done in a multitude of ways.\footcite[Cf.][p.170]{HickeyElicitationtechniqueselection2003} Independently of which method is applied, due to the communicative nature of eliciting information, the effective communication between the researcher and the stakeholders is of key importance.\footcite[Cf.][p.1]{AlvarezDiscourseanalysisrequirements2002} These methods do not necessarily originate from a computer sciences background, instead they are mostly derived from natural sciences, knowledge engineering and practical experience.\footcite[Cf.][p.19]{ZowghiRequirementselicitationsurvey2005}

The list of available methods below is divided into categories, which roughly describe the common nature of the methods within that category: 
\begin{itemize}
    \item \textbf{Conversational methods}: Interview, questionnaire, survey\footcites[Cf.][chapter 3]{PohlRequirementsengineeringfundamentals2011}[cf.][p.170]{HickeyElicitationtechniqueselection2003}
    \item \textbf{Observational methods}: Social analysis, observation, ethnographic study\footcites[Cf.][p.227]{ZhangEffectiverequirementsdevelopmentA2007}[cf.][p.173]{HickeyElicitationtechniqueselection2003}
    \item \textbf{Creative methods}: Brainstorming, analogy techniques\footcite[Cf.][chapter 3]{PohlRequirementsengineeringfundamentals2011}
    \item \textbf{Analytic methods}: Requirement reuse, documentation study, protocol analysis, discourse analysis \footcites[Cf.][p.12]{GoguenTechniquesrequirementelicitation1993}[cf.][pp.227-228]{ZhangEffectiverequirementsdevelopmentA2007}[cf.][p.2]{TiwariMethodologySelectionRequirement2017}
    \item \textbf{Synthetic methods}: Scenarios, prototyping, joint application development, perspective based reading\footcites[Cf.][p.228]{ZhangEffectiverequirementsdevelopmentA2007}[cf.][chapter 3]{PohlRequirementsengineeringfundamentals2011}[cf.][p.3]{TiwariMethodologySelectionRequirement2017}
\end{itemize}

Conversational methods, especially interviews, form the primary method to effectively collect knowledge, because they provide means of verbal communication between the researcher and one or more stakeholders and provide for them a natural way to express their ideas, problems, and questions. On the basis of the resulting conversation, the requirements are formulated.\footcite[Cf.][pp.226/227]{ZhangEffectiverequirementsdevelopmentA2007} The observational methods put the researcher in a more passive role, where she/he observes the stakeholders in a certain situation and notes the work process, potential mistakes, risks, and open questions, which constitute the inputs for the formulation of the requirements.\footcite[Cf.][chapter 3]{PohlRequirementsengineeringfundamentals2011} Creative methods, as the name suggest, serve the purpose to discover new and innovative requirements.\footcite[Cf.][chapter 3]{PohlRequirementsengineeringfundamentals2011} Analytic methods focus on extracting requirements from existing documents, making them especially useful to quickly formulate fine-grained requirements from existing documentation.\footcite[Cf.][p.228]{ZhangEffectiverequirementsdevelopmentA2007} Finally, synthetic methods are a combination of conversational, observational, creative, and analytic techniques.\footcite[Cf.][p.228]{ZhangEffectiverequirementsdevelopmentA2007} All techniques and methods have their strengths and weaknesses, the choice therefore depends on a number of factors, for example the nature of the project, its size, or its application domain.\footcite[Cf.][p.42]{ZowghiRequirementselicitationsurvey2005}
In practice, to make use of their relative strengths and to limit their weaknesses, a combination of methods is used. 


As \cite{WhiteProbingunderstanding1992} stated: \enquote{[An] interview is the most direct method, among all the probes, of assessing a person’s understanding} His view is shared with a great number of authors in the field.\footcites[Cf.][p.174]{MacaulayRequirementscapturecooperative1993}[cf.][p.105]{SommervilleSoftwareengineering2011}[cf.][p.25]{ZowghiRequirementselicitationsurvey2005}[cf.][p.172]{HickeyElicitationtechniqueselection2003}[cf.][p.227]{ZhangEffectiverequirementsdevelopmentA2007}[cf.][p.92]{MasonQualitativeresearching2002}
Interviews were and remain the most commonly used approach to elicit information. An interview is defined as motivating the interviewee to share their information, opinion, attitude and knowledge regarding certain topics.\footcite[Cf.][p.133]{KrugerqualitativeInhaltsanalyseMethode2004} Interviews did not get this status as No.1 without reason, they allow to identify the facts and subjective opinions of the stakeholders by face to face conversation, but also to uncover conflicts or politics.\footcite[Cf.][p.2]{TiwariMethodologySelectionRequirement2017} The requirements can directly be extracted from the stakeholders' verbalized thoughts and if needed, the conversation allows further queries to treat specific topics more thoroughly and discover hidden requirements. Due to these characteristics, interviews are deemed the best approach to gather the requirements for the dashboard.

Because interviews are so popular and widely used, a number of variations have developed that may be categorized by different aspects. An interview may be closed/structured (pre-defined set of questions, looking for clear answers) semi-structured (a more flexible set of questions allowing the researcher to adapt/innovate)\footcites[Cf.][p.39]{EdwardsWhatqualitativeinterviewing2013} or open-ended/unstructured (a discussion to discover requirements together, stakeholder talks from their own perspective).\footcites[Cf.][p.2]{TiwariMethodologySelectionRequirement2017}[cf.][p.40]{EdwardsWhatqualitativeinterviewing2013} The two latter types are qualitative interviews, characterized by less structured information and more flexibility in conversation flow.\footcite[Cf.][p.13]{EdwardsWhatqualitativeinterviewing2013} These types of interviews are chosen for when the data is most feasibly generated by asking the stakeholders for their opinion and stands about certain topics.\footcite[Cf.][p.76]{MasonQualitativeresearching2002} One instant of qualitative interviews are expert interviews, which are described in more detail in the following paragraph.


\begin{itemize}
    \item the requirements will be extracted from expert interviews
    \begin{itemize}
        \item Why an interview? X
        \item What kind of interview? explorative expert interview
        \item What experts? Frank Bloch, Bernd Kammholz, Torsten Milsmann, Daniel Kaltenbach, Thomas Maier -> all experts in their respective field
        \item What questions should I ask them? What do you not understand about blockchain? What would you wish to understand the most? Where do you feel lies the difficulty to blockchain and its transactions? What do you imagine under such a term as a dashboard visualizing blockchain transactions? What would be the best way to learn about it? Do you want it to be portable?
    \end{itemize}
    \item the interviews will be analyzed using the Qualitative Content Analysis of Mayring
    \begin{itemize}
        \item Why this method and not a quantitative method or different qualitative? -> Steigleider citation about how it is both the broadest and most exact technique. 
        \item How does the analysis work? Which process steps are there?
        \item Choose a summarizing content analysis and explain the steps for this
    \end{itemize}
\end{itemize}

\subsection{Expert interviews}
An expert interview is a semi-structured interview used for explorative purposes.\footcites[Cf.][p.179]{Flickintroductionqualitativeresearch2009}[cf.][p.465]{MeuserExperteninterviewkonzeptionelleGrundlagen2009}[cf.][p.31]{BognerInterviewsmitExperten2014} These purposes include 1) the orientation in a new field of research, 2) the collection of context information to complement other research methods, and 3) the development of a theory or typology by synthesizing knowledge from various experts.\footcite[Cf.][p.180]{Flickintroductionqualitativeresearch2009} The knowledge generated from these interviews is focused on a particular body of knowledge, the specific knowledge the expert has gained by specializing in a certain field.\footcites[Cf.][p.423]{BuberQualitativeMarktforschungKonzepte2007}[cf.][p.450]{PfadenhauerExperteninterviewGesprachauf2007} It gathers the expert's knowledge on their field of expertise and explicitly documents it.\footcites[Cf.][p.466]{MeuserExperteninterviewkonzeptionelleGrundlagen2009}[cf.][p.451]{PfadenhauerExperteninterviewGesprachauf2007}[cf.][p.172]{HickeyElicitationtechniqueselection2003} and even though it is a complex and elaborate method to generate data, it is applied in various fields of research due to the high quality of insights gained from the expert's knowledge.\footcites[Cf.][p.459]{PfadenhauerExperteninterviewGesprachauf2007}[cf.][p.442]{MeuserExpertInneninterviewsvielfacherprobt1991}[cf.][p.424]{BuberQualitativeMarktforschungKonzepte2007} 
%The expert interview in itself may be categorized into four different types depending on the interview's purpose.

The type of expert interview led during this research is of explorative nature. It helps gain insights into the problem space, and thus allows the better definition of the problem and the generation of hypotheses on how to solve this problem.\footcite[Cf.][p.28]{BognerInterviewsmitExperten2014} The main goals of this type of interviews are 1) to collect relevant information about the research environment and the problem space as well as 2) to discover the experts' opinion and attitudes regarding that problem.\footcite[Cf.][pp.28/29]{BognerInterviewsmitExperten2014} Therefore, this type of interview is well suited for the purpose of gathering the requirements for the artifact as well as to collect more information regarding the problem. Other variants of expert interviews are of systematizing (systematic extraction of knowledge regarding a particular topic), sustaining (scientific definition, explanation and connection of terms on the basis of the expert's knowledge) or theory-generative (discovery of implicit decision criteria, routines, or world views on the basis of the expert's subjective opinion.\footcites[Cf.][pp.27-30]{BognerInterviewsmitExperten2014}[cf.][p.421]{AghamanoukjanQualitativeInterviews2007}[cf.][p.180]{Flickintroductionqualitativeresearch2009}

But who exactly may call themselves an expert? 

3. Define expert interview: 

\subsection{Qualitative Content Analysis}

Mayring Vorgehen vereinfacht deutsch

Festlegung des Materials: Welches Material wird analysiert? Z.B. nur relevante Interviewabschnitte.

Analyse der Entstehungssituation: Wie ist die Situation zu kenn­zeichnen? Z.B. Auflistung anwesender Personen oder Betrachtung des soziokulturellen Rahmens.

Formale Charakterisierung des Materials: In welcher Form liegt das Material vor? Z.B. als wörtliche Transkription.

Richtung der Analyse: Worauf soll sich der Interpretationsfokus richten? Z.B. eher kognitive oder eher emotionale Aspekte be­trachten.

Theoriegeleitete Differenzierung der Fragestellung: Nach welcher Forschungsfrage wird das Material untersucht?

Bestimmung der Analysetechnik: Welches Verfahren soll bei der Materialanalyse eingesetzt werden? Z.B. Zusammenfassung, Explikation oder Strukturierung (s.u.).

Definition der Analyseeinheit: Welche Kriterien werden bei der Auswahl und Kategorisierung von Textabschnitten angelegt? Dabei legt die Kodiereinheit den kleinsten und die Kontexteinheit den größten Materialbestandteil fest, welche noch in eine Kategorie fallen. Schließlich bestimmt die Auswertungseinheit die Abfolge bei der Bearbeitung der Textabschnitte (Flick, 2002, S. 280).

Analyse des Materials: eigentlicher Analysevorgang, bei dem eine oder mehrere der drei verfügbaren Techniken angewendet wird.

Interpretation, um abschließend in Richtung der Hauptfragestellung die einzelnen Fälle zu generalisieren.

\section{Data collection via Expert Interviews}

The explorative expert interview is ideally suited for the purpose of requirements gathering regarding the dashboard design process, because collecting a rather broad palette of information to explore why the problems in understanding blockchain transactions exist and how they might be solved. By leading an open, little standardized conversation,  

\paragraph{Sampling} Who is interviewed and why?

\paragraph{Design of the question compendium} What kind of questions are going to be asked?



\section{Gathering existing solution approaches}