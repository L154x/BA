\chapter{Implementation and Evaluation}

\section{Exemplary demonstration: Onboarding new students} \label{sec:demo}
circa eine Seite

demonstration serves as a light-weight evaluation to prove the feasibility, that the artifact solves one of the problem instantiations

The demonstration is done for new students that are going to ework in the department of StudyLab which treats the blockchain technology. 

\section{Evaluation} \label{sec:Evaluation}

The exemplary demonstration lends itself excellently for a more thorough evaluation of the created artifact. As stated by \cite{HevnerDesignScienceResearch2004} and agreed upon by the design science research community, such evaluation plays a \enquote{crucial} role in design science research. This importance is further emphasized by \cite{MarchDesignnaturalscience1995} who regard evaluation as one of the two activities in design science (see \ref{subsec:GuidelinesDesignScience}).\footcites[Cf.][p.258]{MarchDesignnaturalscience1995}[Cf. in addition][]{PfeffersDesignScienceResearch2007}{HevnerDesignScienceResearch2004}{Pries-HejeComprehensiveFrameworkEvaluation2012}{Pries-HejeStrategiesDesignScience} 

Importance of evaluation in design science 
evidence that the artifact is useful

a lot of different evaluation methods are available. As \cite{PfeffersDesignScienceResearch2012} found out, most use technical experiments followed by illustrative scenarios. However, the method chosen depends largely on the type of artifact and qualities of the artifact to be evaluated. In Information Science Research, more qualitative approaches are usually chosen such as case studies or subject based experiments.\footcites[Cf.][p.4 et seq]{PfeffersDesignScienceResearch2012} 

validity, utility, quality, efficacy; Purposeful? and implementability?

\subsection{Method} \label{subsec:EvaluationMethod}

\paragraph{Why was this evaluation method chosen?} -> Ex-Post strategy, naturalistic, 
\footcite{PfeffersDesignScienceResearch2012} \cite{Pries-HejeComprehensiveFrameworkEvaluation2012} \cite{Pries-HejeStrategiesDesignScience}

\paragraph{Participants}
Gruppen von 8 Personen aufgeteilt in 2 Lager -> eine Gruppe sieht sich die zwei meistgeschauten YouTube-Videos an und die andere durchläuft das Tutorial und die Overview.
Bei der Auswahl werden die Fragen zur Bestimmung der Zielgruppe gestellt. Nur, wenn sie in die Zielgruppe fallen, kommen sie als Teilnehmer infrage.

\paragraph{Materials}
die Videos + kurze Beschreibung;
meine Webseite;
Fragebogen
\paragraph{Procedure}
Vor Beginn wird der Fragebogen durchlaufen, nicht alle Fragen, aber die für den Vergleich schon.
Die Testpersonen sehen sich alle gleichzeitig die Videos sowie das Tutorial an.
Sofort nach Ende der Durchführung werden die Teilnehmer gebeten an ihren Arbeitslaptops den Fragebogen in schriftlicher Form auszufüllen. 


\subsection{Results} \label{subsec:EvaluationResults}
- Scores and answers

\subsection{Discussion}

\section{Identification of possible optimizations} \label{sec:Optimizations}

How could the artifact be refined? As Design Science Research is an iterative process -> will lead to further research for the future.
