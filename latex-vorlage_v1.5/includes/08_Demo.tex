\chapter{Implementation and Evaluation} \label{chap:Evaluation}

This chapter evaluates the artifact. First, the artifact's feasibility is demonstrated by utilizing it as an alternative to existing learning resources for new students of the organization's blockchain team. This demonstration is then extended to serve as the basis for a rigorous naturalistic evaluation which assesses the most relevant properties of the evaluand. The results are presented and discussed. Finally, on the basis of the discussion's insights, possible improvements to the artifact are presented.

\section{Exemplary demonstration: Onboarding new students to the blockchain team} \label{sec:demo}

Once the artifact is developed, a suitable context needs to be identified in which the artifact may be implemented and demonstrated. As the artifact's purpose is to give a short overview of blockchain technology and its use cases to people who are yet unfamiliar with this technology, the context must present an environment that applies to the problem space. Possible situations may be: informal client meetings, seminars, various conferences, or the onboarding of new employees. This paper suggests that the artifact may be utilized in any of these situations.

The demonstration serves as a light-weight evaluation to prove the artifact's feasibility, i.e., that the artifact solves one instantiation of the formulated problem. In this case, the instantiation of the problem is that of the onboarding process of corporate students to the blockchain team, which is the enterprise's \enquote{StudyLab} department working on blockchain related projects. Contrary to recent hires, corporate students are already familiar with the organization's culture, structure, communication, and tools. Therefore, the onboarding process does not focus on introducing these students to the organization. It solely focuses on introducing them to blockchain technology and its use cases. Generally, the students are given access to a \enquote{Student Learning Guide} which is a collection of various blockchain learning resources, including $\geq$ 60 hours of video material, numerous introductory articles as well as various technical resources regarding specific blockchain implementations. Although the learning guide is extensive, working through the resources is time intensive, and novices appear to have problems gaining an initial high-level understanding because they focus on technical details.\footcites[Cf.][]{RalphB_Interview} Hence, this situation offers an appropriate opportunity to assess the artifact's feasibility.

Since the artifact is a web page, it is quickly deployed and may be accessed from within the organization's intranet. New students joining the blockchain team are given the link to the artifact (also referred to as the tutorial) and are asked to go through it. Because no other appropriate situations were available for a more thorough evaluation at that point of time, and because the described problem context is considered excellently suited to prove not only the artifact's feasibility but also its quality and effectiveness, the demonstration of the artifact is extended to also serve as the evaluation. The next section describes the evaluation in more detail.

%The demonstration is done for new students that are going to work in the department of StudyLab which treats the blockchain technology. 

\section{Evaluation} \label{sec:Evaluation}

The demonstration lends itself excellently for a more thorough evaluation of the created artifact. As stated by \cite{HevnerDesignScienceResearch2004} and agreed upon by the design science research community, such evaluation plays a \enquote{crucial} role in design science research.\footcites[Cf.][p.258]{MarchDesignnaturalscience1995}[Cf. in addition][]{PfeffersDesignScienceResearch2007}{HevnerDesignScienceResearch2004}{Pries-HejeComprehensiveFrameworkEvaluation2012}{Pries-HejeStrategiesDesignScience} A rigorous evaluation of the artifact and its qualities must take place to provide evidence that the artifact achieves its purpose, i.e., that it solves the identified problem.\footcites[Cf.][p.425]{Pries-HejeComprehensiveFrameworkEvaluation2012} The evaluation should furthermore assess how well the artifact solves the problem. This is done by first identifying the relevant evaluation criteria. These may include - in addition to the in guideline 3 \textit{Design Evaluation} described properties of a design artifact (utility, quality, and efficacy) - properties such as Checkland and Scholes five E's (efficiency, effectiveness, efficacy, ethicality, elegance), or the artifact's purposefulness and implementability.\footcites[Cf.][p.427]{Pries-HejeComprehensiveFrameworkEvaluation2012}[cf. in addition][]{ChecklandSoftSystemsMethodology1990}
% Overall, the candidates of evaluation criteria should be chosen in such a way that they adequately identify the relevant strengths and short comings of the artifact so that its utility may be adequately assessed. At first, focusing on certain criteria and disregarding others may appear to inhibit a rigorous approach that is necessary to ensure the scientific nature of the artifact. But this selection of important criteria in fact ensures something of similar interest, the relevance of the applied methods with regard to the overall relevance of the design science research.\footnote{For more information about the interplay of relevance and rigor in design science research, see \ref{topic:relevance cycle}}

% While it is important to rigorously conduct the relevant evaluation methods in order to ensure the scientific nature of the design science research, In other words, even though rigorous evaluation methods must be executed to ensure the design science process is of scientific nature, the evaluation should nonetheless focus on the important properties in order to remain relevant.

Numerous different evaluation methods exist, ranging from observational, analytical, and descriptive to experimental, and testing methods such as illustrative scenarios, technical experiments, case studies, or action research.\footcites[Cf.][p.86]{HevnerDesignScienceResearch2004}[cf.][p.4 et seq]{PfeffersDesignScienceResearch2012} %Yet, these do not represent the entire list of relevant evaluation methods available to design science researchers. As \cite{PfeffersDesignScienceResearch2012} found out, the majority of researchers conduct technical experiments, followed by illustrative scenarios and case studies. However, the method chosen depends largely on the type of artifact and its qualities to be evaluated. In the case of information systems research, more qualitative approaches such as case studies or subject-based experiments are pursued as these are deemed more relevant.\footcites
The following paragraphs present the chosen evaluation method based on Pries-Heje et al.'s evaluation strategy framework before they illustrate and discuss the results of that evaluation. 

\subsection{Method} \label{subsec:EvaluationMethod}

While Hevner et al. have identified numerous different evaluation methods, they have not offered any guidance on the choice of appropriate evaluation strategies. It is for this reason that Pries-Heje et al. have proposed a framework for evaluation in design science research which constitutes the scientific basis of this evaluation process.\footcites[Cf.][p.11 et seq]{Pries-HejeComprehensiveFrameworkEvaluation2012} The framework presents a matrix that compares artificial and naturalistic evaluation methods with ex ante (before artifact construction) and ex post (after artifact construction) evaluation methods. The authors also suggest a four-step approach which (1) analyzes the requirements for the evaluation, (2) connects these requirements to the dimensions in the matrix, (3) selects the appropriate evaluation method that aligns with the chosen strategy quadrants in the matrix, and (4) conducts the evaluation.\footcites[Cf.][p.13]{Pries-HejeComprehensiveFrameworkEvaluation2012} 

\paragraph{Method selection} Corresponding to the first step, the requirements for the evaluation context are defined. The evaluation must assess the utility, efficacy, efficiency, purposefulness, and implementability. Ethicality and elegance, as well as other properties, may be disregarded as, considering the constraints of time, cost, and resources, they do not help to directly identify how well the instantiation solves the problem. Since the problem statement also infers that existing blockchain explanations are insufficient, the evaluation must also compare the developed artifact against existing explanations. Taking these requirements into consideration, a naturalistic ex post evaluation method is deemed the most fitting. Naturalistic evaluations evaluate the artifact in its real environment, and even though they are more costly to perform and may suffer from misinterpretation or confounding variables, a naturalistic approach is considered to be the \enquote{real proof of the pudding}.\footcites[Cf.][p.3 et seqq]{Pries-HejeStrategiesDesignScience}[cf.][p.80]{PfeffersDesignScienceResearch2012} Once an appropriate strategy is selected, the third step is concluded by deciding on the best fitting evaluation method. Possible methods, among others, are action research, case studies, participant observation and surveys. The latter two are considered most appropriate for this evaluation as they allow real users to test the artifact in a real environment, and surveys collect the evaluation data (whether quantitative or qualitative) in the form of questionnaires. 

\paragraph{Materials} The material for the evaluation comprises a questionnaire as well as the evaluand (the artifact) and the two most relevant YouTube videos (\href{https://www.youtube.com/watch?v=r43LhSUUGTQ}{\textit{Understand the Blockchain in Two Minutes} by the Institute for the Future} and \href{https://www.youtube.com/watch?v=SSo_EIwHSd4}{\textit{How does a blockchain work} by Simply Explained Savjee}) for comparison. The videos, previously identified during the requirements gathering phase, are well suited as a comparison as they cover the same concepts as presented in the artifact but lack its interactivity. The questionnaire is divided into two sections. The first part tests the participants' level of knowledge before viewing the artifact or the videos, and the other part tests the participants' level of knowledge afterward. This allows the identification of changes in their understanding caused by the explanations. The questionnaire is included in the appendix \ref{anhang:EvaluationQuestionnaire}. 

\paragraph{Participants and procedure}
For this evaluation, 22 students from the HPE Dualstudy program are asked to participate. While the sample size is relatively small and not representative of the total population of \enquote{novices}, this convenience sample should be sufficient to collect a variety of opinions allowing an assessment of the artifact's qualities. The participants are given a brief description of the overall project and the purpose of the evaluation before being asked to fill in the first part of the questionnaire. The division of the questionnaire serves two purposes: 1) It includes the questions defined in figure \ref{fig:TargetGroup} used to identify the participants that match the target group and 2) some questions are asked both before and after the participants learn about blockchain to assess their growth of understanding. Of the 22 participants, four fall outside of the target group and do not partake in the evaluation. The remaining 18 participants are grouped in two groups (control group and experimental group) based on their given answers to question 2 (their perceived understanding of blockchain).\footnote{The division is executed manually by comparing the two highest values and putting the respective participants in separate groups. If both values are equal, then the next two values are compared in the same way. This continues until there are either no more participants to be grouped (example: six participants with the levels of 6, 6, 4, 4, 1, 1) or until the two highest values are not equal. Then the participant with the highest value is put in the first and the latter in the second group. To counterbalance this, should such a case occur again, the participant with the higher value is then assigned to the second group and vice versa. This continues until all participants are assigned to a group.}
They are then given access to the second part of the questionnaire as well as the videos (group 1) or the tutorial (group 2). If during this process any questions appear, the researcher is available to respond to the participants' questions. They are asked to view the resources, to respond to the remaining questions directly afterward, and to send their responses via email to the researcher. The next section presents these results.

\subsection{Results} \label{subsec:EvaluationResults}

To put the artifact's qualities to a test, the evaluation focuses on the question whether and how well the artifact achieves its goal of explaining blockchain technology to a novice audience compared to existing resources such as videos. It is for this reason that the questionnaire is divided into two sections containing redundant questions (among others), hence not only allowing a comparison of the two different groups (videos and tutorial) but also their learning progress (a priori and a posteriori). The questions and results can be found in \ref{anhang:EvaluationQuestionnaire} and \ref{anhang:ResponsesQuestionnaire}.

\paragraph{Questions 1 and 5}The first questions 1a-e establish that the participants fit the target group. The participants are exposed to the same questions again (5a-e) after having reviewed their respective resources to determine whether they have gained enough understanding to no longer be considered \enquote{novices}.

As illustrated by figures \ref{fig:ResultsVideos} and \ref{fig:ResultsTutorial}, the a priori responses show that almost all participants have heard of bitcoin (17 of 18) and blockchain (15 of 18), but fewer feel they understand bitcoin (10 of 18) or blockchain (9 of 18). Contrary to these results, one participant responds positively to the final question of this set (\enquote{Do you know what cryptoeconomics means?}) while the others - of which eight have responded exclusively positively to the previous questions - negate the final question. When presented with these questions for the second time, the participants answer more questions positively. While there is no change in responses to question 1a (18 of 19), the overall number of positive responses has risen significantly (+48\%). All participants state they know and understand blockchain technology and the final question also records a growth from 1 to 8 positive responses.

\begin{figure}[htbp]
    \centering
    \begin{tabular}{l r}
            \begin{tikzpicture}
	        \begin{axis}[name=plot1,
	            xbar stacked,
            	bar width=15pt,
                enlargelimits=0.15,
                legend style={at={(0.5,-0.20)},
                  anchor=north,legend columns=-1},
                xlabel={\# participants},
                xmin=0,
                xmax=9,
                symbolic y coords={1e, 1d, 1c, 1b, 1a},
                ytick=data,
                nodes near coords,
            	]
        	\addplot coordinates
        		{(1,1e) (4,1d) (7,1c) (5,1b) (9,1a)};
        	\addplot coordinates
        		{(8,1e) (5,1d) (2,1c) (4,1b) (0,1a)};
        	\legend{\strut yes, \strut no}
        	\end{axis}
            \end{tikzpicture}
            % \caption{Distribution of the \enquote{a priori} answers of group 2 (artifact)}
        & 
            \begin{tikzpicture}
	        \begin{axis}[
	            xbar stacked,
	            nodes near coords,
            	bar width=15pt,
                enlargelimits=0.15,
                legend style={at={(0.5,-0.20)},
                  anchor=north,legend columns=-1},
                xlabel={\# participants},
                symbolic y coords={5e, 5d, 5c, 5b, 5a},
                xmin= 0,
                xmax=9,
                ytick=data,
            	]
        	\addplot coordinates
        		{(2,5e) (9,5d) (9,5c) (7,5b) (9,5a)};
        	\addplot coordinates
        		{(7,5e) (0,5d) (0,5c) (2,5b) (0,5a)};
        	\legend{\strut yes, \strut no}
        	\end{axis}
            \end{tikzpicture}
            % \caption{Distribution of the \enquote{a posteriori} answers of group 2 (artifact)}
        \\
    \end{tabular}
    \caption{Distribution of the \enquote{a priori} (Q1) and \enquote{a posteriori} (Q5) answers of group 1 (videos)}
    \label{fig:ResultsVideos}
\end{figure}

\begin{figure}[htbp]
    \centering
    \begin{tabular}{l r}
            \begin{tikzpicture}
	        \begin{axis}[name=plot1,
	            xbar stacked,
            	bar width=15pt,
                enlargelimits=0.15,
                % legend style={at={(0.5,-0.20)},
                %   anchor=north,legend columns=-1},
                xlabel={\# participants},
                xmin=0,
                xmax=9,
                symbolic y coords={1e, 1d, 1c, 1b, 1a},
                ytick=data,
                nodes near coords,
            	]
        	\addplot coordinates
        		{(0,1e) (5,1d) (8,1c) (5,1b) (8,1a)};
        	\addplot coordinates
        		{(9,1e) (4,1d) (1,1c) (4,1b) (1,1a)};
        	% \legend{\strut yes, \strut no}
        	\end{axis}
            \end{tikzpicture}
            % \caption{Distribution of the \enquote{a priori} answers of group 2 (artifact)}
        & 
            \begin{tikzpicture}
	        \begin{axis}[
	            xbar stacked,
	            nodes near coords,
            	bar width=15pt,
                enlargelimits=0.15,
                % empty legend,
                % legend style={at={(0.5,-0.20)},
                %   anchor=north,legend columns=-1},
                xlabel={\# participants},
                symbolic y coords={5e, 5d, 5c, 5b, 5a},
                xmin= 0,
                xmax=9,
                ytick=data,
            	]
        	\addplot coordinates
        		{(6,5e) (9,5d) (9,5c) (8,5b) (8,5a)};
        	\addplot coordinates
        		{(3,5e) (0,5d) (0,5c) (1,5b) (1,5a)};
        	%\legend{\strut yes, \strut no}
        	\end{axis}
            \end{tikzpicture}
            % \caption{Distribution of the \enquote{a posteriori} answers of group 2 (artifact)}
        \\
    \end{tabular}
    \caption{Distribution of the \enquote{a priori} (Q1) and \enquote{a posteriori} (Q5) answers of group 2 (artifact)}
    \label{fig:ResultsTutorial}
\end{figure}

While there is no visible difference between the two different groups a priori,\footnote{As should be expected \& hoped for, since the two groups should be comparable for the evaluation} however, focusing on the responses a posteriori, in group 1 a significantly lower number responds positively to the final question. Seven participants of group 1 state they do not know what cryptoeconomics means, whereas in group 2 only three participants voice that opinion. Furthermore, a total number of seven participants answer all questions in the affirmative, thus no longer falling into the target group. Of these seven, five participants are part of group 2 (70\%).

\paragraph{Questions 2 \& 6} Tables \ref{tab:EvalResultsYesNoVIDEO} and \ref{tab:EvalResultsYesNoTUTORIAL} show the respective answers of group 1 (videos) and group 2 (artifact) to the question \enquote{[...] how confident do you feel about your knowledge about blockchain technology?}.
It is apparent from these tables that 78\% (14 out of 18 participants) feel more confident about their blockchain knowledge. However, there is a significant difference between the two different groups and the distance between both values depending on the initial (a priori) value. Firstly, the confidence level of group 1 has in average increased by 0.89 points while that of the latter group has increased by 1.67 (an increase of 100\% compared to group 1). Secondly, the increase is higher for participants who have stated a lower (<4) level than those who have stated a higher level initially. In fact, participants of level 7, 8, or 9 have not stated any change, while participants of lower levels have estimated an increase of (in average) 1.9 points (ranging from 1 to 3).   


\begin{table}[]
    \centering
    \begin{tabular}{l | l l l l l l l l l }
          Self-assessment & K1.1 & K1.2 & K1.3 & K1.4 & K1.5 & K1.6 & K1.7 & K1.8 & K1.9  \\
         \hline
         A priori (Q2) & 7 & 3 & 4 & 5 & 1 & 8 & 1 & 3 & 1 \\
         A posteriori (Q6) & 7 & 5 & 4 & 6 & 2 & 8 & 3 & 4 & 2 \\
          \hline 
         Difference ($\Delta_{2,6}$) & 0 & 2 & 0 & 1 & 1 & 0 & 2 & 1 & 1 \\
         Average & & & & & & & & & \textbf{0.89} \cellcolor[gray]{0.9} \\
    \end{tabular}
    \caption{Results of the self-assessment of group 1 before and after watching the resources}
    \label{tab:EvalResultsYesNoVIDEO}
\end{table}

\begin{table}[]
    \centering
    \begin{tabular}{l | l l l l l l l l l }
         Self-assessment & K2.1 & K2.2 & K2.3 & K2.4 & K2.5 & K2.6 & K2.7 & K2.8 & K2.9  \\
         \hline
         A priori (Q2) & 5 & 1 & 3 & 2 & 5 & 9 & 3 & 1 & 5 \\
         A posteriori (Q6) & 6 & 4 & 5 & 4 & 5 & 9 & 5 & 4 & 6 \\
         \hline
         Difference ($\Delta_{2,6}$) & 1 & 3 & 2 & 2 & 1 & 0 & 2 & 3 & 1 \\
         Average & & & & & & & & & \cellcolor[gray]{0.9} \textbf{1.67} \\
    \end{tabular}
    \caption{Results of the self-assessment of group 2 before and after doing the tutorial}
    \label{tab:EvalResultsYesNoTUTORIAL}
\end{table}

\paragraph{Questions 3 \& 7} The following question (Q3: \enquote{How would you explain the blockchain technology to your peers?}), which is identical with question 7, aims to assess the participants' knowledge of blockchain technology on a more qualitative level.\footnote{The responses, listed in \ref{anhang:ResponsesQuestionnaire}, are analyzed following Mayer's qualitative content analysis method.} By paraphrasing the responses and subsequently assigning those paraphrases to categories, it is possible to analyze the number of concepts known before and after accessing the resources. Table \ref{tab:responses3+7} shows the results of this analysis, which presents a significant increase of different concepts included in the a posteriori explanations.

% \begin{table}[]
%     \centering
%     \begin{adjustbox}{max width=\textwidth}
%     \begin{tabular}{l|c c c c c c c c }
%          Category & Linking blocks & Transactions & Digital value & Mining \& consensus & Hash functions & Trust & Accountability & Decentralisation\\
%          \hline
%         Group 1 (a priori) & 6 & 1 & 0 & 1 & 2 & 2 & 2 & 1 \\
%         Group 1 (a posteriori) & 7 & 3 & 1 & 3 & 3 & 4 & 5 & 8 \\
%         \hline
%         Group 2 (a priori) & 4 & 3 & 4 & 2 & 2 & 3 & 3 & 4 \\
%         Group 2 (a posteriori) & 5 & 3 & 7 & 4 & 6 & 6 & 4 & 7 \\
%         \hline
%         \hline
%         Total (a priori) & 10 & 4 & 4 & 3 & 4 & 5 & 5 & 5 \\
%         Total (a posteriori) & 12 & 6 & 8 & 7 & 9 & 10 & 9 & 15 \\
%     \end{tabular}
%     \end{adjustbox}
%     \caption{Different categories derived from the participants' responses to questions 3 \& 7}
%     \label{tab:responses3+7}
% \end{table}

\begin{table}[]
    \centering
    \begin{tabular}{l|cc|cc|cc}
    \multicolumn{1}{c}{Category}      & \multicolumn{2}{c}{Group 1} & \multicolumn{2}{c}{Group 2} & \multicolumn{2}{c}{Total} \\
          & a priori   & a posteriori  & a priori   & a posteriori   & a priori  & a posteriori  \\
        \hline
Linking blocks      &   6       &        7       &     4       &      5          &    10      &       12      \\
Decentralization    &   1       &        8       &     4       &       7         &     5      &      15       \\
Trust               &   2       &        4       &     3      &       6         &     5       &      10        \\
Accountability      &   2       &        5       &     3       &       4         &     5      &      9        \\
Hash functions      &   2       &        3        &     2       &   6         &     4         &      9       \\
Digital value       &   0       &        1       &      4      &         7       &     4      &      8        \\
Transactions        &   1       &        3       &     3       &     3        &     4         &       6      \\
Mining \& consensus &   1       &        3       &      2      &       4       &     3        &      7        \\
\end{tabular}
    \caption{Different categories derived from the participants' responses to questions 3 (a priori) \& 7 (a posteriori)}
    \label{tab:responses3+7}
\end{table}


Before accessing the resources, the majority of participants (10 of 18) mention \textit{Linking of blocks} in their explanation of blockchain. The categories \textit{Trust}, \textit{Accountability} and \textit{Decentralization} each appear five times in the a priori explanations. After the execution of the experiment, the number of times the different categories are included in the explanation increases for all respective categories. While the average rate of new concepts added to the explanations amounts to 4.5, it ranges from 2 (\textit{Linking blocks} as well as \textit{Transactions}) to 10 (\textit{Decentralization}). Because of this, \textit{Decentralization} appears in >80\% of all explanations (15 of 18) followed by \textit{Linking blocks} (12 of 18).
What is also interesting about this data is the difference between group 1's a posteriori responses compared to those of group 2. While explanations including \textit{Decentralization} increase by seven (from 1 to 8) in group 1, group 2 records a significantly smaller increase of three (from 4 to 7) appearances. In contrast to this, \textit{Hash functions} are significantly more often included in the a posteriori responses of group 2 (from 2 to 6) than in those of group 1 (from 2 to 3). In general, even in those cases where there is no difference between a priori and a posteriori responses with regard to the different categories,\footnote{Cf. K1.1, K2.6 in \ref{anhang:ResponsesQuestionnaire}} the explanations are more structured and detailed.\footnote{K1.4, Q3: \enquote{A chain of blocks in a row} and Q7: \enquote{A chain of data where the following one always saved the previous one to see if something was manipulated.[...]} or K2.6, Q3: \enquote{Blockchain enables digital money, similar to online banking.[...]} and Q7: \enquote{Blockchain enables digital money in a decentralized network. There, the transactions/data is saved in immutable blocks.[...]} in \ref{anhang:ResponsesQuestionnaire}}

\setlength\dashlinedash{0.2pt}
\setlength\dashlinegap{1.5pt}
\setlength\arrayrulewidth{0.3pt}

\begin{table}[]
    \centering
    \begin{tabular}{l|cc|cc|cc}
          & \multicolumn{2}{c}{Group 1} & \multicolumn{2}{c}{Group 2} & \multicolumn{2}{c}{Total} \\
          & a priori   & a posteriori  & a priori   & a posteriori   & a priori  & a posteriori  \\
        \hline
Financial/sending value &   3         &        6       &     4       &      7          &    7       &       13      \\
Smart contracts          &   4         &        7       &     0       &        2        &     4      &       9      \\
Decentralized storage   &     1       &        4       &      3      &         5       &     4      &      9        \\
Cryptocurrency          &   5         &       6        &      1      &       2         &     6      &      8        \\
\hdashline
Asset tracking          &     0       &       3        &     0       &       4         &     0      &       7       \\
Elections               &     0       &       2        &     0       &       2         &     0      &      4        \\
Machine payments        &     0       &       0        &     0       &       2         &     0      &      2        \\
Licensing               &     0       &       0        &     0       &       2         &     0      &       2       \\
\end{tabular}
    \caption{Different categories derived from the participants' responses to questions 4 (a priori) \& 8 (a posteriori)}
    \label{tab:results48}
\end{table}

\paragraph{Questions 4 \& 8} In response to the question \enquote{Why and for what purpose would somebody want to use blockchain technology?}), a range of different responses is elicited. Table \ref{tab:results48} shows the results when the responses are split into different statements and grouped by use cases as well as a priori and a posteriori responses.
Four broad themes have emerged from the analysis of the a priori question: The participants estimate that the blockchain technology may be used for financial use cases such as transactions of value (7), for the implementation of cryptocurrencies (6), for smart contracts (4), or for decentralized storage (4). Four of the 18 participants state, however, that they do not know of any use cases.\footnote{Cf. K1.3, K1.5, K1.7, and K2.8 in \ref{anhang:ResponsesQuestionnaire}} In contrast to this, all participants identify one or more use cases after having reviewed the resources. These cases also include newly identified use cases (see table \ref{tab:results48}). The most common case, similarly to the a priori result, is the \textit{Financial} use case which is included in 13 different responses, followed by the other three prior identified use cases.
Another interesting aspect of the data is the increase of responses which mention \textit{Asset tracking} as a use case. It has increased from non-existent (0) to seven, with a growth rate (+7) surpassing that of \textit{Financial/sending value} (+6). While there is no considerable difference between the two groups when comparing the a priori and a posteriori results for the first four use cases, the table gives substantial evidence that only participants of group 2 have identified the use cases \textit{Machine payments} and \textit{Licensing}.

\paragraph{Questions 9 \& 10}The two final questions ask the participants for their personal opinion to assess what features and properties they liked (Q9) and disliked (Q10). One prevailing view expressed by the majority of participants (8 of 9) of group 1 is that they liked the graphical presentation of the material. One interviewee states that the videos were \enquote{visually appealing}\footnote{K1.3 in \ref{anhang:ResponsesQuestionnaire}} and another comments that the visual explanation was very easy to understand.\footnote{Cf. K1.9 in \ref{anhang:ResponsesQuestionnaire}} In addition to the visual presentation, two participants have explicitly appreciated the succinct explanation.\footnote{K1.5: \enquote{sententiously} and K1.7: \enquote{You did not have to read a lot.} in \ref{anhang:ResponsesQuestionnaire}} However, the vast majority of group 1 (8 of 9) criticizes that the videos were lacking detailed information. Four participants express the opinion that they needed more information about the consensus mechanisms.\footnote{K1.9: \enquote{What is proof of work? The explanation was just too quick.} in \ref{anhang:ResponsesQuestionnaire}} Some would also like to learn more about possible use cases,\footnote{K1.8: \enquote{I was missing more exact examples for possible applications.}, cf. K1.5 in \ref{anhang:ResponsesQuestionnaire}} whereas others state they would like to learn more about hashes,\footnote{K.1.1: \enquote{How is a hash created?}, cf. K1.2 in \ref{anhang:ResponsesQuestionnaire}} smart contracts,\footnote{Cf. K1.3, K1.1 in \ref{anhang:ResponsesQuestionnaire}} and the network.\footnote{Cf. K1.6, K1.9 in \ref{anhang:ResponsesQuestionnaire}}
Further points of criticism are that the videos were not engaging enough and that the explanation's speed was too fast .\footnote{K1.7: \enquote{It was not interactive, I quickly lost the thread.}, cf. K1.9 in \ref{anhang:ResponsesQuestionnaire}}

Similarly to the view expressed by group 1, participants of group 2 also state that they liked the succinctness of the tutorial's explanation,\footnote{K2.1: \enquote{The explanation is compact and easy to understand [...]}, cf. K2.5, K2.6 in \ref{anhang:ResponsesQuestionnaire}} the graphical elements,\footnote{K2.1: \enquote{[..]The visual presentation with some interactivity makes the content much more interesting.[..]}, K2.3: \enquote{[I liked] that one worked with visual elements.}, cf. K2.4, K2.7 in \ref{anhang:ResponsesQuestionnaire}} and the practical examples.\footnote{K2.7: \enquote{The examples and visualizations}, cf. K2.8 in \ref{anhang:ResponsesQuestionnaire}} However, the most praised feature, as agreed upon by 90\% of group 2, is the possibility to interact with the information. The participants mention explicitly that they liked that they were able to \enquote{click through [the tutorial] by themselves},\footnote{K2.2 in \ref{anhang:ResponsesQuestionnaire}} to \enquote{discover new things}\footnote{K2.9 in \ref{anhang:ResponsesQuestionnaire}} in the overview, and to play around with the content.\footnote{Cf. K2.4 in \ref{anhang:ResponsesQuestionnaire}} Notwithstanding, participants also express that they disliked the user experience\footnote{K2.1: \enquote{The elements that are clickable should stand out from the rest, so you don't have to search the whole page and don't miss anything.}, K2.2: \enquote{A user guidance would be good, so that I know where I currently am.}, cf. K2.6 in \ref{anhang:ResponsesQuestionnaire}} and the amount of written text, wishing for more graphical elements.\footnote{K2.8: \enquote{More visual elements, less text.}, cf. K2.4 in \ref{anhang:ResponsesQuestionnaire}}

The results presented above allow the evaluation of the previously identified properties of the artifact (utility, efficacy, efficiency, and effectiveness). The following section discusses whether the artifact fulfills these properties, and - on a more fundamental level - how conclusive and reliable the evidence for this is. 

\subsection{Discussion}
By conducting a naturalistic ex post  evaluation in the form of participant observation and questionnaires, the evaluation is situated in a real setting and should hence offer a strong internal validity. This real environment takes the human factor into account, which is an important part of \ac{IS} research. However, due to the method of data collection, the human factor may also be a vulnerability to the evaluation's validity as many qualitative answers are subjective and not easily reproducible. This fact should be acknowledged before any further results are discussed.

The results of questions 1/5 and 2/6 provide tangible evidence that the designed artifact constitutes a better learning resource about blockchain technology than the YouTube videos which served as a comparison. The results show that not only do the participants feel in average 1.6 points more confident in their level of understanding, but five of nine participants no longer fit into the target group definition after reviewing the artifact, whereas only two of nine participants who watched the videos are dismissed from the target group. While it should be noted that progressing from level 1 to level 3 requires different information than progressing from level 7 to level 9, it does not directly affect the experiment's results as the participants are divided into equal groups based on their initial self-assessment (Q2), so that both groups have a similar level of knowledge. This calculated division into two equal groups ensures the comparability of both groups. Nonetheless, a self-assessment is a subjective opinion about a person's knowledge which may differ more or less from the person's actual knowledge. In addition, different persons may interpret the scale from 1 to 10 differently (even if a description of all levels was given), thus classifying themselves differently even though they might have a similar estimation of their knowledge. With regard to questions 1 and 5, the nature of the questions themselves might skew the results. As the questions are designed for a context that necessitates a quick identification of a person's knowledge level (i.e., in a trade fair context), they only allow a rough estimate. It is also questionable and subject to interpretation how much of the blockchain technology is understood so that the questioned person would answer \enquote{Yes} to the question \enquote{Do you understand blockchain technology?}. Restating these limitations is essential to remain attentive and rigorous, but as the questions aim to be as relevant as possible in the context for which they are designed, the questions and their results may be considered sufficiently reliable. Therefore, to conclude the discussion of the more quantitative questions 1/5 and 2/6, they provide strong evidence that the artifact does indeed provide a better learning resource for a novice audience than videos.

With regard to the following set of questions (3/7), it is evident that most participants describe blockchain technology by its most prevalent (and namesake) characteristic: the concatenation of blocks that altogether present the blockchain data structure. While this insight is not surprising, the difference between the two groups' responses before and after accessing the resources is significant. For participants of group 1, the concept of decentralization has gained much in importance. It even surpasses blockchain's namesake characteristic. Similarly, the responses of group 2 also focus on decentralization. However, the recorded increase for the concept of decentralization is within average, whereas a posteriori explanations most often add the concept of hash functions as well as the concept of digital value. A possible explanation for these results may be the different focus of the two resources. The artifact provides access to a guided tour and subsequently lets the user interact with more detailed concepts (such as hash functions). The videos, on the other hand, focus on presenting the role of decentralization and how the blockchain stores all data publicly accessible (accountability). In addition, it should be considered that the extent of the explanation may be subject to the participants' motivation and estimation of what they regard as important. Furthermore, the addition of new concepts to an explanation does not necessarily constitute \textit{new knowledge}. The information may have been known beforehand but not deemed important enough to mention in the a priori explanation. The conclusiveness of the responses is also weakened by the short time difference between learning about blockchain and replying to the questions. The progress might only exist because the information is still being processed and it is possible that some information, if it has not been fully understood, is forgotten after some time. Nevertheless, the addition of these concepts to the explanation proves that a change has taken place. Regardless of the different concepts and limitations, all a posteriori responses are more structured, more detailed and more technically correct than their counterparts. This shows that both explanations effectively serve the purpose of deepening the understanding of the blockchain technology. 

The fourth and eighth questions reveal similar insights. The majority of participants (12 of 18) concurs that blockchain technology is used mainly for cryptocurrencies and the digital transfer of money or other valuable assets. Since only those participants who list further use cases before accessing the resources have indicated a level of $\geq$5, this paper suggests that a novice audience solely relates blockchain technology use cases to cryptocurrencies or transactions of other types of value. This is supported by one participant listing \enquote{money laundering and drug trafficking} as possible use cases.\footnote{K2.6 in \ref{anhang:ResponsesQuestionnaire}} It is sensible then that a person with little knowledge of blockchain assumes the technology would not have any value for their profession since they solely consider financial use cases. Furthermore, the list of use cases mentioned by the participants has doubled after accessing the resources. Since these new use cases are not only more accessible and varied, but also more relevant to real-world applications, it appears that the explanations help the participants in gaining a better understanding in which situations blockchain might be useful.

Finally, the last two questions provide insights into what learning environment and instructional elements the participants like or dislike. Common to both groups and to both learning resources (tutorial and videos) is that a short and compact explanation of the topic is appreciated. Both groups also praise the use of visual elements. However, the participants of group 2 feel that the artifact contains too much text and would benefit from more graphics and visualizations. These remarks align well with Mayer's multimedia and contiguity principles and show that the artifact may be improved to follow these principles more closely. Further interesting points of data are a) one participant's response criticizing that the video was not interactive and they therefore quickly felt bored and less engaged\footnote{K1.7: \enquote{It was not interactive, I quickly lost the thread.} in \ref{anhang:ResponsesQuestionnaire}} and b) another participant has criticized that the explanation was too quick to understand it completely.\footnote{K1.9: \enquote{The explanation was just too quick.} in \ref{anhang:ResponsesQuestionnaire}} Point a) presents an unanticipated response as an implicit criterion for the video selection was that the videos are not longer than ten minutes so that they remain as engaging as possible. The question therefore arises that if watching educational videos for ten minutes results in disengagement and passivity, how much would the participant learn from a 30-minutes long video?\footnote{Numerous more detailed videos on YouTube span about 30 minutes.} Interestingly, the tutorial does not experience these criticisms as it actively engages the participants in the learning environment by requiring their action to proceed in the guided tour and letting them decide the speed of the presentation. In addition to this, the tutorial also provides a solution to the participants' statements criticizing the lack of more detailed information in the videos. By first providing a guided tour introducing the most relevant concepts and then allowing the participants to learn more about underlying technical concepts, the participants have access to more information if they are interested. This freedom - to choose different concepts according to the participants' interests - presents another advantage of the artifact when compared with traditional video learning resources.

Having critically discussed the individual results and compared the artifact to existing multimedia instructions regarding blockchain, the following list summarizes these findings while discussing the evaluand's properties: efficacy, effectiveness, efficiency, utility, purposefulness, and implementability.

\begin{itemize}
    \item \textbf{Efficacy}: The artifact's efficacy is determined by how well it produces its desired outcome and that the artifact, and not a confounding variable, produces this outcome.\footcite[Cf.][p.427]{Pries-HejeComprehensiveFrameworkEvaluation2012} This thesis argues that the artifact fulfills this characteristic as the evaluation has established that the participants have increased their knowledge of blockchain. As the participants have all replied to the same set of questions at the same time without interacting with each other, the possibility of external variables skewing the results is reduced to a minimum. The results also provide substantial evidence that the artifact (mostly because of its interactive elements) has improved the participants' understanding more than the videos.
    \item \textbf{Effectiveness}: An artifact is considered effective if it works in a real situation.\footcite[Cf.][p.428]{Pries-HejeComprehensiveFrameworkEvaluation2012} By evaluating the artifact in a naturalistic fashion setting the evaluation in a real environment with real people (introducing new students to the blockchain team) and yielding such positive results, it is evident that the artifact fulfills this property, too. The consideration of the personalization principle, as well as the overall cognitive load theory and the approach of a guided tour strongly support this argument. It seems therefore likely that the artifact's utilization in practice will have the same effects.
    \item \textbf{Efficiency}: The efficiency of the artifact is described by how well it works within resources constraints.\footcite[Cf.][p.427]{Pries-HejeComprehensiveFrameworkEvaluation2012} While the design of artifact as an online instruction makes it easy and cost-effective to set up, access, and use, going through the tutorial takes more time (\textasciitilde15 minutes, depending on the participant) than watching the videos. However, the artifact may be used by the company for events as it may easily be restyled to fit corporate branding restrictions,\footnote{All styling is done in a separate \ac{CSS} file} whereas the videos would either have to be remade entirely or bought from the creators. Since both of these options would demand many resources, they cannot be considered efficient, thus supporting the argument that the artifact is more efficient. If however, an explanation is needed for personal interest, videos or other online learning resources may be considered a more efficient alternative to the artifact because the user has basically free access to a considerable number of different approaches explaining the topic.
    \item \textbf{Utility}: Two variants of utility exist: The first measures the quality of the artifact for people in real use and the second measures the quality for an organization in real use.\footcites[Cf.][p.13]{HevnerDesignResearchInformation2010} Regarding the first interpretation, the thesis makes a strong point for the artifact's utility as the evaluation reports an increase of understanding (or at the very least in a structured approach to explaining the technology) surpassing that of the control group. The organization also benefits from the artifact as it may use it to introduce their employees to blockchain or to enable clients to think beyond the financial use cases.
    \item \textbf{Purposefulness}: The artifact is considered purposeful if it addresses an important problem.\footcites[Cf.][p.82]{HevnerDesignScienceResearch2004} While the evaluation as such has not focused on assessing the purposefulness of the artifact and hence provides no evidence supporting or refuting this property, the expert interviews indicate that the artifact does address an important problem. For example, the interviewee Kaltenbach states that the industry is missing an approach to explain the technology in layman's terms which is why he considers the work of this research important.\footcites[Cf.][]{DanielKaltenbach_Interview} Although this should not be considered rigorous evidence of the artifact's purposefulness, it supports the argument to a certain degree.
    \item \textbf{Implementability}: Since the designed artifact presents an instantiation that is -by design- easy to implement (\ac{HTML}, \ac{CSS}, and \ac{JS}), it indicates that the artifact meets this property. Another supporting factor is that the artifact is easily implemented into the organization. As it is a small, light-weight application which does not interfere with existing processes, it should be no problem to include the artifact in the existing organization. However, since the evaluation has no clear evidence to support this argument, it is not possible to give a definitive statement beyond this indication.
\end{itemize}

%However, although the evaluation was conducted in as a rigorous fashion as was possible due constraints in time and resources, there are some factors which might have influenced the outcome of the results. 
The evaluation's intention is to provide evidence regarding to which extent the artifact solves the problem of providing an adequate learning resource to persons with little to no knowledge about blockchain technology (see this thesis' research question: \textit{What and how should the artifact visualize information, so it explains the blockchain technology in a comprehensible way to a novice audience?}).
Based on the evaluation's insights, this paper argues that the designed artifact offers various qualities proving its rigor. But the insights have also made apparent that the evaluation is subject to certain limitations, and that the artifact, as well as the research process, could be further refined to solve the problem in a more efficient manner and to rigorously prove the artifact's implementability and purposefulness as well as further evaluation criteria. 

\section{Identification of possible optimizations} \label{sec:Optimizations}
The evaluation reveals that the artifact may be further refined to accommodate the participants' criticisms. While there are certainly measures to further improve the overall evaluation process, these issues are addressed in a following chapter and this section focuses on the artifact itself. As stated in \ref{sec:DevArtifact}, the artifact does not implement an advanced design or elaborate animations. Neither is it responsive or easily-accessible via mobile devices. These are certainly measures among many others that could enhance the artifact. However, these improvements do not constitute insights derived from the evaluation and are hence not further discussed. Contrary to this, the ideas listed below are based on the evaluation and may therefore be discussed more thoroughly.

\paragraph{More visualizations} The students participating in the demonstration and evaluation criticize that the tutorial contains too much written text and is lacking visual elements. This implies that the multimedia principle might not have been sufficiently applied. One solution would be to develop more elaborate visualizations which do not necessitate as much explanation as the previous artifact. Another solution, which might ideally be combined with the previous idea, is to include the modality principle, replacing the on-screen text with an audio description. While this thesis has disregarded the modality principle because including audio-files would result in higher cost of set-up and might not even be usable in a loud environment, it might be a sensible idea to provide the user with both options. However, it should be ensured that not both explanations (spoken and on-screen text) are viewed by the user at the same time as this would hinder the learning progress (redundancy principle). But if the user has the freedom to choose their preferred explanation, the artifact may become more relevant as it offers higher flexibility to accommodate individual preferences.

\paragraph{Better user experience} In addition to these optimizations, a next iteration of the artifact should also provide a better user experience. Participants mention that they have found it difficult to discover the interactive elements in the overview and that they were pointlessly searching the page for more discoverable elements. This could be addressed by more vividly cueing these elements.\footnote{While the interactive elements of the overview do give visual feedback (shadows, 3D-effects, and tooltips) upon being hovered on, their particular importance should be made clear even without users' initial actions.} Not only would this improve the user experience but it would also ensure that the user does not overlook a topic by mistake (hence making the artifact more consistent). Another participant comments that a better orientation throughout the tutorial would be appreciated, suggesting numbering the different text fields. A site map might present another idea which would allow a better orientation within the artifact, as it would also indicate the user's position in the tutorial (site map principle). A better user experience might overall enhance the artifact's elegance and style, should these properties be assessed in an ensuing design science research cycle. 

\paragraph{More interactive elements} Many participants praise the artifact's feature which lets the user create their own hash codes. It may also be because of this feature that the participants of group 2 included the role of hash functions in their explanation of blockchain (Q6). It seems therefore sensible to include more deeply engaging features\footnote{rather than more \enquote{click to read more} elements} which let users play with the presented information. A playful encounter with new information might enrich the learning process and further enhance meaningful learning. More interactive features might also make the artifact stand out more strongly from other multimedia blockchain learning resources. 

\paragraph{More practical examples} Although the artifact has listed numerous possible use cases of the blockchain technology, the participants were missing more explicit examples of the technology's application. Adding more examples might solve this problem, ideally one pithy example for each item on the list, or as an intermediate step: one example for each area of use cases (use cases regarding transactions of value, use cases regarding proof of existence, and use cases regarding automatic execution and machine to machine payments). These examples should be on a similar abstraction level as the existing example in the artifact (tracking of a car's mileage) so that users may easily understand them and apply these use cases to their field of work. In addition to this, the artifact could also include practical instructions that show how some of the concepts may be utilized in every-day life, such as the private and public key infrastructure. Letting users experience the effects of these technologies in reality might prove highly valuable to their assessment of the technologies' value. However, this approach should be subjected to a thorough examination as it might conflict with the coherence principle.

To conclude, the artifact's evaluation has revealed four possible areas of improvement: The addition of further visualizations, practical examples, and interactive elements, as well as the creation of a better user experience by including a site map and cueing important features more clearly. The identified improvements might enhance the artifact's utility, quality, and other properties. To assess the value of these improvements, they should be implemented and evaluated in an ensuing iteration of this research process.