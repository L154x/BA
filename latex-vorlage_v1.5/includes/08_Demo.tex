\chapter{Implementation and Evaluation}

\section{Exemplary demonstration: Onboarding new students} \label{sec:demo}
circa eine Seite

demonstration serves as a light-weight evaluation to prove the feasibility, that the artifact solves one of the problem instantiations

The demonstration is done for new students that are going to ework in the department of StudyLab which treats the blockchain technology. 

\section{Evaluation} \label{sec:Evaluation}

The exemplary demonstration lends itself excellently for a more thorough evaluation of the created artifact. As stated by \cite{HevnerDesignScienceResearch2004} and agreed upon by the design science research community, such evaluation plays a \enquote{crucial} role in design science research. This importance is further emphasized by \cite{MarchDesignnaturalscience1995} who regard evaluation as one of two activities in design science (see \ref{subsec:GuidelinesDesignScience}).\footcites[Cf.][p.258]{MarchDesignnaturalscience1995}[Cf. in addition][]{PfeffersDesignScienceResearch2007}{HevnerDesignScienceResearch2004}{Pries-HejeComprehensiveFrameworkEvaluation2012}{Pries-HejeStrategiesDesignScience} Once the (first iteration of the) build-activity is concluded, a rigorous evaluation of the artifact and its qualities must take place in order to provide evidence that the artifact achieves its purpose for which it was designed, i.e. that it solves the initially identified problem. \footcites[Cf.][p.425]{Pries-HejeComprehensiveFrameworkEvaluation2012} The evaluation should furthermore also assess how well the artifact solves the problem. This is done by first identifying the possible evaluation criteria candidates. These include - in addition to the in guideline 3 \textit{Design Evaluation} described properties of a design artifact (utility, quality, and efficacy) - other properties such as Checkland and Scholes five E's (efficiency, effectiveness, efficacy, ethicality, elegance) or an artifact's purposefulness and implementability.\footcites[Cf.][p.427]{Pries-HejeComprehensiveFrameworkEvaluation2012}[cf. in addition][]{ChecklandSoftSystemsMethodology1990}
Overall, the candidates of evaluation criteria should be chosen in such a way that they adequately identify the relevant strengths and short comings of the artifact so that its utility may be adequately assessed. At first, focusing on certain criteria and disregarding others may appear to inhibit a rigorous approach that is necessary to ensure the scientific nature of the artifact. But this selection of important criteria actually ensures something of similar importance, the relevance of the applied methods with regard to the overall relevance of the design science research.\footnote{For more information about the interplay of relevance and rigor in design science research, see \ref{topic:relevance cycle}}

% While it is important to rigorously conduct the relevant evaluation methods in order to ensure the scientific nature of the design science research, In other words, even though rigorous evaluation methods must be executed to ensure the design science process is of scientific nature, the evaluation should nonetheless focus on the important properties in order to remain relevant.

However, independent of the evaluation criteria, there exist a wide number of possible evaluation methods to choose from: Hevner et al. identified twelve different methods which may be put in the following categories: observational, analytical, experimental, testing, and descriptive methods.\footcites[Cf.][p.86]{HevnerDesignScienceResearch2004} Yet, these do not represent the entire list of relevant evaluation methods available to design science researchers. As \cite{PfeffersDesignScienceResearch2012} found out, the majority of researchers conduct technical experiments,  followed by illustrative scenarios and case studies. However, the method chosen depends largely on the type of artifact and its qualities to be evaluated. In the case of information systems research, more qualitative approaches such as case studies or subject-based experiments are pursued as these are deemed more relevant.\footcites[Cf.][p.4 et seq]{PfeffersDesignScienceResearch2012}

The following paragraphs introduce therefore the evaluation method for this artifact as well as the strategy followed to identify the most appropriate evaluation method before presenting and discussing the results of that evaluation. 
%validity, utility, quality, efficacy; Purposeful? and implementability?

\subsection{Method} \label{subsec:EvaluationMethod}

While Hevner et al. have identified a number of different evaluation methods, they have not offered any guidance on the choice of appropriate evaluation strategies. It is for this reason that Pries-Heje et al. have proposed a framework for evaluation in design science research which presents the scientific basis for this evaluation process.\footcites[Cf.][p.11 et seq]{Pries-HejeComprehensiveFrameworkEvaluation2012}


\paragraph{Why was this evaluation method chosen?} -> Ex-Post strategy, naturalistic, 
\footcite{PfeffersDesignScienceResearch2012} \cite{Pries-HejeComprehensiveFrameworkEvaluation2012}

In the ex post perspective, a chosen system or technology is evaluated after it is acquired or implemented. p.3, 
\cite{Pries-HejeStrategiesDesignScience}

\paragraph{Participants}
Gruppen von 8 Personen aufgeteilt in 2 Lager -> eine Gruppe sieht sich die zwei meistgeschauten YouTube-Videos an und die andere durchläuft das Tutorial und die Overview.
Bei der Auswahl werden die Fragen zur Bestimmung der Zielgruppe gestellt. Nur, wenn sie in die Zielgruppe fallen, kommen sie als Teilnehmer infrage.

\paragraph{Materials}
die Videos + kurze Beschreibung;
meine Webseite;
Fragebogen
\paragraph{Procedure}
Vor Beginn wird der Fragebogen durchlaufen, nicht alle Fragen, aber die für den Vergleich schon.
Die Testpersonen sehen sich alle gleichzeitig die Videos sowie das Tutorial an.
Sofort nach Ende der Durchführung werden die Teilnehmer gebeten an ihren Arbeitslaptops den Fragebogen in schriftlicher Form auszufüllen. 


\subsection{Results} \label{subsec:EvaluationResults}
- Scores and answers

\subsection{Discussion}

\section{Identification of possible optimizations} \label{sec:Optimizations}

How could the artifact be refined? As Design Science Research is an iterative process -> will lead to further research for the future.
