\chapter{Discussion of Design Science}
\textbf{circa 8 Seiten}


The Information Systems Research is by nature a design-oriented science.

One objective of Information Systems Research is to analyze, optimize and create information systems. These information systems are sociotechnical systems that comprise people, organizations and technology as well as their relationship to each other.\footcites[Cf.][p.98]{HevnerDesignScienceResearch2004}[cf.][p.11]{OsterleGestaltungsorientierteWirtschaftsinformatikPladoyer2010} The term people should be interpreted in this case as any person interacting with an information system. Organizations include not only the functions, structures and management but also business processes.\footcite[Cf.][p.98]{HevnerDesignScienceResearch2004} Finally, the term technology should be understood as anything useful and practical embodied in an implement or artifact, instead of thinking of it in form of a concept. According to March and Smith, technology should be understood as an expression of intelligence, including the tools, techniques and sources of power that humans have developed to reach their goals. \footcite[Cf.][p.252]{MarchDesignnaturalscience1995} 

The approach to research in Information Systems is multi-facetted. It builds upon observation, experimenting, systems development as well as theory building to contribute to research. These four approaches are also necessary to inspect the different aspects of a research question in Information Systems.\footcite[Cf.][p.86]{NunamakerSystemsdevelopmentInformation1991} This multi-methodological approach to research can be seen in the output of Information Systems research. Objective of ISR is to create constructs, models, methods, or instances,\footcite[Cf.][p.12]{OsterleGestaltungsorientierteWirtschaftsinformatikPladoyer2010} that 

As these artifacts are produced in a complex environment, where people and organization

\section{What is Design Science Research?} 
\begin{itemize}
    \item Put it in the context of a mixed method approach
    \item For what kind of research/problems is it used?
    \item What is its result?
    \item Discuss IT Artifact/IS Artifact
    \item Show that it has two sides (behaviour sciences and...?)
    \item Show the two kinds of knowledge (presciptive and descriptive)
    \item Demonstrate the ISR Framework
\end{itemize}


\section{What are its components/cycles/guidelines?}
\begin{itemize}
    \item Show the 7 step approach
    \item Show Pfeffer's approach
    \item Show the engineering and research cycle
\end{itemize}

\section{What does the design process of the dashboard look like?}
\begin{itemize}
    \item Argue why design science research is the appropriate method for this problem
    \item Classify the presented steps from above in the context of the dashboard design
\end{itemize}