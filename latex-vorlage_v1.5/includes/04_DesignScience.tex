\chapter{Discussion of design science} \label{chap:DesignScience}

Building on the theoretical understanding of blockchain and instructional multimedia design, the following chapter sets the foundation for the artifact design process. The design process must be led by a methodology grounded in theory. Since the artifact, a multimedia instruction, presents an information system (\acs{IS}) whose creation underlies a creative process, a pragmatic design science approach is deemed to be the best way to methodologically conduct the research. This chapter, therefore, presents design science as a research method in \acs{IS}. The first section establishes design science as a method in the \acs{IS} discipline. The ensuing pages describe the method in more detail including the design cycles and process steps of design science to examine the utilized design process for the blockchain explanation.

\section{Design science in the context of information systems research} \label{DesignScienceInISR}

\acf{IS} are by nature a design-oriented science.\footcites[Cf. in addition][]{OsterleGestaltungsorientierteWirtschaftsinformatikPladoyer2010} The object of these scientific studies is the design of \ac{IS} in business and societal contexts (also defined as sociotechnical systems comprising people, organizations, and technology).\footcites[Cf.][p.671]{OsterleMemorandumzurgestaltungsorientierten2010}[cf.][p.98]{HevnerDesignScienceResearch2004}[cf.][p.11]{OsterleGestaltungsorientierteWirtschaftsinformatikPladoyer2010}[cf.][p.252]{MarchDesignnaturalscience1995}
% As technology is an essential part of most information systems, it is important According to March and Smith, technology should be understood as an expression of intelligence, including the tools, techniques and sources of power that humans have developed to reach their goals.\footcite[Cf.][p.252]{MarchDesignnaturalscience1995} 
The approach to research in \ac{IS} is multi-faceted. It builds upon observation, experimenting, systems development, and theory building to contribute to research as well as to inspect the different aspects of a research question in \ac{IS}.\footcite[Cf.][p.86]{NunamakerSystemsdevelopmentInformation1991} The output of \ac{IS} research reflects this: The objective of \ac{IS} research is to create constructs, models, methods, or instances that were previously unknown to the community or that have not yet existed.\footcites[Cf.][p.12]{OsterleGestaltungsorientierteWirtschaftsinformatikPladoyer2010}[cf.][p.130]{ThomasBekannteundweniger2014} 
Irrespective of the form of the result, the ultimate goal of \ac{IS} research is to create normative artifacts which solve relevant problems.\footcite[Cf.][p.130]{ThomasBekannteundweniger2014}

%As these artifacts are produced in a complex environment, where people and organizations are important variables, a solution cannot be found in a deterministic manner. The success and validity of the conducted research therefore cannot be formally proven (or very rarely), instead it depends on the acceptance in the Information systems community.\footcite[Cf.][p.671]{OsterleMemorandumzurgestaltungsorientierten2010}

\paragraph{Design Science}
The idea of design as a science was first conceptualized by Simon in 1996\footcite[Cf.][]{Simonsciencesartificial1996} as a research paradigm to create innovative artifacts which solve real-world problems.\footcite[Cf.][p.9]{HevnerDesignResearchInformation2010}
Much of the relevant literature regarding design science state relevance and utility for the practice to be of high importance. They emphasize that the creation of useful artifacts is an essential element of the research.\footcites[Cf.][p.253, 254]{MarchDesignnaturalscience1995}[cf.][p.9, 11]{HevnerDesignResearchInformation2010}[cf.][p.77]{HevnerDesignScienceResearch2004}[cf.][p.1]{PapalambrosDesignScienceWhy2015}[cf.][p.330,342]{GregorPositioningpresentingdesign2013} Design science can be defined as a problem-solving paradigm\footcite[Cf.][p.77]{HevnerDesignScienceResearch2004} that designs, creates and evaluates innovative \ac{IT} artifacts\footcite[Cf.][p.90]{HevnerDesignScienceResearch2004} by focusing human creativity\footcite[Cf.][p.13]{HevnerDesignResearchInformation2010} on real-world problems to create value.\footcite[Cf.][p.1]{PapalambrosDesignScienceWhy2015} %As emphasized by \cite{HevnerDesignResearchInformation2010}, it is important to differentiate \textit{design as research} from \textit{researching design} as well as from \textit{routine design}. Design science as \textit{design as research} creates new knowledge and contributes to the knowledge base by designing innovative artifacts.Even though the artifacts may be strongly influenced by their specific problem domain and there may be more research needed to generalize the research results, they are contributions to the knowledge base. In contrast, \textit{researching design} sets the focus on the methods of designing, mainly situated in the disciplines of engineering, product design and architecture and \textit{routine design} misses the aspect of creating new knowledge as it is more a creative process that already has accumulated any necessary knowledge.\footcite[Cf.][p.15 et seqq]{HevnerDesignResearchInformation2010}
Design science research consists of several components, of which the \ac{IT} artifact has already been named. The following section discusses these components in more detail.

\section{Components of design science} \label{sec:ComponentsDesignScience}
Another brief definition is given by \cite{VaishnaviDesignScienceResearch} (p.6), who define design science as \enquote{Learning through building artifacts}. Similarly, \cite{MarchDesignnaturalscience1995} (p.254) characterize design science by their products: \enquote{artifacts and artificial phenomena}. These definitions reveal that artifacts play a crucial role in design science research and hence deserve a more thorough examination.

\paragraph{Artifact} \cite{Simonsciencesartificial1996} originally defined an artifact as an artificial, man-made thing which imitates natural things, and which is designed to attain certain objectives in a specific way.\footcites[Cf.][p.5]{Simonsciencesartificial1996} The term \ac{IT} artifact builds on top of this definition, but it is discussed critically in the relevant literature. For some writers it implies only technical artifacts\footcites[Cf.][p.186]{BenbasatEmpiricalresearchinformation1999}[cf.][p.50]{AlterconceptITartifact2015}[cf.][p.121]{OrlikowskiResearchcommentaryDesperately2001} while for others it also includes social artifacts or a combination of both.\footcites[Cf.][p.1, p.6]{LeeGoingbackbasics2015}[cf.][p.59]{AlterconceptITartifact2015} Herein, the term artifact (as well as \ac{IT} artifact) is defined in accordance with \cite{HevnerDesignScienceResearch2004} to describe something purposeful, innovative, and man-made supporting problem-solving and organizational capabilities by providing intellectual as well as computational tools.\footcites[Cf.][p.76 et seqq]{HevnerDesignScienceResearch2004}[cf.][p.340]{GregorPositioningpresentingdesign2013} The different possible artifacts, as results of design science research, are presented in table \ref{tab:Artifacts}. \footcites[Cf.][p.256 et seq]{MarchDesignnaturalscience1995}[cf.][p.50]{PuraoDesignResearchTechnology2002}[cf.][p.77]{HevnerDesignResearchInformation2010}

Artifacts may also be described by their type of research contribution. Artifacts of level 1 are instantiations as they present a situated implementation of an artifact and therefore contribute more specific, limited and less mature knowledge to the knowledge base. Level 2 describes artifacts such as constructs, methods, models, and design principles that contribute knowledge in the form of operational principles or architectures summarized as nascent design theory. The third level refers to well-developed design theory about embedded phenomena that constitute more abstract, complete and mature knowledge.\footcite[Cf.][p.340]{GregorPositioningpresentingdesign2013} \label{topic:levels} 

\paragraph{Contributing to the knowledge base} There are two different types of knowledge which comprise the knowledge base to which design science research contributes: Prescriptive knowledge (already available insights such as constructs, models, methods, instantiations, and design theory) and descriptive knowledge (making sense of the environment with the help of natural laws,
\begin{table}[H]
\setlength\extrarowheight{2pt} % for a bit of visual "breathing space"
  \centering
  \begin{tabularx}{\textwidth}{|l|X|l|}
    \hline
        \textbf{Output} & \textbf{Description} & \textbf{Example}  \\ \hline\hline
        Constructs & Conceptual description of problems in form of a specialized language and shared knowledge & Vocabulary, Symbols \\
        Models & Representations and sets of propositions or statements expressing relationships among constructs. & Abstractions, Syntax \\
        Frameworks & Real or conceptual guides to serve as support or guide & \\
        Architectures & High level structures of systems & \\ 
        Design Principles & Core principles and concepts to guide design & Design rules \\
        Methods & Sets of steps (based on constructs and models) used to perform tasks & Algorithms, guidelines \\
        Instantiations & Realization of an artifact in its environment, operationalization of constructs, models, and methods to demonstrate feasibility and effectiveness & Systems, products \\
        Design theories & A prescriptive set of statement on how to do something to achieve a particular goal. Theory includes other abstract artifacts (constructs, models, frameworks, architectures, design principles and methods) & \\ \hline
    \end{tabularx}
    \caption[The different outputs of design science.]{The different outputs of design science.\protect\footnotemark }
    \label{tab:Artifacts}
\end{table}
\footnotetext{With changes taken from \cite{MarchDesignnaturalscience1995} and \cite{VaishnaviDesignScienceResearch}.}

patterns, principles, and theories). Based on this distinction, Gregor and Hevner have identified a knowledge contribution framework, as presented in figure \ref{fig:DSRKnowledgeContribution}. An artifact may either be classified as routine design (no knowledge contribution), improvement (prescriptive knowledge contribution), invention (prescriptive and/or descriptive knowledge contribution), or as exaptation (prescriptive and/or descriptive knowledge contribution).\footcite[Cf.][p.344 et seqq]{GregorPositioningpresentingdesign2013} It is interesting to note that artifacts which are very specific to a problem or application domain nonetheless contribute to the knowledge base since they may embody design ideas and theories yet to be articulated.\footcite[Cf.][p.340]{GregorPositioningpresentingdesign2013}

\begin{figure}[H]
    \centering
    \includesvg[width=0.36\textwidth]{graphics/DSRKnowledge}
    \caption[The knowledge contribution framework in design science.]{The knowledge contribution framework in design science.\protect\footnotemark}
    \label{fig:DSRKnowledgeContribution}
\end{figure}
\footnotetext{With changes taken from \cite{GregorPositioningpresentingdesign2013}, p.345}


\paragraph{Information Systems Research Framework} \phantomsection \label{topic:design cycle}
As part of their research efforts, Hevner et al. have developed a detailed framework for \ac{IS} research based on elements of design science (build/evaluate cycle) as well as behavioral science (develop/justify cycle).\footcite[Cf.][p.80 et seqq]{HevnerDesignScienceResearch2004}
As can be seen in figure \ref{fig:ISRFramework}, the research is conducted in these two complementary phases which constitute the design cycle. The design cycle addresses three questions to continuously and iteratively refine the artifact: 1) What is the artifact? 2) Which design processes will be used to build the artifact? 3) What evaluations are performed and what design improvements are identified?\footcites[Cf.][p.19]{HevnerDesignResearchInformation2010}[cf.][p.90]{Hevnerthreecycleview2007}[cf.][p.89]{Hevnerthreecycleview2007}

\begin{figure}
    \centering
    \includesvg[width=0.8\textwidth]{graphics/ISRFramework_New.svg}
    \caption[The information systems research framework]{The information systems research framework.\protect\footnotemark}
    \label{fig:ISRFramework}
\end{figure}
\footnotetext{With changes taken from \cite{HevnerDesignScienceResearch2004}, p.80. and \cite{Hevnerthreecycleview2007}, p.88}


\phantomsection \label{topic:relevance cycle}
Additionally, the framework consists of two more cycles: The relevance cycle, which continuously tracks the business needs from and the application in the environment (problem space) to ensure relevancy (usefulness of scientific research for real-world use).\footcites[Cf.][p.70]{Simonsciencesartificial1996}[cf.][p.79]{HevnerDesignScienceResearch2004}[cf.][p.129]{ThomasBekannteundweniger2014} For research to be relevant, it should be interesting (address challenges that are of interest to \ac{IS} professionals), applicable (utilizable by practitioners), current (focus on current technologies and business needs), as well as accessible (written in a way to be understood by \ac{IS} professionals).\footcite[Cf.][p.5]{BenbasatEmpiricalresearchinformation1999} The relevance cycle, therefore, acts as a bridge between the environment and the design science research.\footcite[Cf.][p.89]{Hevnerthreecycleview2007} 

% On one hand, the business needs which the artifact is trying to resolve are pulled from its environment, which defines the problem space.\footcites[Cf.][p.??]{Simonsciencesartificial1996}[cf.][p.79]{HevnerDesignScienceResearch2004} The problem space includes people, organizations and technology and the perceived goals, tasks, problems and opportunities, together these define the business needs or problem. \footcite[Cf.][p.79]{HevnerDesignScienceResearch2004} The relevance cycle tracks the business needs and the artifact's application in the appropriate environment by continuously assessing the design requirements, how the artifact is introduced in the environment, how it is field tested and whether the research question has been satisfactorily addressed. \footcites[Cf.][p.89]{Hevnerthreecycleview2007}[cf.][p.19]{HevnerDesignResearchInformation2010} This is done to ensure relevance of the research. Relevance describes the usefulness of scientific research for real world use, which is a typical objective of applied sciences, including ISR.\footcite[Cf.][p.129]{ThomasBekannteundweniger2014} For research to be relevant, it should be interesting (address challenges that are of interest to IS professionals), applicable (utilizable by practitioners), current (focus on current technologies and business needs), as well as accessible (written in a way to be understood by IS Professionals).\footcite[Cf.][p.5]{BenbasatEmpiricalresearchinformation1999} The relevance cycle therefore works as a bridge between the environment and the design science research and ensures the relevance of the research to the environment.\footcite[Cf.][p.89]{Hevnerthreecycleview2007} 
The third cycle is called rigor cycle, which acts as a bridge between the knowledge base and the conducted research by enforcing a satisfactory level of rigor during research. Rigor describes the way in which research is conducted, strictly and closely following formal methodologies and foundations, ensuring compliance with scientific requirements and standards and subsequently ensuring the acceptance by peers in the scientific community.\footcites[Cf.][p.130]{ThomasBekannteundweniger2014} Recent discussions have inspected the relation between rigor and relevance and state that too much rigor may lessen relevance, and that to conduct good design science research, both cycles must be considered.\footcites[Cf.][p.5]{BenbasatEmpiricalresearchinformation1999}[cf.][p.88]{HevnerDesignScienceResearch2004}[cf.][p.130]{ThomasBekannteundweniger2014}[cf.][p.91]{Hevnerthreecycleview2007}
% On the other hand, IS Research draws from the existing knowledge base which is composed of methodologies and foundations. Foundational theories, frameworks, models and instantiations are applied in the develop/build phase, whereas existing methodologies help evaluate the artifact in the second phase. The rigor cycle sets the ground for the research in the sense that it assesses which theories support the artifact design and design process and what new knowledge is added to the knowledge base in what form.\footcite[Cf.][p.19]{HevnerDesignResearchInformation2010} It provides past knowledge to ensure the problem space is not yet fully matured and the research is in fact innovative.\footcite[Cf.][p.90]{Hevnerthreecycleview2007} The rigor cycle aims to ensure a satisfactory level of rigor during research. Rigor describes the way in which research is conducted, following formal methodologies and foundations in a strict and exact manner which ensures compliance with scientific requirements and standards.\footcite[Cf.][p.88]{HevnerDesignScienceResearch2004} Rigor is important for researchers so that their work is accepted by their peers in the scientific community. Unfortunately, scientific research has typically put emphasis on rigor over relevance, as identified by \cite{BenbasatEmpiricalresearchinformation1999}, which actually lessens relevance.\footcites[Cf.][p.88]{HevnerDesignResearchInformation2010}[cf.][p.5]{BenbasatEmpiricalresearchinformation1999} The authors argue for a stronger focus on relevance on top of rigor and set recommendations how to achieve this shift of mindset and to more accurately address the dimensions of relevance (interesting, current, applicable, accessible).\footcite[Cf.][pp.5,8-14]{BenbasatEmpiricalresearchinformation1999} However, focusing on relevance should in no case imply that research may be carried out with less rigor.\footcite[Cf.][p.5]{BenbasatEmpiricalresearchinformation1999} Instead, both the relevance and the rigor cycles must be considered to ensure an acceptable adhesion to scientific standards as well as relevancy to the real world IS practitioners.\footcite[Cf.][p.130]{ThomasBekannteundweniger2014} The synergy between these two cycles is essential for good design science research.\footcite[Cf.][p.91]{Hevnerthreecycleview2007}

\section{The design science process} \label{sec:DesignScienceProcess}
While the previous section has outlined important components of design science research, it is important to identify the steps and guidelines which the research process should follow. This is the focus of the following sections.

\subsection{Guidelines and activities for conducting design science research} \label{subsec:GuidelinesDesignScience}
As mentioned before, conducting design science research is inherently a creative process which should be guided by scientific theories nonetheless. For this purpose, Gregor and Hevner developed seven guidelines, which are \enquote{largely accepted as integral to top quality design science research}\footcite[p.19]{HevnerDesignResearchInformation2010} and which are incorporated in numerous publications.\footcites[Cf.][p.20 et seqq]{PfeffersDesignScienceResearch2007}

\begin{enumerate}
    \item \textbf{Design as an artifact}: The result of design science research is a purposeful \ac{IT} artifact.
    \item \textbf{Problem relevance}: The objective of design science research is to solve important and relevant problems which occur by the interaction of people, organizations and \ac{IT}.
    \item \textbf{Design evaluation}: The resulting artifact must be rigorously evaluated via well-executed methods. A thorough evaluation demonstrates the quality, utility, efficacy, and style of the artifact.
    \item \textbf{Research contributions}: The contributions of design science research to the knowledge base must be clear and should therefore belong to one single level of contribution type (see section \ref{topic:levels}).
    \item \textbf{Research rigor}: The research needs to be grounded in existing scientific knowledge and methods when building and evaluating the artifact.
    \item \textbf{Design as a search process}: The search process comprises the abstraction and representation of appropriate means to reach desired ends by respecting the laws in the problem space.\footnote{The means, ends, and laws may be represented mathematically to allow a formal way of computing the solution space via standard operations research techniques. However, the problem and solution space is often so complex that automatically computing the solution is unfeasible.}
    \item \textbf{Communication of research}: The presentation of the research should be aimed towards technology-oriented as well as managerial audiences focusing on the problem relevance and innovative nature of the artifact.\footcites[Cf.][p.iv]{ZmudEditorComments1997}[cf.][p.82 et seqq]{HevnerDesignScienceResearch2004}[cf.][p.viii]{WeberEditorCommentsStill2003}
\end{enumerate}

Taking the guidelines into account, Pfeffers et al. have identified six activities that every well-formed design science research approach should take into consideration. Starting with the \textbf{problem identification and motivation} (justifying the value of the solution), the second activity is the \textbf{definition of the solution's objective} (inferring the solution objectives from the problem statement), followed by \textbf{design and development} of the artifact (determining and implementing the functionality and architecture of the artifact). The fourth activity, the \textbf{demonstration}, aims to prove the utility of the artifact leading to the artifact's \textbf{evaluation} (comparing the objectives of the solution with the achieved results during the demonstration, assessing quality, effectiveness, utility, and style). At the end of this activity, the researcher may either go one step back and refine the artifact (see \ref{topic:design cycle}) or proceed to the final activity: \textbf{Communication} of the research's results.\footcite[Cf.][p.12 et seqq]{PfeffersDesignScienceResearch2007}

% \begin{enumerate}
%     \item \textbf{Problem identification and motivation} The specific research problem is defined and the value of the solution is justified. The problem should be described in great detail so that the artifact can respond to all aspects of the problem space.
%     \item \textbf{Define the objective for a solution} The objectives of the solution are inferred from the problem and the researcher's knowledge of what is possible and economically viable.
%     \item \textbf{Design and development} This activity comprises the creation of the artifact but also the determination of the artifact's desired functionality and architecture. 
%     \item \textbf{Demonstration} The artifact is demonstrated to prove its utility to solve one or more instances of the problem.
%     \item \textbf{Evaluation} The artifact is rigorously evaluated in order to determine how well it solves the problem. The evaluation compares the objectives of the solution with the achieved results of the artifact and may furthermore include other quantitative or qualitative measures to determine the quality, effectiveness, utility and style of the artifact. At the end of this activity, the researcher may go one step back to refine the artifact as outlined in the design cycle (see 
%     \item \textbf{Communication} The conducted research is presented to the scientific community. The structure of the communication may be analogous to the followed research process (the activities above).
% \end{enumerate}

\subsection{Presentation of the artifact design process}
The research runs through each of the activities outlined above, starting with the problem identification and motivation in the introduction and a more detailed presentation in chapter \ref{chap:ReqEng}. Starting point for a more detailed problem definition are the interviews that are conducted with experts from various fields of expertise. The second activity, which also builds on the interviews and existing solutions on the internet, is executed by inferring the solution from the elicited requirements and the stated problem. During the third activity, the artifact is thereafter conceptualized, a design is created and the artifact is developed. This activity comprises designing a static/non-working draft that shows which elements the artifact should contain and how the interaction flow should work. The final development of the artifact then transforms the not-yet-working design into a fully functional application. After the development activity is concluded, the fourth activity takes place. The artifact is demonstrated during a workshop with students who are questioned afterward in order to evaluate how well the artifact fulfills its purpose. The presented guidelines (from above) are incorporated into these activities. Specifically, the first (design as an artifact) and fourth (research contributions) guidelines are represented by the visualization artifact, as it serves a clear purpose and can be categorized as an artifact of level 1. The second guideline (problem relevance) is represented by the further investigation of the problem to validate its relevance. Furthermore, the third guideline (design evaluation) corresponds directly to the evaluation activity, taking place in chapter \ref{chap:Evaluation}. Referring back to the fourth guideline, the research's knowledge contribution should be clearly defined and the research may be categorized by the form of the artifact (instantiation, model, method, construct, ...) as well as by the utilized research activities.\footcite[Cf.][p.255]{MarchDesignnaturalscience1995} This paper conducts the design science research related activities Build and Evaluate, as indicated in figure \ref{fig:researchFR}. The paper embraces the remaining three guidelines (research rigor, design as a search process, and communication of research) throughout the research, documentation and publication process. How well these guidelines are respected correlates directly with the quality of the research.\footcite[Cf.][p.19]{HevnerDesignResearchInformation2010} A complete discussion of the application of these guidelines takes place in \ref{sec:Reflection}.

\begin{figure}
    \centering
    \includesvg[width=0.8\textwidth]{graphics/researchAct}
    \caption[The research framework by March and Smith.]{The research framework by March and Smith. This research is conducts the build and evaluate activities\protect\footnotemark}
    \label{fig:researchFR}
\end{figure}
\footnotetext{With changes taken from \cite{MarchDesignnaturalscience1995}, p.255}
