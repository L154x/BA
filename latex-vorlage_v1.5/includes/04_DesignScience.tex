\chapter{Discussion of Design Science}
\textbf{circa 8 Seiten}

Building on the theoretic understanding of blockchain and dashboard design, the following chapter sets the foundation of the dashboard design process. Since the dashboard is to be understood as an information system whose creation underlies a creative process, a design science approach is deemed to be the best way to methodologically/on a scientific basis conduct the research. This chapter therefore presents design science research as a research method in Information Systems. The first paragraph establishes design science as a method in the Information Systems discipline. The ensuing pages will describe the method in more detail; the chapter will describe the design cycles and process steps of design science in order to finally examine the utilized dashboard design process.


\section{Design Science in the context of Information Systems Research}

Information Systems is by nature a design-oriented science. The object of scientific studies is the design information systems in business and society contexts.\footcite[Cf.][p.671]{OsterleMemorandumzurgestaltungsorientierten2010} These information systems are sociotechnical systems that comprise people (any person interacting with the information system), organizations (functions, management and business processes) and technology (anything useful and practical embodied in an implement or artifact. The term technology should be understood as an expression of intelligence, including the tools, techniques and sources of power that humans have developed to reach their goals) as well as their relationship to each other.\footcites[Cf.][p.98]{HevnerDesignScienceResearch2004}[cf.][p.11]{OsterleGestaltungsorientierteWirtschaftsinformatikPladoyer2010}[cf.][p.252]{MarchDesignnaturalscience1995}
% As technology is an essential part of most information systems, it is important According to March and Smith, technology should be understood as an expression of intelligence, including the tools, techniques and sources of power that humans have developed to reach their goals.\footcite[Cf.][p.252]{MarchDesignnaturalscience1995} 

The approach to research in Information Systems is multi-faceted. It builds upon observation, experimenting, systems development, and theory building to contribute to research. These four approaches are also necessary to inspect the different aspects of a research question in Information Systems.\footcite[Cf.][p.86]{NunamakerSystemsdevelopmentInformation1991} This multi-methodological approach to research can be seen in the output of Information Systems research. Objective of ISR is to create constructs, models, methods, or instances, that were unknown to the community before or that have not yet existed.\footcites[Cf.][p.12]{OsterleGestaltungsorientierteWirtschaftsinformatikPladoyer2010}[cf.][p.130]{ThomasBekannteundweniger2014} 
Irrespective of the form of the result, the ultimate goal of Information Systems Research is to create normative artifacts which solve problems.\footcite[Cf.][p.130]{ThomasBekannteundweniger2014}

As these artifacts are produced in a complex environment, where people and organizations are important variables, a solution cannot be found in a deterministic manner. The success and validity of the conducted research therefore cannot be formally proven (or very rarely), instead it depends on the acceptance in the Information systems community.\footcite[Cf.][p.671]{OsterleMemorandumzurgestaltungsorientierten2010}

\paragraph{Design Science}
The idea of design as a science was first conceptualized by Simon in 1996\footcite[Cf.][]{Simonsciencesartificial1996} as a research paradigm to create innovative artifacts which solve real-world problems.\footcite[Cf.][p.9]{HevnerDesignResearchInformation2010}
Much of the relevant literature regarding design science state relevance and utility for the practice to be of high importance and emphasize the creation of artifacts as a key element of the research.\footcites[Cf.][p.253, 254]{MarchDesignnaturalscience1995}[cf.][p.9, 11]{HevnerDesignResearchInformation2010}[cf.][p.77]{HevnerDesignScienceResearch2004}[cf.][p.1]{PapalambrosDesignScienceWhy2015}[cf.][p.330,342]{GregorPositioningpresentingdesign2013} Design science can therefore be defined as a problem-solving paradigm\footcite[Cf.][p.77]{HevnerDesignScienceResearch2004} that designs, creates and evaluates innovative IT artifacts\footcite[Cf.][p.90]{HevnerDesignScienceResearch2004} by focusing human creativity\footcite[Cf.][p.13]{HevnerDesignResearchInformation2010} on real-world problems to create value.\footcite[Cf.][p.1]{PapalambrosDesignScienceWhy2015} As emphasized by \cite{HevnerDesignResearchInformation2010}, it is important to differentiate \textit{design as research} from \textit{researching design} as well as from \textit{routine design}. Design science as \textit{design as research} creates new knowledge and contributes to the knowledge base by designing innovative artifacts. Even though the artifacts may be strongly influenced by their specific problem domain and there may be more research needed to generalize the research results, they are contributions to the knowledge base.\footcite[Cf.][p.15]{HevnerDesignResearchInformation2010} In contrast, \textit{researching design} sets the focus on the methods of designing, mainly situated in the disciplines of engineering, product design and architecture \footcite[Cf.][p.15]{HevnerDesignResearchInformation2010} and \textit{routine design} misses the aspect of creating new knowledge as it is more a creative process that already has accumulated any necessary knowledge.\footcite[Cf.][p.16]{HevnerDesignResearchInformation2010}

A brief, although less comprehensive definition is given by \cite{VaishnaviDesignScienceResearch} (p.6), who define Design Science as \enquote{Learning through building artifacts}. Similarly, \cite{MarchDesignnaturalscience1995} (p.254) characterize Design Science by their products: \enquote{artifacts and artificial phenomena}. Interestingly, even though there is a general agreement on the overall definition of design science and the production of artifacts, the definition of (IT) artifact is source of many discussions. For some writers the IT artifact implies only technical artifacts, while for others it also includes social artifacts.
The original definition by \cite{Simonsciencesartificial1996} (p.5) names four characteristics of an artifact.
\begin{enumerate}
    \item Artifacts, or artificial things, are created by human beings.
    \item Artifacts may reflect or imitate existing phenomena and natural things.
    \item Artifacts can be described by their terms of functions, objectives and adaptation.
    \item Artifacts are considered in terms of imperatives as well as descriptives. This means that the artifact is designed to attain certain objectives as well as to function in a specific way.  
\end{enumerate}

From this definition, the term IT artifact was born. However, there is no clear definition of what exactly an IT artifact is. While some writers argue that an IT artifact is the application of IT within a certain context to enable or support some tasks\footcites[Cf.][p.186]{BenbasatEmpiricalresearchinformation1999}[cf.][p.50]{AlterconceptITartifact2015}, others understand the term as a collection of things, such as materials and cultural characteristics, in form of hardware and/or software that can be socially recognized.\footcite[Cf.][p.121]{OrlikowskiResearchCommentaryDesperately2001} This has led to a hasty use of IT artifact describing a variety of concepts in Information Systems.\footcite[Cf.][p.49]{AlterconceptITartifact2015} Calls are made to retire the term IT artifact and to replace it with more precise terms\footcite[Cf.][p.59]{AlterconceptITartifact2015} or to create an IS artifact that consists of a technology artifact, an information artifact as well as a social artifact(in order to set the focus away from IT and instead back to IS).\footcite[Cf.][pp.1,6]{LeeGoingbackbasics2015} This paper does not participate in the above discussion and uses the term IT artifact in accordance with \cite{HevnerDesignScienceResearch2004} to describe something purposeful, innovative, and man-made supporting problem-solving and organizational capabilities by providing intellectual as well as computational tools.\footcites[Cf.][pp.76,82]{HevnerDesignScienceResearch2004}[cf.][p.340]{GregorPositioningpresentingdesign2013}
The IT artifacts that are produced while conducting design science research can be categorized as constructs, models, methods, instantiations as well as frameworks, architectures, design principles and design theories.\footcites[Cf.][pp.256-258]{MarchDesignnaturalscience1995}[cf.][p.343]{GregorPositioningpresentingdesign2013}[cf.][p.50]{PuraoDesignResearchTechnology2002} [cf.][p.77]{HevnerDesignResearchInformation2010} They are described in more detail in table \ref{tab:Artifacts}. 

\begin{table}
\setlength\extrarowheight{2pt} % for a bit of visual "breathing space"
  \centering
  \begin{tabularx}{\textwidth}{|l|X|l|}
    \hline
        \textbf{Output} & \textbf{Description} & \textbf{Example}  \\ \hline\hline
        Constructs & Conceptual description of problems in form of a specialized language and shared knowledge & Vocabulary, Symbols \\
        Models & Representations and sets of propositions or statements expressing relationships among constructs. & Abstractions, Syntax \\
        Frameworks & Real or conceptual guides to serve as support or guide & \\
        Architectures & High level structures of systems & \\ 
        Design Principles & Core principles and concepts to guide design & Design rules \\
        Methods & Sets of steps (based on constructs and models) used to perform tasks & Algorithms, guidelines \\
        Instantiations & Realization of an artifact in its environment, operationalization of constructs, models, and methods to demonstrate feasibility and effectiveness & Systems, products \\
        Design theories & A prescriptive set of statement on how to do something to achieve a certain goal. Theory includes other abstract artifacts (constructs, models, frameworks, architectures, design principles and methods) & \\ \hline
    \end{tabularx}
    \caption[The different outputs of Design Science.]{The different outputs of Design Science.\footnotemark }
    \label{tab:Artifacts}
\end{table}
\footnotetext{adapted from \cite{MarchDesignnaturalscience1995} and \cite{VaishnaviDesignScienceResearch}.}

Artifacts also may be described by their type of research contribution. Artifacts of level 1 are instantiations as they present a situated implementation of an artifact and therefore contribute more specific, limited and less mature knowledge to the knowledge base. Level 2 describes artifacts such as constructs, methods, models, and design principles, that contribute knowledge in form of operational principles or architectures, summarized as nascent design theory. The third level refers to well-developed design theory about embedded phenomena that constitute more abstract, complete and mature knowledge. Artifacts of this level are design theories.\footcite[Cf.][p.340]{GregorPositioningpresentingdesign2013}

\paragraph{Contributing to the knowledge base} Gregor and Hevner have drawn a distinction between prescriptive and descriptive knowledge in the existing knowledge base. Prescriptive knowledge refers to insights that are already available. Constructs, models, methods, instantiations and design theory all present prescriptive knowledge. Descriptive knowledge, in contrast, describes certain phenomena (natural, artificial or human) and includes natural laws as well as patterns, principles, and theories to make sense of the overall environment.\footcite[Cf.][p.344]{GregorPositioningpresentingdesign2013} On the basis of this distinction, Gregor and Hevner have also identified a knowledge contribution framework in design science. Depending on the solution maturity and the application domain maturity, a produced artifact can be classified as a routine design, an improvement, an invention or an exaptation. Routine design happens when both solution and application domain show high levels of maturity. There is no major knowledge contribution as known solutions are applied to already known problems.\footcite[Cf.][p.348]{GregorPositioningpresentingdesign2013} Even though it is listed in the matrix, see figure X???, it should not be understood as research but rather as professional design. Improvements are characterized by a low solution maturity and a high application domain maturity. The goal of DSR in this quadrant is to create better solutions for known problems and thus to contribute to the prescriptive knowledge base in the form of artifacts (no matter what level).\footcite[Cf.][p.346]{GregorPositioningpresentingdesign2013} Inventions happen when the solution and the application domain are only poorly understood and new solutions for new problems are discovered. Research contributions in this field are exceptionally new artifacts (but it can also be the articulation of an unknown problem itself). Knowledge that is contributed can be prescriptive and/or descriptive.\footcite[Cf.][pp.345,346]{GregorPositioningpresentingdesign2013} The fourth quadrant, exaptation, is characterized by a high degree of solution maturity and a low application domain maturity. This means that existing design knowledge is extended to new problems. Research in this quadrant creates prescriptive knowledge in the form of artifacts but may also create descriptive knowledge if the used artifacts are understood in greater detail.\footcite[Cf.][p.347]{GregorPositioningpresentingdesign2013}

Improvement, Invention and Exaptation all are valid research opportunities that contribute to the knowledge base. One should however refrain from routine design in DSR as there will be no major knowledge contribution.


To conclude, you can say that design science research produces artifacts that contribute to the knowledge base even if they are very specific to a certain problem or application domain since they may embody design ideas and theories that are yet to be articulated.\footcite[Cf.][p.340]{GregorPositioningpresentingdesign2013}

\section{What is Design Science Research?} 
\begin{itemize}
    \item Put it in the context of a mixed method approach X
    \item For what kind of research/problems is it used? X
    \item What is its result? X
    \item Discuss IT Artifact/IS Artifact X
    \item Show that it has two sides (behaviour sciences and...?)
    \item Show the two kinds of knowledge (presciptive and descriptive) X
    \item Demonstrate the ISR Framework
\end{itemize}


\section{What are its components/cycles/guidelines?}
\begin{itemize}
    \item Show the 7 step approach
    \item Show Pfeffer's approach
    \item Show the engineering and research cycle
\end{itemize}

\section{What does the design process of the dashboard look like?}
\begin{itemize}
    \item Argue why design science research is the appropriate method for this problem
    \item Classify the presented steps from above in the context of the dashboard design
\end{itemize}