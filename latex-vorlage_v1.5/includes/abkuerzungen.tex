\chapter*{Abkürzungsverzeichnis}
\addcontentsline{toc}{chapter}{Abkürzungsverzeichnis}

Ein Abkürzungsverzeichnis ist optional. Das Paket \verb|acronym| kann weit mehr, als hier gezeigt.\footnote{siehe \url{http://ctan.org/pkg/acronym}}
Beachten Sie allerdings, dass Sie die Einträge selbst in sortierter Reihenfolge angeben müssen.

\begin{acronym}[DHBW] 
% Argument definiert die Breite der ersten Spalte anhand des längsten vorkommenden Eintrags
\acro{CRM}{Customer Relationship Management}
\acro{DHBW}{Duale Hochschule Baden-Württemberg}
\acro{IEEE}{Institute of Electrical and Electronics Engineers}
\acro{ITIL}{IT Infrastructure Library}
\acro{RoI}{Return-On-Invest}
\acro{UCS}{Universal Character Set}
\acro{UTF-8}{8-Bit UCS Transformation Format} 
\end{acronym}

\vspace{2em}
{\footnotesize
\textbf{Ergänzende Bemerkung:}
Eine im Text verwendete Abkürzung sollte bei ihrer ersten Verwendung erklärt werden. Falls Sie sich nicht selbst darum kümmern möchten, kann das das Paket \verb|acronym| übernehmen und auch automatisch Links zum Abkürzungsverzeichnis hinzufügen. Dazu ist an allen Stellen, an denen die Abkürzung vorkommt, \verb|\ac{ITIL}| zu schreiben. 

Das Ergebnis sieht wie folgt aus: 
\begin{itemize}
\item erstmalige Verwendung von \verb|\ac{ITIL}| ergibt: \ac{ITIL},
\item weitere Verwendung von \verb|\ac{ITIL}| ergibt: \ac{ITIL}
\end{itemize}
Wo benötigt, kann man mit dem Befehl \verb|\acl{ITIL}| wieder die Langfassung ausgeben lassen: \acl{ITIL}.

Falls man die Abkürzungen durchgängig so handhabt, kann man durch Paket-Optionen (in \verb|_dhbw_praeambel.tex|)
erreichen, dass im Abkürzungsverzeichnis nur die tatsächlich verwendeten Quellen aufgeführt werden (Option: \verb|printonlyused|) und zu jedem Eintrag die Seite der ersten Verwendung angegeben wird (Option: \verb|withpage|).
}