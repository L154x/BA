\chapter{Solution definition} \label{chapter:Solution}
\textbf{roughly 3 pages}

\textit{Show what the artifact is supposed to do}
The objectives of the solution have to be inferred rationally from the problem statement. Therefore the problem should be atomized conceptually, so that the solution can capture its complexity.

\begin{itemize}
    \item the requirements will be extracted from expert interviews
    \begin{itemize}
        \item Why an interview?
        \item What kind of interview? explorative expert interview
        \item What experts? Frank Bloch, Bernd Kammholz, Torsten Milsmann, Daniel Kaltenbach, Thomas Maier -> all experts in their respective field
        \item What questions should I ask them? What do you not understand about blockchain? What would you wish to understand the most? Where do you feel lies the difficulty to blockchain and its transactions? What do you imagine under such a term as a dashboard visualizing blockchain transactions? What would be the best way to learn about it? Do you want it to be portable?
    \end{itemize}
    \item the interviews will be analyzed using the Qualitative Content Analysis of Mayring
    \begin{itemize}
        \item Why this method and not a quantitative method or different qualitative? -> Steigleider citation about how it is both the broadest and most exact technique. 
        \item How does the analysis work? Which process steps are there?
        \item Choose a summarizing content analysis and explain the steps for this
    \end{itemize}
\end{itemize}

\section{Reiteration of problem}

\section{Requirements analysis}


\subsection{Research method and objective}

\subsubsection{Expert interviews}

\subsubsection{Qualitative Content Analysis}

\subsection{Data collection and analysis}

\section{Objectives}

\subsection{Dashboard content}
\textit{What exactly should the dashboard visualize?}
State the requirements that come from the requirements analysis

\subsection{Technology for dashboard development}
\begin{itemize}
    \item What should the technology be able to do? What are the requirements? (Should come from the interviews)
    \item What possible technologies are there to build a dashboard? 
    \item Compare these techs with respect to the formulated requirements and choose a solution/combination(?)
\end{itemize}

