%%% LaTeX-Vorlage Version 1.4 %%%
\documentclass[a4paper,fontsize=11pt,oneside,parskip=half,headings=normal]{scrreprt} 
% \usepackage{showframe} % nur für Kontrolle der Ränder 

%%% Präambel einbinden %%%
\input{includes/_dhbw_praeambel.tex}

%%% Name Ihrer Literatur-Datenbank (ggf. anpassen) %%%
\bibliography{includes/literatur-datenbank.bib}

\begin{document}
%%% Deckblatt einbinden %%% 
% Anpassungen nötig (Name, Titel etc.)
% HIER EDITIEREN: 
% Typ der Arbeit (für Deckblatt und ehrenwörtliche Erklärung)
% - bitte Zutreffendes auswählen
%\newcommand{\typMeinerArbeit}{1. Projektarbeit} 
%\newcommand{\typMeinerArbeit}{2. Projektarbeit} 
%\newcommand{\typMeinerArbeit}{Seminararbeit} 
\newcommand{\typMeinerArbeit}{Bachelorarbeit} 

% Thema der Arbeit (für ehrenwörtliche Erklärung)
% HIER EDITIEREN: 
\newcommand{\themaMeinerArbeit}{Mein Titel}

\thispagestyle{empty}

\begin{spacing}{1}
\begin{center}	
~\vspace{0mm}

% HIER EDITIEREN: Titel der Arbeit
{\sffamily
\LARGE  
% \Large  % bei sehr langen Titeln ggf. etwas kleinere Schriftart wählen
\textbf{Titel der Arbeit}

\bigskip
\textbf{ggf. etwas länger}
}


\vspace{15mm}

% HIER EDITIEREN
{\Large \typMeinerArbeit}

\vspace{1cm}

% HIER ggf. EDITIEREN
vorgelegt am \today 

\vspace{15mm}

Fakultät Wirtschaft
\medskip

Studiengang Wirtschaftsinformatik
\medskip

% HIER EDITIEREN: Kurs eintragen
Kurs ... 

\vspace{10mm}

von

\vspace{10mm}

% HIER EDITIEREN: Vorname und Name eintragen 
{\large\textsc{Vorname Nachname}}

\vspace{10mm}
\end{center}

\vfill

% HIER EDITIEREN
\begin{tabular}{ll}
Betreuer in der Ausbildungsstätte: & DHBW Stuttgart: \\
\hspace{0.4\linewidth} & \\
Name des Unternehmens & Titel, Vorname und Nachname \\
Titel, Vorname und Nachname des Betreuers 
& des wissenschaftlichen Betreuers/Prüfers \\
Funktion des Betreuers \\
\\
Unterschrift des Betreuers \\
\end{tabular}


\vspace{1cm}
%(etwas Platz für die Unterschrift des Betreuers aus der Ausbildungsstätte)
\end{spacing}

% falls ein Sperrvermerk erforderlich ist: 
%\begin{center}
%\small
%\textbf{Sperrvermerk}:
%Der Inhalt dieser Arbeit darf weder als Ganzes noch in Auszügen \\
%Personen außerhalb des Prüfungsprozesses und des Evaluationsverfahrens zugänglich gemacht werden, sofern keine anders lautende Genehmigung der Ausbildungsstätte vorliegt. 
%\end{center}


% HIER EDITIEREN: Meta-Daten für PDF-Datei
\hypersetup{pdftitle={Titel der Arbeit}}
\hypersetup{pdfauthor={Ihr Name}}
\hypersetup{pdfsubject={Bachelorarbeit DHBW Stuttgart 2015}}

%%% Umstellung der Seiten-Nummerierung auf i, ii, iii ... %%%
\pagenumbering{Roman} 

%%% Abstract (optionale Kurzfassung Ihrer Arbeit) %%%
%\input{includes/abstract.tex}
%\cleardoublepage

\renewcommand{\listfigurename}{List of figures}
\renewcommand{\listtablename}{List of tables}

%%% Inhalts-, Abbildungs-, Tabellenverzeichnisse %%%
% sollen einzeilig gesetzt werden, um Platz zu sparen 
\begin{spacing}{1}
\tableofcontents
\clearpage
\chapter*{List of abbreviations}
\addcontentsline{toc}{chapter}{List of abbreviations}


\begin{acronym}[DHBW] 
% Argument definiert die Breite der ersten Spalte anhand des längsten vorkommenden Eintrags
\acro{CSS}{Cascading Style Sheet}
\acro{HTML}{Hypertext Markup Language}
\acro{IS}{Information Systems}
\acro{IT}{Information Technology}
\acro{JS}{JavaScript}
\acro{SHA}{Secure Hash Algorithm}
\end{acronym}

\clearpage
\thispagestyle{kapitelkopfzeile}
\listoffigures
\addcontentsline{toc}{chapter}{List of figures} % Abb.verz. ins Inh.verz. aufnehmen
\clearpage
\addcontentsline{toc}{chapter}{List of tables}   % Tab.verz. ins Inh.verz. aufnehmen
\listoftables
\end{spacing}

%%% Umstellung der Seiten-Nummerierung auf 1, 2, 3 ... %%%
\cleardoublepage
\pagenumbering{arabic}

%%% Ihr eigentlicher Inhalt %%%
% Empfehlung: strukturieren Sie Ihren Text in einzelnen Dateien 
% und binden Sie diese hier mit \input{includes/dateiname.tex} ein

\chapter{Introduction}
\label{chapter:Intro}

\textbf{Length: About 3 Pages!}

Blockchain technology has seen an enormous rise in interest in the recent years, coming to its climax at the end of 2017. The driving factor behind this is 1) Bitcoin, the most known cryptocurrency in the market and 2) 
Linear to the rising interest, articles trying to explain what lies behind the name have popped up in the thousands. Simple Google search requests for "what is bitcoin" and "what is blockchain" have peaked in December 2017, see figure \ref{fig:SearchRequests}, and yield 10.6 million and 3.8 million results. Of these results, the top ones claim to explain the technology in as little words as possible ("WTF is The Blockchain?
The ultimate 3500-word guide in plain English to understand Blockchain.",\footcite{MamoriaWTFBlockchain2017}, "Explain Bitcoin Like I'm 5"\footcite{CustodioExplainBitcoinFive2013}, "Blockchain: everything you want to know about the technology but were too embarrassed to ask"\footcite{HeathmannBlockchaineverythingyou2018} or "Still don't understand the blockchain? This explainer will help"\footcite{LeighSinodStilldonunderstand2018}) The internet offers A LOT of resources to learn about blockchain and the cryptocurrencies that build on top, but most of these articles only cover the very basics of the technology and do not point out its limits, thus (maybe unconsciously) taking part in causing the hype. Many (including big consulting companies such as Deloitte) claim that blockchain has dozens of potential applications in almost every major industry, (financial services, technology, media, and telecommunications, consumer and industrial products, life sciences and health care, public sector, energy and resources and horizontal applications(e.g. audit and cyber security)).\footcite{SchatskybitcoinBlockchaincoming2015} These claims are picked up from a hyped up management that not fully understands the technology and yet wants to utilize it -> because Hype.

\begin{figure}
    \centering
    \includegraphics[width=\textwidth]{latex-vorlage_v1.5/graphics/BCRQ.png}
    \caption[Google Trends results for "blockchain" search requests.]{Google Trends results for "blockchain" search requests.\footnotemark}
    \label{fig:SearchRequests}
\end{figure}
\footnotetext{https://trends.google.com/trends/explore?date=today\%205-y\&q=\%2Fm\%2F0138n0j1}

"Writing a description for this thing for general audiences is bloody hard. There's nothing to relate it to."\footcite{NakamotoReSlashdotSubmission2010}

https://moguldom.com/14060/100-of-the-best-quotes-on-bitcoin-and-blockchain/
“Every informed person needs to know about Bitcoin because it might be one of the world’s most important developments.”

–Leon Luow, Nobel Peace prize nominee 
\section{Problem statement} \label{sec:Problem}

The problem with this is however that many people suffer from the hype and solely want to establish a blockchain in their enterprise to show off. But in many cases, a blockchain is not necessarily the best option. This insight is quickly uncovered, if one deals with the topic a little more deeply than what most of the "what is blockchain" articles cover. Once one has reached a deeper level of understanding, blockchain's use cases are a lot more realistic and down-to-earth and one is aware of its limitations and where the technology is best applied. 

On the other hand, many people might be overwhelmed by the sheer mass of information/articles, that are out there regarding blockchain technology. One reason, as to why there are so many different attempts to explain the technology is because the authors feel that all the other articles do not correctly present the topic. The problem there is that the blockchain technology is indeed hard to explain, as it builds on a variety of different topics, such as cryptography, game theory, computer networking, data transmission as well as monetary theory.\footcite[Cf.][]{LoppNobodyUnderstandsBitcoin2017}

Show the motivation: Why is this topic important and interesting?

Describe the context of the problem: People don't understand blockchain and how transactions work, therefore they cannot imagine possible use cases as easily.

\section{Objective and scope} \label{sec:Objective}
\begin{itemize}
    \item Touch on the hype on Blockchain and take away the mystery/have people understand how it works and when it may be useful in order to not have inflated expectations.
    \item Say why I want to create an artifact to provide more details about blockchain technology
    \item Limit the scope of the research to only transactions (and one/a few selected coin/s(?))
    \item Articulate the research question that I want to explore in this thesis: How can blockchain transactions and smart contracts best be explained in a dashboard to persons that have only very limited knowledge about this topic?
    \item On basis of this research question, show which research method is going to be used for finding the answer.
    
    Objective should only be 2 to 3 sentences long!
    
    The study's research question is therefore: \textbf{What should the artifact visualize so it explains the blockchain technology in a comprehensible way?}
    
\end{itemize}

\section{Thesis overview} \label{sec:ThesisOverview}
How is this paper structured? What will be presented in the following chapters?



\chapter{Blockchain technology}
\textbf{circa 8 Seiten}

\section{What is Blockchain?} \label{sec:Blockchain}

\section{What are transactions?} \label{sec:TX}

Inspect transactions in detail, show how they are broadcast and put into blocks
-> only in ethereum or bitcoin or should I inspect the overall design of transaction by looking at different coins?

\section{What are smart contracts?} \label{sec:SmartContracts}

-> here also, only on ethereum or also others? What about legally binding smart contracts?

\chapter{Interactive multimedia instructions} \label{chapter:Multimedia}

Multimedia is defined as the joint presentation of both words (in verbal or written form) and pictures.\footcites[Cf.][p.2]{MayerMultimediaLearning2009}[cf.][p.1205]{MarraffinoApplyingMultimediaLearning2016}[cf.][p.13]{MayerAnimationAidMultimedia2001} The use of multimedia in instructions or multimedia learning environments is proven to allow people to learn more deeply and to promote better learning outcomes and transfer.\footcites[Cf.][p.3]{MayerMultimediaLearning2009}[cf. in addition][]{MunzerLearningmultimediapresentations2009} For this reason, the paper argues that the artifact design should take the following theory and derived principles into account to present a relevant, useful and meaningful learning resource. Since it appears that existing learning resources have not been proven successful, it is suggested that a theory-led approach results in a better learning experience. This chapter, therefore, presents Mayer's cognitive theory of multimedia learning and the guidelines which are derived from it to establish a basis of good design principles for the ensuing artifact design. Additionally, one section investigates which role interactivity plays during the learning process. 

\section{Mayer's cognitive theory of multimedia learning} \label{sec:MayersCTML}
While researching in the field of educational psychology for over 20 years, Richard E. Mayer has specialized in the area of multimedia design in educational psychology. As part of this research, he has developed a cognitive theory of multimedia learning regarding the mechanics in the human brain in different learning environments.\footcites[Cf.][]{MayerMultimediaLearning2009}
The theory, promoting a learner-centered approach (based on how the human brain works) to instructional design, presents an important contribution to educational psychology, and is one of the most recognized principles of instruction.\footcites[Cf.][chapter 1, paragraph 3]{ClarkElearningscienceinstruction2016}[cf.][p.4 et seqq]{MayerMultimediaLearning2009}[cf. in addition][]{SordenCognitiveTheoryMultimedia2012} Because of its proven reputation, the research relies on this theory to create an effective and useful explanation of the blockchain technology.
Mayer's theory is based on Sweller's cognitive load theory and the findings derived from it, which are briefly outlined in the following paragraph.

\paragraph{Cognitive load theory} Sweller developed this theory to create a framework for instructional design. It starts from the idea that human processing capacity is limited and that only few information items may be processed by working memory at a time.\footcites[Cf.][p.250]{SwellerCognitiveArchitectureInstructional1998}[cf.][p.490]{Gervenefficiencymultimedialearning2003} The author argues that the information imposed on the working memory (the cognitive load) must not exceed simple cognitive activities or be segmented in smaller chunks, otherwise overwhelming the brain and impeding schema construction (knowledge acquisition in the long-term memory).\footcites[Cf.][p.255 et seqq]{SwellerCognitiveArchitectureInstructional1998}[cf.][p.2]{SwellerVisualisationInstructionalDesign2002}[cf.][p.562]{Gervenefficiencymultimedialearning2003} Sweller therefore states that instructional design must take these working memory limitations into consideration. He also provides examples and guidelines to be accommodated in good design. For instance, he argues that instruction should be focused on domain-specific information rather than teaching general reasoning strategies.\footcites[Cf.][p.255]{SwellerCognitiveArchitectureInstructional1998}[cf.][p.301]{SwellerCognitiveloadtheory1994} Additionally, alternative problem statements are identified which allow the learners to adapt to new tasks: goal-free problems (\enquote{calculate all possible values}), worked examples (providing all steps for the problem solution), and completion problems (providing selected intermediate steps).
As part of this theory, Sweller has also identified a continuum of element interactivity which shows that if several information items need to be processed simultaneously, the working memory is under a high load, which in turn complicates understanding/learning of this material.\footcites[Cf.][p.261]{SwellerCognitiveArchitectureInstructional1998} Sweller and Mayer also have identified the following effects: the split-attention effect (separate presentation of pictures and words results in extraneous load), the redundancy effect (unnecessary extraneous load if both representations contain the same information), and the modality effect (use of different sensory processing channels allows deeper learning). Sweller argues that well-designed formats of multimedia instructions, which consider these effects, lead to a better acquisition of complex information than regular instructions (e.g., plain text), and thus enhance learning.\footcites[Cf.][p.4]{PaasCognitiveLoadTheory2004} What exactly constitutes learning and how it may be enhanced is discussed in the following.

\paragraph{Meaningful learning} Learning is broadly defined as the acquisition of knowledge.\footcites[Cf.][p.226]{MayerRotemeaningfullearning2002} This includes the strengthening of correct and weakening of incorrect responses, the addition of new information to the learner's memory (rote learning), and the making-sense of the presented material by attending to relevant information, mentally reorganizing it, and connecting it with what the learner already knows (meaningful learning).\footcites[Cf.][chapter 2, paragraph 5]{ClarkElearningscienceinstruction2016} Meaningful learning differentiates from rote learning in so far as meaningful learning allows the construction of mental schemata and transfer of these schemata to new situations.% For such construction to occur, instructions must go beyond the simple presentation of factual knowledge.
\footcites[Cf.][p.227]{MayerRotemeaningfullearning2002}[cf.][p.299]{SwellerCognitiveloadtheory1994} Meaningful learning on the basis of multimedia instructions occurs when the learner is engaged in the five cognitive processes\footnote{The five processes divide into three parallel processes: selecting, organizing and integrating} presented in figure \ref{fig:CTML_process}.\footcites[Cf.][p.111]{MayerCognitiveTheoryMultimedia1999}[cf.][p.35]{SordenCognitiveTheoryMultimedia2012}[cf.][p.43]{MayerNineWaysReduce2003} This process relies on three assumptions, which form the basis of Sweller's theory: 1) There are different channels for auditory and visual sensory input (dual processing); 2) There is only limited processing capacity available (limited capacity); and 3) Learning requires significant cognitive processing in both channels (active processing).\footcites[Cf.][p.44]{MayerNineWaysReduce2003}

Based on this theory, \cite{MayerNineWaysReduce2003} have identified several methods to reduce the cognitive load during instructional design. Cognitive load may be reduced by offloading (moving processing from visual to auditory channel), segmenting (dividing information into small segments), weeding (eliminating unnecessary information), signaling (providing cues for important information), aligning (placing words near corresponding graphics), eliminating redundancy (avoiding identical information), or by synchronizing (presenting narration and animation simultaneously).\footcites[Cf.][p.46]{MayerNineWaysReduce2003}

If these aspects are considered during instructional design, multimedia serves as a cognitive aid to knowledge construction enabling a deeper and more meaningful understanding of the presented information. It is needless to say that for such understanding to occur, instructions must go beyond the simple presentation of factual knowledge.\footcites[Cf.][p.229]{MayerRotemeaningfullearning2002}

\begin{figure}[!htb]
    \centering
    \includesvg[width=0.8\linewidth]{graphics/CTMLProcess}
    \caption[The cognitive processes for meaningful learning with multimedia instructions.]{The cognitive processes for meaningful learning with multimedia instructions.\protect\footnotemark}
    \label{fig:CTML_process}
\end{figure}
\footnotetext{With changes taken from \cite{MayerCognitiveTheoryMultimedia1999}, p.111}


\paragraph{Critique} Even though many studies have shown strong support for Mayer's thesis, others were not able to reproduce these findings. For instance, \cite{RaschInteractivenoninteractivepictures2009} did not see any improvements when comparing comprehension from written and graphical instructions with that from written instructions.\footcites[Cf.][]{RaschInteractivenoninteractivepictures2009} However, the researchers acknowledge these problems and continue to develop the theory and the utilized research techniques.\footcites[Cf.][p.52]{SordenCognitiveTheoryMultimedia2012} 

\section{The role of interactivity in the learning process} \label{sec:Interactivity}
Mayer focuses on the role of multimedia instructions. In addition to his cognitive theory of multimedia learning, Mayer has also presented further research (among other authors) arguing for interactivity to be considered during instructional design.\footcites[Cf.][chapter 2, paragraph 12]{ClarkElearningscienceinstruction2016} Interactivity refers to the possibility of user interaction with educational content.\footcites[Cf.][p.292]{PatwardhanWhendoeshigher2015} In the narrower context of multimedia instructions, this paper defines interactivity as the reciprocal activity between a learner and the learning environment, in which an action of one party triggers a response from the other party.\footcites[Cf.][p.1025]{DomagkInteractivitymultimedialearning2010} This kind of interactivity is suggested to support meaningful learning (understanding) by engaging the user in active learning, for example by manipulating virtual objects or by simulating experiments or industrial processes.\footcites[Cf.][p.161]{CairncrossInteractiveMultimediaLearning2001}[cf.][p.1159]{Evansinteractivityeffectmultimedia2007} Interactive features may be categorized into five different types: 1) Dialoguing (a learner receives questions and answers or feedback on their input); 2) Controlling (a learner controls the pace and/or the order of the presentation); 3) Manipulating (a learner is free to set parameters for a simulation, or to move objects around); 4) Searching (a learner may find new information by specifically querying for it); 5) Navigating (a learner moves to different content areas).\footcites[Cf.][p.311]{MorenoInteractiveMultimodalLearning2007}

Although interactivity has been heralded by many as one of the key features of multimedia learning to support the learning process, results have been inconclusive. Some studies support this hypothesis and report better scores in interactive multimedia learning environments, whereas others report the opposite suggesting that interactive features might actually hinder understanding.\footcites[Cf.][p.1024]{DomagkInteractivitymultimedialearning2010}[cf.][]{MayerWhenLearningJust2001}[cf.][p.156]{CairncrossInteractiveMultimediaLearning2001}[cf.][p. 1148 et seqq]{Evansinteractivityeffectmultimedia2007}[cf.][p.48]{SordenCognitiveTheoryMultimedia2012} These mixed results may originate from the different types of experiments conducted, the evaluation methods used to assess the differences in understanding, or from the different types of interactivity utilized. When designing multimedia instructions, interactive features should, therefore, be skillfully designed to draw learners attention to important pieces of information, to motivate and to help learners achieve an understanding of the material, and not to distract.\footcites[Cf.][p.21]{KirshInteractivitymultimediainterfaces1997}[cf.][p.15]{LeeScreenDesignGuidelines1999} For this purpose, \cite{MorenoInteractiveMultimodalLearning2007} have developed some design principles which will be presented in the following section.

\section{Guidelines for designing multimedia instructions} \label{sec:GuidelinesMultimediaInstr}
As this paper's aim is to design an easy to understand explanation of the blockchain technology, the utilized multimedia instructions must be well-designed in order to support meaningful learning. For this purpose, the paper relies on existing guidelines and best practices for instructional design. While different aspects, ranging from psychological and aesthetic questions to the distribution of the final instructions, need to be considered, this section primarily focuses on design principles that are derived from the cognitive theory of multimedia learning. Thereafter, general best practices regarding screen design and user experience are discussed which constitute the foundation for the ensuing artifact development.

\subsection{Principles derived from the cognitive theory of multimedia learning}
On the basis of the cognitive theory of multimedia learning, a number of principles are identified. Some of these may be mapped directly to the nine ways to reduce cognitive load. All principles presented below are empirically studied and proven to enhance the learning process, which is why their consideration during instructional design is of key importance.\footcites[Cf. for more detail][]{SordenCognitiveTheoryMultimedia2012}

\begin{enumerate}
    \item \textbf{Multimedia principle}: The first principle states that the combined presentation of graphics (images, animations, or visualizations) and words (written or spoken) is better than the presentation of a single material. This is because by presenting both materials, students build a mental connection of the two presentations, and hence engage in meaningful learning.\footcites[Cf.][p.19]{MayerAnimationAidMultimedia2001}[cf.][chapter 4, paragraph 7]{ClarkElearningscienceinstruction2016}[cf.][p.13]{MayerCognitiveTheoryMultimedia1999} It is important to consider that not all graphics are equally useful for this tasks. Only graphics that are relevant to the instructional purpose should be used (these may be organizational, transformational or interpretative graphics). Decorative graphics may even hurt understanding (additional cognitive load). Furthermore, the multimedia principle is especially important for novices (little to no prior knowledge), as experts already have existing mental schemata of the material.\footcites[Cf.][chapter 4, paragraphs 7 et seq]{ClarkElearningscienceinstruction2016}[cf. in addition][]{MayerWhenillustrationworth1990}
    \item \textbf{Contiguity principle}: By presenting words and pictures or by coordinating spoken words and graphics in proximity to each other (both in place and in time), students are said to learn more deeply. The rationale behind this principle is that students build mental schemata more easily if both information pieces are processed at the same time.\footcites[Cf.][p. 19 et seqq]{MayerAnimationAidMultimedia2001}[cf.][chapter 5, paragraphs 1 et seq]{ClarkElearningscienceinstruction2016} If screen space is limited, this may also be achieved by tooltips which appear when hovering over a specific piece of material.
    \item \textbf{Modality principle}: Also sometimes referred to as the split-attention principle, it states that words should be presented in spoken form rather than written form. This corresponds to the offloading method presented in \ref{sec:MayersCTML}, as some of the processing load is taken off the visual channel and transferred to the auditory channel.\footcites[Cf.][p.22]{MayerAnimationAidMultimedia2001}[cf.][p.14]{MayerCognitiveTheoryMultimedia1999}
    \item \textbf{Redundancy principle}: Based on the same rationale as the modality principle, this principle states that students learn more deeply when the same information is not presented in more than one format. This means that graphics should be explained with words in either audio or written format, but not both as redundant information causes additional extraneous load.\footcites[Cf.][chapters 6 and 7]{ClarkElearningscienceinstruction2016}[cf.][p.6]{MayerMultimediaLearning2009}[cf.][p.22]{MayerAnimationAidMultimedia2001}[cf. in addition][]{MayerPrinciplesreducingextraneous2014}
    \item \textbf{Coherence principle}: The coherence principle describes the fact that extra material (be it for interest, technical depth or to expand on key ideas) hurts meaningful learning. It suggests the designers should focus on concise descriptions and to delete any material that does not support the instructional goal. This principle relates to the weeding method from \ref{sec:MayersCTML}, as any extraneous words and pictures are removed to provide one coherent summary instead of a longer version.\footcites[Cf.][chapter 8, paragraphs 1 et seq]{ClarkElearningscienceinstruction2016}[cf.][p.6]{MayerMultimediaLearning2009}[cf.][p.22]{MayerAnimationAidMultimedia2001}
    \item \textbf{Personalization \& embodiment principle}: Mayer and Moreno suggest that by using effective on-screen coaches (embodiment) and polite wording in a conversational style (personalization) a feeling of social presence is created which stimulates students to more deeply engage in the learning process. 
    \item \textbf{Segmenting \& pretraining principle}: In situations involving complex material, it is helpful to manage the complexity by breaking the lesson into smaller parts (segmenting, see also segmenting in \ref{sec:MayersCTML}) and to ensure that learners know the names and characteristics of key concepts (pretraining).\footcites[Cf.][chapters 9 and 10]{ClarkElearningscienceinstruction2016}
    \item \textbf{Guided discovery principle}: The guided discovery principle states that students learn better when given an orientation in a discovery-based learning environment.\footcites[Cf.][p.7]{MayerMultimediaLearning2009}
    \item \textbf{Animation \& interactivity principle}: The animation and interactivity principle states that animations as well as interactive visualizations facilitate understanding because dynamic processes that may be difficult to imagine are visualized.\footcites[Cf.][p.290]{Betrancourtanimationinteractivityprinciples2005}[cf.][p.81]{MunzerLearningmultimediapresentations2009}[cf.][p.19]{LeeScreenDesignGuidelines1999}[cf.][p.814]{MayerNineWaysReduce2003} However, as stated earlier (see \ref{sec:Interactivity}), animations and interactive visualizations are not always helpful, and hence, their presentation should be skillfully designed. For this purpose, \cite{MorenoInteractiveMultimodalLearning2007} have defined more detailed principles regarding guided activity, reflection, feedback, and pacing.\footcites[Cf.][p.292]{PatwardhanWhendoeshigher2015}[cf.][p.7]{MayerMultimediaLearning2009}[cf.][p.316]{MorenoInteractiveMultimodalLearning2007} % \begin{enumerate}
    %     \item \textbf{Guided activity:} A pedagogical agent who helps guide students' cognitive processing is helpful for meaningful learning.
    %     \item \textbf{Reflection}: By asking students to reflect upon answers, the process of sense-making is facilitated.
    %     \item \textbf{Feedback}: Students learn better with explanatory rather than corrective feedback.
    %     \item \textbf{Pacing}: If allowed to control the pace of presentation of the learning material, students are allowed to adapt the presentation to their processing rate, thus facilitating learning.
    % \end{enumerate}
    \item \textbf{Site map principle}: Students learn better if the learning environment contains a map showing the learner's progress.\footcites[Cf.][p.7]{MayerMultimediaLearning2009}
  \item \textbf{Individual differences principle}: This principle relativizes the afore-described principles in so far as it states that multimedia, contiguity and split-attention effects depend on individual differences in the learners.\footcites[Cf.][p.15]{MayerCognitiveTheoryMultimedia1999}
\end{enumerate}
By following these principles, instructional designers may create learning environments that minimize extraneous loads, effectively managing the brain's limited processing capacity, and thus foster meaningful learning.\footcites[Cf.][chapter 2, paragraph 6]{ClarkElearningscienceinstruction2016}

\subsection{Best practices for instructional design} \label{subsec:BestPracticesDesign}
As the listed principles apply to all multimedia instructions, they need to be considered during the artifact design process. However, they do not cover all aspects of instructional design, which is why the next paragraphs add to these principles by providing additional guidelines for instructional design.

\paragraph{Preliminary work and student engagement} Before the instructions are designed, their objectives should be defined. By defining the target group, analyzing the potential users' needs and identifying the system requirements, this preliminary work allows the designers to adapt the learning environment in such a way that it may better engage students, designing it to be more relevant to their needs. Additionally, student engagement may be enhanced by pursuing active learning strategies (e.g., focused-is-more approach: giving guidance to motivate meaningful learning), or by promoting flexibility (with modules or layers). For the latter, the use of clear navigational techniques (tooltips, guided tours, or defined learning outcomes) is suggested.\footcites[Cf.][p.202]{BlummerBestPracticesCreating2009}

\paragraph{Cueing} Cueing is defined as giving visual cues to the learner. This directs the learner's attention to a specific part of the material. Cueing serves three functions: 1) guide the attention to facilitate the selection of essential information; 2) emphasize major parts of instruction and their organization; and 3) foster integration by making relationships between information elements more visible. Cues may be implemented by animating or increasing the luminance of an element. Even though cues are shown to capture attention effectively, this does not necessarily improve understanding. The implementation of such features should always be considered under the light of the principles mentioned above (e.g., coherence).\footcites[Cf.][p.114 et seqq]{deKoningFrameworkAttentionCueing2009}[cf.][chapter 2, paragraph 14]{ClarkElearningscienceinstruction2016}

\paragraph{User experience} For a good user experience, the instructions should satisfy the principle of visibility. This means that users ought to see available actions, receive immediate feedback from executed actions and get timely information about the consequences of these actions. Additionally, the environment should be aesthetically pleasing, which may be achieved by eliminating clutter (offloading, weeding), selecting matching colors, consistent style and use of elements, as well as by following the general rules of visual composition (balance, symmetry, unity, harmony).\footcites[Cf.][p. 16 et seqq]{LeeScreenDesignGuidelines1999}[cf.][p. 16 et seqq]{Nadelhoffer10BestPractices}[cf.][p.20]{KirshInteractivitymultimediainterfaces1997}

% Regarding the distribution of the created multimedia instructions, the format of online learning environments offers the highest accessibility. Computers are a well-relied on medium today as they represent one of the most flexible and widespread media options and by providing the instructions in the form of a web application, anybody with computer access may view and learn from it.\footcites[Cf.][chapter 1, paragraphs 6 et seq]{ClarkElearningscienceinstruction2016}[cf.][p.906]{BaharinEvaluationSatisfactionUsing2015}

This chapter has presented the fundamental principles that are considered during the artifact design. Such a theory-led design approach sets a theoretical foundation for any design decisions. However, the designing of the artifact is still a creative process which relies heavily on human creativity which theoretical guidelines may not (or only arduously) represent.\footcites[Cf.][p.7]{VaishnaviDesignScienceResearch}
\chapter{Discussion of Design Science}
\textbf{circa 8 Seiten}

Building on the theoretic understanding of blockchain and dashboard design, the following chapter sets the foundation of the dashboard design process. Since the dashboard is to be understood as an information system whose creation underlies a creative process, a design science approach is deemed to be the best way to methodologically/on a scientific basis conduct the research. This chapter therefore presents design science research as a research method in Information Systems. The first paragraph establishes design science as a method in the Information Systems discipline. The ensuing pages will describe the method in more detail; the chapter will describe the design cycles and process steps of design science in order to finally examine the utilized dashboard design process.


\section{Design Science in the context of Information Systems Research}

Information Systems is by nature a design-oriented science. The object of scientific studies is the design information systems in business and society contexts.\footcite[Cf.][p.671]{OsterleMemorandumzurgestaltungsorientierten2010} These information systems are sociotechnical systems that comprise people (any person interacting with the information system), organizations (functions, management and business processes) and technology (anything useful and practical embodied in an implement or artifact. The term technology should be understood as an expression of intelligence, including the tools, techniques and sources of power that humans have developed to reach their goals) as well as their relationship to each other.\footcites[Cf.][p.98]{HevnerDesignScienceResearch2004}[cf.][p.11]{OsterleGestaltungsorientierteWirtschaftsinformatikPladoyer2010}[cf.][p.252]{MarchDesignnaturalscience1995}
% As technology is an essential part of most information systems, it is important According to March and Smith, technology should be understood as an expression of intelligence, including the tools, techniques and sources of power that humans have developed to reach their goals.\footcite[Cf.][p.252]{MarchDesignnaturalscience1995} 

The approach to research in Information Systems is multi-faceted. It builds upon observation, experimenting, systems development, and theory building to contribute to research. These four approaches are also necessary to inspect the different aspects of a research question in Information Systems.\footcite[Cf.][p.86]{NunamakerSystemsdevelopmentInformation1991} This multi-methodological approach to research can be seen in the output of Information Systems research. Objective of ISR is to create constructs, models, methods, or instances, that were unknown to the community before or that have not yet existed.\footcites[Cf.][p.12]{OsterleGestaltungsorientierteWirtschaftsinformatikPladoyer2010}[cf.][p.130]{ThomasBekannteundweniger2014} 
Irrespective of the form of the result, the ultimate goal of Information Systems Research is to create normative artifacts which solve problems.\footcite[Cf.][p.130]{ThomasBekannteundweniger2014}

As these artifacts are produced in a complex environment, where people and organizations are important variables, a solution cannot be found in a deterministic manner. The success and validity of the conducted research therefore cannot be formally proven (or very rarely), instead it depends on the acceptance in the Information systems community.\footcite[Cf.][p.671]{OsterleMemorandumzurgestaltungsorientierten2010}

\paragraph{Design Science}
The idea of design as a science was first conceptualized by Simon in 1996\footcite[Cf.][]{Simonsciencesartificial1996} as a research paradigm to create innovative artifacts which solve real-world problems.\footcite[Cf.][p.9]{HevnerDesignResearchInformation2010}
Much of the relevant literature regarding design science state relevance and utility for the practice to be of high importance and emphasize the creation of artifacts as a key element of the research.\footcites[Cf.][p.253, 254]{MarchDesignnaturalscience1995}[cf.][p.9, 11]{HevnerDesignResearchInformation2010}[cf.][p.77]{HevnerDesignScienceResearch2004}[cf.][p.1]{PapalambrosDesignScienceWhy2015}[cf.][p.330,342]{GregorPositioningpresentingdesign2013} Design science can therefore be defined as a problem-solving paradigm\footcite[Cf.][p.77]{HevnerDesignScienceResearch2004} that designs, creates and evaluates innovative IT artifacts\footcite[Cf.][p.90]{HevnerDesignScienceResearch2004} by focusing human creativity\footcite[Cf.][p.13]{HevnerDesignResearchInformation2010} on real-world problems to create value.\footcite[Cf.][p.1]{PapalambrosDesignScienceWhy2015} As emphasized by \cite{HevnerDesignResearchInformation2010}, it is important to differentiate \textit{design as research} from \textit{researching design} as well as from \textit{routine design}. Design science as \textit{design as research} creates new knowledge and contributes to the knowledge base by designing innovative artifacts. Even though the artifacts may be strongly influenced by their specific problem domain and there may be more research needed to generalize the research results, they are contributions to the knowledge base.\footcite[Cf.][p.15]{HevnerDesignResearchInformation2010} In contrast, \textit{researching design} sets the focus on the methods of designing, mainly situated in the disciplines of engineering, product design and architecture \footcite[Cf.][p.15]{HevnerDesignResearchInformation2010} and \textit{routine design} misses the aspect of creating new knowledge as it is more a creative process that already has accumulated any necessary knowledge.\footcite[Cf.][p.16]{HevnerDesignResearchInformation2010}

A brief, although less comprehensive definition is given by \cite{VaishnaviDesignScienceResearch} (p.6), who define Design Science as \enquote{Learning through building artifacts}. Similarly, \cite{MarchDesignnaturalscience1995} (p.254) characterize Design Science by their products: \enquote{artifacts and artificial phenomena}. Interestingly, even though there is a general agreement on the overall definition of design science and the production of artifacts, the definition of (IT) artifact is source of many discussions. For some writers the IT artifact implies only technical artifacts, while for others it also includes social artifacts.
The original definition by \cite{Simonsciencesartificial1996} (p.5) names four characteristics of an artifact.
\begin{enumerate}
    \item Artifacts, or artificial things, are created by human beings.
    \item Artifacts may reflect or imitate existing phenomena and natural things.
    \item Artifacts can be described by their terms of functions, objectives and adaptation.
    \item Artifacts are considered in terms of imperatives as well as descriptives. This means that the artifact is designed to attain certain objectives as well as to function in a specific way.  
\end{enumerate}

From this definition, the term IT artifact was born. However, there is no clear definition of what exactly an IT artifact is. While some writers argue that an IT artifact is the application of IT within a certain context to enable or support some tasks\footcites[Cf.][p.186]{BenbasatEmpiricalresearchinformation1999}[cf.][p.50]{AlterconceptITartifact2015}, others understand the term as a collection of things, such as materials and cultural characteristics, in form of hardware and/or software that can be socially recognized.\footcite[Cf.][p.121]{OrlikowskiResearchCommentaryDesperately2001} This has led to a hasty use of IT artifact describing a variety of concepts in Information Systems.\footcite[Cf.][p.49]{AlterconceptITartifact2015} Calls are made to retire the term IT artifact and to replace it with more precise terms\footcite[Cf.][p.59]{AlterconceptITartifact2015} or to create an IS artifact that consists of a technology artifact, an information artifact as well as a social artifact(in order to set the focus away from IT and instead back to IS).\footcite[Cf.][pp.1,6]{LeeGoingbackbasics2015} This paper does not participate in the above discussion and uses the term IT artifact in accordance with \cite{HevnerDesignScienceResearch2004} to describe something purposeful, innovative, and man-made supporting problem-solving and organizational capabilities by providing intellectual as well as computational tools.\footcites[Cf.][pp.76,82]{HevnerDesignScienceResearch2004}[cf.][p.340]{GregorPositioningpresentingdesign2013}
The IT artifacts that are produced while conducting design science research can be categorized as constructs, models, methods, instantiations as well as frameworks, architectures, design principles and design theories.\footcites[Cf.][pp.256-258]{MarchDesignnaturalscience1995}[cf.][p.343]{GregorPositioningpresentingdesign2013}[cf.][p.50]{PuraoDesignResearchTechnology2002} [cf.][p.77]{HevnerDesignResearchInformation2010} They are described in more detail in table \ref{tab:Artifacts}. 

\begin{table}
\setlength\extrarowheight{2pt} % for a bit of visual "breathing space"
  \centering
  \begin{tabularx}{\textwidth}{|l|X|l|}
    \hline
        \textbf{Output} & \textbf{Description} & \textbf{Example}  \\ \hline\hline
        Constructs & Conceptual description of problems in form of a specialized language and shared knowledge & Vocabulary, Symbols \\
        Models & Representations and sets of propositions or statements expressing relationships among constructs. & Abstractions, Syntax \\
        Frameworks & Real or conceptual guides to serve as support or guide & \\
        Architectures & High level structures of systems & \\ 
        Design Principles & Core principles and concepts to guide design & Design rules \\
        Methods & Sets of steps (based on constructs and models) used to perform tasks & Algorithms, guidelines \\
        Instantiations & Realization of an artifact in its environment, operationalization of constructs, models, and methods to demonstrate feasibility and effectiveness & Systems, products \\
        Design theories & A prescriptive set of statement on how to do something to achieve a certain goal. Theory includes other abstract artifacts (constructs, models, frameworks, architectures, design principles and methods) & \\ \hline
    \end{tabularx}
    \caption[The different outputs of Design Science.]{The different outputs of Design Science.\footnotemark }
    \label{tab:Artifacts}
\end{table}
\footnotetext{adapted from \cite{MarchDesignnaturalscience1995} and \cite{VaishnaviDesignScienceResearch}.}

Artifacts also may be described by their type of research contribution. Artifacts of level 1 are instantiations as they present a situated implementation of an artifact and therefore contribute more specific, limited and less mature knowledge to the knowledge base. Level 2 describes artifacts such as constructs, methods, models, and design principles, that contribute knowledge in form of operational principles or architectures, summarized as nascent design theory. The third level refers to well-developed design theory about embedded phenomena that constitute more abstract, complete and mature knowledge. Artifacts of this level are design theories.\footcite[Cf.][p.340]{GregorPositioningpresentingdesign2013}

\paragraph{Contributing to the knowledge base} Gregor and Hevner have drawn a distinction between prescriptive and descriptive knowledge in the existing knowledge base. Prescriptive knowledge refers to insights that are already available. Constructs, models, methods, instantiations and design theory all present prescriptive knowledge. Descriptive knowledge, in contrast, describes certain phenomena (natural, artificial or human) and includes natural laws as well as patterns, principles, and theories to make sense of the overall environment.\footcite[Cf.][p.344]{GregorPositioningpresentingdesign2013} On the basis of this distinction, Gregor and Hevner have also identified a knowledge contribution framework in design science. Depending on the solution maturity and the application domain maturity, a produced artifact can be classified as a routine design, an improvement, an invention or an exaptation. Routine design happens when both solution and application domain show high levels of maturity. There is no major knowledge contribution as known solutions are applied to already known problems.\footcite[Cf.][p.348]{GregorPositioningpresentingdesign2013} Even though it is listed in the matrix, see figure X???, it should not be understood as research but rather as professional design. Improvements are characterized by a low solution maturity and a high application domain maturity. The goal of DSR in this quadrant is to create better solutions for known problems and thus to contribute to the prescriptive knowledge base in the form of artifacts (no matter what level).\footcite[Cf.][p.346]{GregorPositioningpresentingdesign2013} Inventions happen when the solution and the application domain are only poorly understood and new solutions for new problems are discovered. Research contributions in this field are exceptionally new artifacts (but it can also be the articulation of an unknown problem itself). Knowledge that is contributed can be prescriptive and/or descriptive.\footcite[Cf.][pp.345,346]{GregorPositioningpresentingdesign2013} The fourth quadrant, exaptation, is characterized by a high degree of solution maturity and a low application domain maturity. This means that existing design knowledge is extended to new problems. Research in this quadrant creates prescriptive knowledge in the form of artifacts but may also create descriptive knowledge if the used artifacts are understood in greater detail.\footcite[Cf.][p.347]{GregorPositioningpresentingdesign2013}

Improvement, Invention and Exaptation all are valid research opportunities that contribute to the knowledge base. One should however refrain from routine design in DSR as there will be no major knowledge contribution.


To conclude, you can say that design science research produces artifacts that contribute to the knowledge base even if they are very specific to a certain problem or application domain since they may embody design ideas and theories that are yet to be articulated.\footcite[Cf.][p.340]{GregorPositioningpresentingdesign2013}

\section{What is Design Science Research?} 
\begin{itemize}
    \item Put it in the context of a mixed method approach X
    \item For what kind of research/problems is it used? X
    \item What is its result? X
    \item Discuss IT Artifact/IS Artifact X
    \item Show that it has two sides (behaviour sciences and...?)
    \item Show the two kinds of knowledge (presciptive and descriptive) X
    \item Demonstrate the ISR Framework
\end{itemize}


\section{What are its components/cycles/guidelines?}
\begin{itemize}
    \item Show the 7 step approach
    \item Show Pfeffer's approach
    \item Show the engineering and research cycle
\end{itemize}

\section{What does the design process of the dashboard look like?}
\begin{itemize}
    \item Argue why design science research is the appropriate method for this problem
    \item Classify the presented steps from above in the context of the dashboard design
\end{itemize}
\chapter{Problem identification} \label{chapter:Problem}
\chapter{Solution definition} \label{chapter:Solution}
\textbf{roughly 3 pages}

\textit{Show what the artifact is supposed to do}
The objectives of the solution have to be inferred rationally from the problem statement. Therefore the problem should be atomized conceptually, so that the solution can capture its complexity.



\section{Objectives}

\subsection{Dashboard content}
\textit{What exactly should the dashboard visualize?}
State the requirements that come from the requirements analysis

\subsection{Technology for dashboard development}
\begin{itemize}
    \item What should the technology be able to do? What are the requirements? (Should come from the interviews)
    \item What possible technologies are there to build a dashboard? 
    \item Compare these techs with respect to the formulated requirements and choose a solution/combination(?)
\end{itemize}


\chapter{Design and development}
\textbf{roughly 15 pages}
\section{Design}
\subsection{Research method and objective}
On the basis of the presented methods, this subsection will present the requirements for the dashboard.

\begin{itemize}
    \item the requirements will be extracted from expert interviews
    \begin{itemize}
        \item Why an interview?
        \item What kind of interview? explorative expert interview
        \item What experts? Frank Bloch, Bernd Kammholz, Torsten Milsmann, Daniel Kaltenbach, Thomas Maier -> all experts in their respective field
        \item What questions should I ask them? What do you not understand about blockchain? What would you wish to understand the most? Where do you feel lies the difficulty to blockchain and its transactions? What do you imagine under such a term as a dashboard visualizing blockchain transactions? What would be the best way to learn about it? Do you want it to be portable?
    \end{itemize}
    \item the interviews will be analyzed using the Qualitative Content Analysis of Mayring
    \begin{itemize}
        \item Why this method and not a quantitative method or different qualitative? -> Steigleider citation about how it is both the broadest and most exact technique. 
        \item How does the analysis work? Which process steps are there?
        \item Choose a summarizing content analysis and explain the steps for this
    \end{itemize}
\end{itemize}

\subsection{Data collection and analysis}
\subsubsection{Conducting the interviews}

\subsubsection{Analyzing the interviews}

\subsection{Requirements for the dashboard content}
\textit{What exactly should the dashboard visualize?}

\subsection{Technology for the dashboard development} \label{subsec:Technology}
\begin{itemize}
    \item What should the technology be able to do? What are the requirements?
    \item What possible technologies are there to build a dashboard?
    \item Compare these techs with respect to the formulated requirements and choose a solution/combination(?)
\end{itemize}

\section{Development}
\chapter{Implementation and Evaluation}

\section{Exemplary demonstration: Onboarding new students} \label{sec:demo}
circa eine Seite

demonstration serves as a light-weight evaluation to prove the feasibility, that the artifact solves one of the problem instantiations

The demonstration is done for new students that are going to ework in the department of StudyLab which treats the blockchain technology. 

\section{Evaluation} \label{sec:Evaluation}

The exemplary demonstration lends itself excellently for a more thorough evaluation of the created artifact. As stated by \cite{HevnerDesignScienceResearch2004} and agreed upon by the design science research community, such evaluation plays a \enquote{crucial} role in design science research. This importance is further emphasized by \cite{MarchDesignnaturalscience1995} who regard evaluation as one of two activities in design science (see \ref{subsec:GuidelinesDesignScience}).\footcites[Cf.][p.258]{MarchDesignnaturalscience1995}[Cf. in addition][]{PfeffersDesignScienceResearch2007}{HevnerDesignScienceResearch2004}{Pries-HejeComprehensiveFrameworkEvaluation2012}{Pries-HejeStrategiesDesignScience} Once the (first iteration of the) build-activity is concluded, a rigorous evaluation of the artifact and its qualities must take place in order to provide evidence that the artifact achieves its purpose for which it was designed, i.e. that it solves the initially identified problem. \footcites[Cf.][p.425]{Pries-HejeComprehensiveFrameworkEvaluation2012} The evaluation should furthermore also assess how well the artifact solves the problem. This is done by first identifying the possible evaluation criteria candidates. These include - in addition to the in guideline 3 \textit{Design Evaluation} described properties of a design artifact (utility, quality, and efficacy) - other properties such as Checkland and Scholes five E's (efficiency, effectiveness, efficacy, ethicality, elegance) or an artifact's purposefulness and implementability.\footcites[Cf.][p.427]{Pries-HejeComprehensiveFrameworkEvaluation2012}[cf. in addition][]{ChecklandSoftSystemsMethodology1990}
Overall, the candidates of evaluation criteria should be chosen in such a way that they adequately identify the relevant strengths and short comings of the artifact so that its utility may be adequately assessed. At first, focusing on certain criteria and disregarding others may appear to inhibit a rigorous approach that is necessary to ensure the scientific nature of the artifact. But this selection of important criteria actually ensures something of similar importance, the relevance of the applied methods with regard to the overall relevance of the design science research.\footnote{For more information about the interplay of relevance and rigor in design science research, see \ref{topic:relevance cycle}}

% While it is important to rigorously conduct the relevant evaluation methods in order to ensure the scientific nature of the design science research, In other words, even though rigorous evaluation methods must be executed to ensure the design science process is of scientific nature, the evaluation should nonetheless focus on the important properties in order to remain relevant.

However, independent of the evaluation criteria, there exist a wide number of possible evaluation methods to choose from: Hevner et al. identified twelve different methods which may be put in the following categories: observational, analytical, experimental, testing, and descriptive methods.\footcites[Cf.][p.86]{HevnerDesignScienceResearch2004} Yet, these do not represent the entire list of relevant evaluation methods available to design science researchers. As \cite{PfeffersDesignScienceResearch2012} found out, the majority of researchers conduct technical experiments,  followed by illustrative scenarios and case studies. However, the method chosen depends largely on the type of artifact and its qualities to be evaluated. In the case of information systems research, more qualitative approaches such as case studies or subject-based experiments are pursued as these are deemed more relevant.\footcites[Cf.][p.4 et seq]{PfeffersDesignScienceResearch2012}

The following paragraphs introduce therefore the evaluation method for this artifact as well as the strategy followed to identify the most appropriate evaluation method before presenting and discussing the results of that evaluation. 
%validity, utility, quality, efficacy; Purposeful? and implementability?

\subsection{Method} \label{subsec:EvaluationMethod}

While Hevner et al. have identified a number of different evaluation methods, they have not offered any guidance on the choice of appropriate evaluation strategies. It is for this reason that Pries-Heje et al. have proposed a framework for evaluation in design science research which presents the scientific basis for this evaluation process.\footcites[Cf.][p.11 et seq]{Pries-HejeComprehensiveFrameworkEvaluation2012}


\paragraph{Why was this evaluation method chosen?} -> Ex-Post strategy, naturalistic, 
\footcite{PfeffersDesignScienceResearch2012} \cite{Pries-HejeComprehensiveFrameworkEvaluation2012}

In the ex post perspective, a chosen system or technology is evaluated after it is acquired or implemented. p.3, 
\cite{Pries-HejeStrategiesDesignScience}

\paragraph{Participants}
Gruppen von 8 Personen aufgeteilt in 2 Lager -> eine Gruppe sieht sich die zwei meistgeschauten YouTube-Videos an und die andere durchläuft das Tutorial und die Overview.
Bei der Auswahl werden die Fragen zur Bestimmung der Zielgruppe gestellt. Nur, wenn sie in die Zielgruppe fallen, kommen sie als Teilnehmer infrage.

\paragraph{Materials}
die Videos + kurze Beschreibung;
meine Webseite;
Fragebogen
\paragraph{Procedure}
Vor Beginn wird der Fragebogen durchlaufen, nicht alle Fragen, aber die für den Vergleich schon.
Die Testpersonen sehen sich alle gleichzeitig die Videos sowie das Tutorial an.
Sofort nach Ende der Durchführung werden die Teilnehmer gebeten an ihren Arbeitslaptops den Fragebogen in schriftlicher Form auszufüllen. 


\subsection{Results} \label{subsec:EvaluationResults}
- Scores and answers

\subsection{Discussion}

\section{Identification of possible optimizations} \label{sec:Optimizations}

How could the artifact be refined? As Design Science Research is an iterative process -> will lead to further research for the future.

\chapter{Conclusion}

\textbf{Length: 3-4 pages!}

\section{Purpose of the thesis} \label{sec:findings}

mention rigor and relevance

\section{Critical Reflection} \label{sec:Reflection}
Evaluation hat sich nur auf eine Instanz des Problemes bezogen. Wie sähe es aus mit der Wirksamkeit, wenn das Tutorial bei einer Messe angewendet wird?
Evaluation hätte anders aufgebaut sein können: mehr Teilnehmer, aus anderen Alters-/Personengruppen, andere Methode der Evaluation (zwar schon eine naturalistic ex post, aber eventuell ein Experiment, oder eine Fokusgruppe), längerer Zeitraum um auch zu sehen ob das im Langzeitgedächtnis eingegangen ist, Fragen vielleicht nicht eindeutig genug um daraus schlüsse zu ziehen

Definition der Anforderungen -> auf Basis der Interviews und bestehender Visualisierungen (aber keine Videos oder Artikel, obwohl die ja auch oft gelesen werden), zu wenig Interview-Partner, selektives Sampel ist nicht repräsentativ, 

Methodik: großer Fokus auf Mayer's cognitive load theory. Ist zwar sehr weit anerkannt, aber es gibt auch andere Theorien die hier nicht beachtet wurden. In dem Feld wird noch sehr viel geforscht und auch nicht alle Ergebnisse bezüglihc Multimedia und Interaktivität bezeugen eindeutig dass sie beim Lernen helfen.

Design Science: Artifakt ist "bloß" eine Instantiation, eventuell zu spezifisch auf den einen Kontext bezogen. Bzw. es gibt schon eine große Vielfalt an interaktiven Webseiten, welche Informationen visualisieren und beibringen, also nicht so eine krasse innovative Lösung -> aber für den spezifischen Fall schon.
Fokus lag klar auf der Relevanz, eventuell hat darunter Rigor geleidet (Evaluation und der Design Prozess hätten rigoroser durchgeführt werden können)

\section{Key findings}

\section{Implications for practice and research} \label{Implications}

\section{Future research} \label{sec:FutureResearch}

%%% Ende des eigentlichen Inhalts %%%


%%% Quellenverzeichnisse (keine Anpassungen nötig) %%%
\clearpage
\chapter*{Bibliography}\label{chapter:quellen}
\addcontentsline{toc}{chapter}{Bibliography}
\spezialkopfzeile{Bibliography}

% alle Quellen außer Gespräche:
\printbibliography[heading=lit,nottype=online,notkeyword=gespraech,notkeyword=ausblenden]

% alle Gespräche (diese müssen vom Typ "misc" sein und unter keyword "gespraech" stehen haben)
\printbibliography[heading=gespraeche,type=misc,keyword=gespraech]

%%% Ehrenwörtliche Erklärung (keine Anpassungen nötig) %%%
% steht ganz am Ende des Dokuments
\cleardoublepage
\clearpage

\thispagestyle{kapitelkopfzeile}

\chapter*{Declaration}
\addcontentsline{toc}{chapter}{Declaration}

% \typMeinerArbeit und \themaMeinerArbeit werden in deckblatt.tex definiert
I hereby ensure that I have personally authored my \typMeinerArbeit\ with the topic: \emph{\themaMeinerArbeit} and have used no sources and aids other than those indicated.

I also ensure that the submitted electronic version corresponds to the printed version. 

\vspace{3cm}

\begin{center}
\begin{tabular}{ccc}
(Place, Date) & \hspace{0.3\linewidth} & (Signature)
\end{tabular}
\end{center}

\end{document}